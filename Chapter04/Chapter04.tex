\chapter {Waves in Cold  Plasmas}\label{s4}
\section{Introduction}
The cold-plasma equations describe waves, and other perturbations, which
propagate through a plasma {\em much faster}\/ than a typical thermal velocity.
It is instructive to consider the relationship between the collective
motions described by the cold-plasma model and the motions of individual
particles that we studied in Sect.~\ref{s2}. The key observation is that in the
cold-plasma model all  particles (of a given species) at a given position effectively move with the
same velocity. It follows that the fluid velocity is identical to the particle
velocity, and is, therefore, governed by the same equations. 
However, the cold-plasma
model goes beyond the single-particle description because it determines the
electromagnetic fields {\em self-consistently}\/ in terms of the
charge and current densities generated by the motions of the 
constituent particles of the  plasma. 

What role, if any, does the geometry of the plasma equilibrium play in
determining the properties of plasma waves? Clearly, geometry plays a
key role for modes whose wave-lengths are comparable to the dimensions of
the plasma. However, we shall show that modes whose wave-lengths are 
{\em much
smaller}\/ than the plasma dimensions 
have properties which are, in a local sense, {\em independent}\/ of the
geometry. Thus, the local properties of small-wave-length oscillations are 
{\em universal}\/ in nature. To investigate these properties, we
may, to a first approximation, represent the plasma as a homogeneous
equilibrium (corresponding to the limit $k\,L\rightarrow 0$, where $k$
is the magnitude of the wave-vector, and $L$ is the characteristic
equilibrium length-scale). 

\section{Plane Waves in a Homogeneous Plasma}\label{s4.2}
The propagation of small amplitude waves is described by {\em linearized equations}.
These are obtained by expanding the equations of motion in powers
of the wave amplitude, and neglecting terms of order higher than unity.
In the following, we use the subscript 0 to distinguish equilibrium
quantities from perturbed quantities, for which we retain the previous
notation. 

Consider a homogeneous, quasi-neutral  plasma, consisting of equal
numbers of electrons and ions, in which both plasma species are at rest.
It follows that ${\bf E}_0 = {\bf 0}$, and ${\bf j}_0 =\nabla\times
{\bf B}_0 = {\bf 0}$. In a homogeneous medium, the
general solution of a system of linear equations can be constructed as
a superposition of plane wave solutions:
\begin{equation}\label{e4.1}
{\bf E} ({\bf r}, t) = {\bf E}_{\bf k} \,\exp[\,{\rm i}\,({\bf k}
\!\cdot\!{\bf r} - \omega\, t)],
\end{equation}
with similar expressions for ${\bf B}$ and ${\bf V}$. The
surfaces of constant phase,
\begin{equation}
{\bf k}\!\cdot\!{\bf r} - \omega\, t = {\rm constant},
\end{equation}
are planes perpendicular to ${\bf k}$, traveling at the velocity
\begin{equation}
{\bf v}_{\rm ph} = \frac{\omega}{k}\,\hat{\bf k},
\end{equation}
where $k\equiv |{\bf k}|$, and $\hat{\bf k}$ is a unit vector
pointing in the direction of ${\bf k}$. Here, ${\bf v}_{\rm ph}$
is termed the {\em phase-velocity}.
Henceforth, we shall omit
the subscript ${\bf k}$ from field variables, for ease of notation.


Substitution of the plane wave solution (\ref{e4.1}) into Maxwell's equations yields:
\begin{eqnarray}\label{e4.4a}
{\bf k}\times{\bf B} &=& - {\rm i}\,\mu_0\,{\bf j} - \frac{\omega}{c^2}\,{\bf E},
\\[0.5ex]
{\bf k}\times{\bf E} &=& \omega\,{\bf B}.\label{e4.4b}
\end{eqnarray}
In linear theory, the current is related to the electric field via
\begin{equation}\label{e4.5}
{\bf j} = \bsigma\!\cdot\!{\bf E},
\end{equation}
where the {\em conductivity tensor}\/ $\bsigma$ is a
function of both ${\bf k}$ and $\omega$. Note that the conductivity
tensor is {\em anisotropic}\/ in the presence of a non-zero  equilibrium
magnetic field. Furthermore, $\bsigma$ completely specifies the
plasma response. 

Substitution of Eq.~(\ref{e4.5}) into Eq.~(\ref{e4.4a}) yields
\begin{equation}\label{e4.6}
{\bf k}\times{\bf B} = -\frac{\omega}{c^2}\,{\bf K}\!\cdot\!{\bf E},
\end{equation}
where we have introduced the {\em dielectric permittivity tensor},
\begin{equation}\label{e4.7}
{\bf K} = {\bf I} + \frac{{\rm i}\,{\bsigma}}{\epsilon_0\,\omega}.
\end{equation}
Here, ${\bf I}$ is the identity tensor. Eliminating the
magnetic field between Eqs.~(\ref{e4.4b}) and (\ref{e4.6}), we obtain 
\begin{equation}\label{e4.8}
{\bf M}\!\cdot\!{\bf E} = {\bf 0},
\end{equation}
where
\begin{equation}\label{e4.9}
{\bf M} = {\bf k}{\bf k} - k^2\,{\bf I} + \frac{\omega^2}{c^2}\,
{\bf K}.
\end{equation}

The solubility condition for Eq.~(\ref{e4.9}),
\begin{equation}
{\cal M} (\omega, {\bf k}) \equiv {\rm det}({\bf M}) = 0,
\end{equation}
is called the {\em dispersion relation}. The dispersion relation
relates the frequency, $\omega$, to the wave-vector, ${\bf k}$. 
Also, as the name
``dispersion relation''  indicates, it allows us to determine the rate at which the
different Fourier components in a wave-train {\em disperse}\/ due to
the variation of their phase-velocity with wave-length. 

\section{Cold-Plasma Dielectric Permittivity}\label{s4.3}
In a collisionless plasma, the linearized cold-plasma equations
are written [see Eqs.~(\ref{e3.162a})--(\ref{e3.162c})]:
\begin{eqnarray}
m_i n\,\frac{\partial {\bf V}}{\partial t} &=& {\bf j}\times{\bf B}_0,\\[0.5ex]
{\bf E} &=& - {\bf V}\times{\bf B}_0 +\frac{{\bf j}\times{\bf B}_0}{ne}
+\frac{m_e}{ne^2}\frac{\partial{\bf j}}{\partial t}.
\end{eqnarray}
Substitution of plane wave solutions of the type (\ref{e4.1}) into the above
equations yields
\begin{eqnarray}\label{e4.12a}
-{\rm i}\,\omega\,m_i n\,{\bf V} &=& {\bf j}\times{\bf B}_0,\\[0.5ex]
{\bf E} &=& - {\bf V}\times{\bf B}_0 +\frac{{\bf j}\times{\bf B}_0}{ne}
-{\rm i}\,\omega\,\frac{m_e}{ne^2}\,{\bf j}.\label{e4.12b}
\end{eqnarray}
Let
\begin{eqnarray}
{ \Pi}_e &=& \sqrt{\frac{n\,e^2}{\epsilon_0\,m_e}},\label{e4.13a}\\[0.5ex]
{ \Pi}_i &=& \sqrt{\frac{n\,e^2}{\epsilon_0\,m_i}},\\[0.5ex]
{\Omega}_e &=& -\frac{e\,B_0}{m_e},\\[0.5ex]
{\Omega}_i &=& \frac{e\,B_0}{m_i},\label{e4.13d}
\end{eqnarray}
be the {\em electron plasma frequency}, the {\em ion plasma frequency},
the {\em electron cyclotron frequency}, and the {\em ion cyclotron frequency},
respectively. The ``plasma frequency,'' $\omega_p$, mentioned in Sect.~\ref{s1}, is 
identical
to the electron plasma frequency, ${ \Pi}_e$. Eliminating the
fluid velocity ${\bf V}$ between Eqs.~(\ref{e4.12a}) and (\ref{e4.12b}), and making
use of the above definitions, we obtain
\begin{equation}
{\rm i}\,\omega\,\epsilon_0\,{\bf E} =
\frac{ \omega^2\,{\bf j} 
 - {\rm i}\,\omega\,{\Omega}_e\,{\bf j}\times
{\bf b} + {\Omega}_e\,{\Omega}_i\,{\bf j}_\perp}{{\Pi}_e^{~2}}.
\end{equation}

The parallel component of the above equation is readily solved to give
\begin{equation}\label{e4.15}
j_\parallel = \frac{{\Pi}_e^{~2}}{\omega^2}\,({\rm i}\,\omega\,\epsilon_0\,
E_\parallel).
\end{equation}
In  solving for ${\bf j}_\perp$, it is helpful to define the
vectors:
\begin{eqnarray}
{\bf e}_+ &=& \frac{ {\bf e}_1 +{\rm i}\,{\bf e}_2}{\sqrt{2}},\\[0.5ex]
{\bf e}_- &=& \frac{ {\bf e}_1 - {\rm i}\,{\bf e}_2}{\sqrt{2}}.
\end{eqnarray}
Here, $({\bf e}_1, {\bf e}_2, {\bf b})$ are a
set of  mutually orthogonal, right-handed
unit vectors. Note that
\begin{equation}
{\bf b}\times {\bf e}_\pm  =  \mp {\rm i}\,{\bf e}_\pm.
\end{equation}
It is easily demonstrated that
\begin{equation}\label{e4.18}
j_\pm = \frac{{\Pi}_e^{~2}}
{\omega^2 \pm \omega\,{\Omega}_e
+ {\Omega}_e\,{\Omega}_i}\,{\rm i}\,\omega\,\epsilon_0\,E_\pm,
\end{equation}
where $j_\pm = {\bf j}\cdot {\bf e}_\pm$, {\em etc.}

The conductivity tensor is {\em diagonal}\/ in the basis $({\bf e}_+, {\bf e}_-,{\bf b})$. Its elements are given by the coefficients of $E_\pm$ and
$E_\parallel$ in Eqs.~(\ref{e4.18}) and (\ref{e4.15}), respectively. Thus,
the dielectric permittivity (\ref{e4.7}) takes the form
\begin{equation}
{\bf K}_{\rm circ} = \left(\begin{array}{ccc}
R&0&0\\
0&L&0\\
0&0&P\end{array}
\right),
\end{equation}
where
\begin{eqnarray}\label{e4.20a}
R &\simeq & 1 - \frac{{\Pi}_e^{~2}}
{\omega^2+\omega\,{\Omega}_e + {\Omega}_e\,{\Omega}_i},\\[0.5ex]
L&\simeq & 1 - \frac{{\Pi}_e^{~2}}
{\omega^2-\omega\,{\Omega}_e+
{\Omega}_e\,{\Omega}_i },\\[0.5ex]
P &\simeq & 1- \frac{{\Pi}_e^{~2}}{\omega^2}.\label{e4.20c}
\end{eqnarray}
Here, $R$ and $L$ represent the permittivities for right- and left-handed
circularly polarized waves, respectively. The permittivity parallel to the
magnetic field, $P$, is identical to that of an unmagnetized plasma. 

In fact, the above expressions are only approximate, because the small
mass-ratio ordering $m_e/m_i\ll 1$ has already been folded into the
cold-plasma equations. The exact expressions, which are most easily
obtained by solving the individual charged particle equations
of motion, and then summing  to obtain the
fluid response, are:
\begin{eqnarray}\label{e4.21a}
R &=& 1 - \frac{{\Pi}_e^{~2}}{\omega^2}
\!\left(\frac{\omega}{\omega + {\Omega}_e}\right)
 -\frac{{\Pi}_i^{~2}}{\omega^2}\!
\left(\frac{\omega}{\omega + {\Omega}_i}\right),\\[0.5ex]
L &=& 1 - \frac{{\Pi}_e^{~2}}{\omega^2}\!
\left(\frac{\omega}{\omega -{\Omega}_e}\right)
 -\frac{{\Pi}_i^{~2}}{\omega^2}\!
\left(\frac{\omega}{\omega -{\Omega}_i}\right),\\[0.5ex]
P&=& 1 - \frac{{\Pi}_e^{~2}}{\omega^2}-\frac{{\Pi}_i^{~2}}{\omega^2}.\label{e4.21c}
\end{eqnarray}
Equations (\ref{e4.20a})--(\ref{e4.20c}) and (\ref{e4.21a})--(\ref{e4.21c}) are equivalent in the limit $m_e/m_i\rightarrow 0$. 
Note that Eqs.~(\ref{e4.21a})--(\ref{e4.21c}) generalize in a fairly obvious manner in plasmas
consisting of more than two particle species. 

In order to obtain the actual dielectric permittivity, it is necessary
to transform back to the Cartesian basis $({\bf e}_1, {\bf e}_2, {\bf b})$.
Let ${\bf b} \equiv {\bf e}_3$, for ease of notation. It follows that
the components of an arbitrary vector ${\bf W}$ in the Cartesian basis are
related to the components in the ``circular'' basis via
\begin{equation}
\left(\!\begin{array}{c} W_1 \\ W_2 \\ W_3 \end{array}\! \right) = 
{\bf U} \!\left(\!\begin{array}{c} W_+ \\ W_- \\ W_3 \end{array}\! \right),
\end{equation}
where the unitary matrix ${\bf U}$ is written
\begin{equation}
{\bf U} = \frac{1}{\sqrt{2}}\left(\begin{array}{ccc}
1&1&0\\
{\rm i} & -{\rm i}&0\\
0&0&\sqrt{2}\end{array}
\right).
\end{equation}
The dielectric permittivity in the Cartesian basis is then
\begin{equation}
{\bf K} = {\bf U}\, {\bf K}_{\rm circ}\, {\bf U}^\dag.
\end{equation}
We obtain
\begin{equation}\label{e4.36x}
{\bf K} = \left(\begin{array}{ccc}
S&-{\rm i}\,D&0\\
{\rm i}\,D & S&0\\
0&0&P\end{array}
\right),
\end{equation}
where 
\begin{equation}
S =\frac{R+L}{2},
\end{equation}
and
\begin{equation}
D = \frac{R-L}{2},
\end{equation}
represent the sum and difference of the right- and left-handed dielectric
permittivities, respectively.

\section{Cold-Plasma Dispersion Relation}
It is convenient to define a vector
\begin{equation}
{\bf n} = \frac{{\bf k}\,c}{\omega},
\end{equation}
which points in the same direction as the wave-vector, ${\bf k}$, 
and whose  magnitude
$n$  is the {\em refractive index}\/ ({\em i.e.}, the ratio of the
velocity of light in vacuum to the phase-velocity). Note that $n$ should not be
confused with the particle density. 
Equation~(\ref{e4.8}) can be rewritten
\begin{equation}\label{e4.29}
{\bf M}\cdot{\bf E} =({\bf n}\cdot{\bf E})\,{\bf n} - n^2\, {\bf K}\!\cdot\!{\bf E} = {\bf 0}.
\end{equation}

We may, without loss of generality, assume that the equilibrium
magnetic field is directed along the $z$-axis, and that the wave-vector,
${\bf k}$, lies in the $xz$-plane. Let $\theta$ be the angle subtended between
${\bf k}$ and ${\bf B}_0$. The eigenmode equation (\ref{e4.29}) can be written
\begin{equation}\label{e4.30}
\left(\!\begin{array}{ccc}
S - n^2\,\cos^2\theta & -{\rm i}\,D & n^2\,\cos\theta\,\sin\theta\\
{\rm i} \,D           & S - n^2     & 0 \\
n^2\,\cos\theta\,\sin\theta & 0 & P - n^2\,\sin^2\theta
\end{array}\!\right)\left(\!
\begin{array}{c} E_x\\ E_y \\ E_z \end{array}\!\right) = {\bf 0}.
\end{equation}
The condition for a nontrivial solution is that the determinant of
the square matrix be zero. With the help of the identity
\begin{equation}
S^2 - D^2 \equiv R\,L,
\end{equation}
we find that
\begin{equation}\label{e4.32}
{\cal M}(\omega,{\bf k}) \equiv A\,n^4- B\,n^2 + C = 0,
\end{equation}
where
\begin{eqnarray}\label{e4.33a}
A &=& S\,\sin^2\theta + P\,\cos^2\theta,\label{e4.33b}\\[0.5ex]
B &=& R\,L\,\sin^2\theta + P\,S\,(1+\cos^2\theta),\\[0.5ex]\label{e4.33c}
C &=& P\,R\,L.
\end{eqnarray}

The dispersion relation (\ref{e4.32}) is evidently a quadratic in $n^2$, with
two roots. 
The solution can be written
\begin{equation}
n^2 = \frac{B\pm F}{2\,A},
\end{equation}
where
\begin{equation}
F^2 = (R\,L - P\,S)^2\,\sin^4\theta + 4\,P^2 \,D^2\,\cos^2\theta.
\end{equation}
Note that $F^2\geq 0$. It follows that $n^2$ is always real, which implies
that $n$ is either purely real or purely imaginary. In other words, the
cold-plasma dispersion relation describes waves which either propagate
without evanescense, or decay  without spatial oscillation. 
The two roots
of opposite sign for $n$, corresponding to a particular root for $n^2$, simply describe
 waves of the same type propagating, or decaying, in opposite directions. 

The dispersion relation (\ref{e4.32}) can also be written
\begin{equation}\label{e4.36}
\tan^2\theta = -\frac{P\,(n^2-R)\,(n^2-L)}{(S\,n^2 - R\,L)\,(n^2-P)}.
\end{equation}
For the special case of wave propagation {\em parallel}\/ to the magnetic
field ({\em i.e.}, $\theta=0$), the above expression reduces to
\begin{eqnarray}
P&=& 0,\label{e4.37a}\\[0.5ex]
n^2 &=& R,\\[0.5ex]
n^2 &=& L.\label{e4.37c}
\end{eqnarray}
Likewise, for the special case of propagation {\em perpendicular}\/ to the
field ({\em i.e.}, $\theta=\pi/2$), Eq.~(\ref{e4.36}) yields
\begin{eqnarray}
n^2 &=& \frac{R\,L}{S},\\[0.5ex]
n^2 &=& P.
\end{eqnarray}

\section{Polarization}
A pure right-handed circularly polarized wave propagating along the
$z$-axis takes the form
\begin{eqnarray}
E_x &=& A\,\cos(k\,z-\omega \,t),\\[0.5ex]
E_y &=& -A\,\sin (k\,z-\omega \,t).
\end{eqnarray}
In terms of complex amplitudes, this becomes
\begin{equation}
\frac{{\rm i}\,E_x}{E_y} = 1.
\end{equation}
Similarly, a left-handed circularly polarized wave is characterized by
\begin{equation}
\frac{{\rm i}\,E_x}{E_y} = -1.
\end{equation}

The polarization of the transverse electric field is obtained from the
middle line of Eq.~(\ref{e4.30}):
\begin{equation}
\frac{{\rm i}\,E_x}{E_y} = \frac{n^2 -S}{D} = \frac{2n^2 - (R+L)}{R-L}.
\end{equation}
For the case of parallel propagation, with $n^2 = R$, the above formula
yields ${\rm i}\,E_x/E_y = 1$. Similarly, for the case of parallel propagation,
with $n^2 = L$, we obtain ${\rm i}\,E_x/E_y = -1$. Thus, it is clear that
the roots $n^2=R$ and $n^2=L$ in Eqs.~(\ref{e4.37a})--(\ref{e4.37c}) correspond to
right- and left-handed circularly polarized waves, respectively. 

\section{Cutoff and Resonance}
For certain values of the plasma parameters, $n^2$ goes to zero
or infinity. In both cases, a transition is made from a region 
of propagation to a region of evanescense, or {\em vice versa}. 
It will be demonstrated later on that {\em reflection}\/ occurs 
wherever $n^2$ goes through zero, and that {\em absorption}\/ takes place 
wherever $n^2$ goes through infinity. The former case is called a
wave {\em cutoff}, whereas the latter case is termed a wave {\em resonance}. 

According to Eqs.~(\ref{e4.32}) and (\ref{e4.33a})--(\ref{e4.33c}), cutoff occurs when
\begin{equation}
P=0,
\end{equation}
or
\begin{equation}
R=0,
\end{equation}
or
\begin{equation}
L=0.
\end{equation}
Note that the cutoff points are independent of the direction of
propagation of the wave relative to the magnetic field.

According to Eq.~(\ref{e4.36}), resonance takes place when
\begin{equation}
\tan^2\theta = -\frac{P}{S}.
\end{equation}
Evidently, resonance points do depend on the direction of propagation
of the wave relative to the magnetic field. For the case of
parallel propagation, resonance occurs whenever $S\rightarrow \infty$.
In other words, when
\begin{equation}
R\rightarrow \infty,
\end{equation}
or
\begin{equation}
L\rightarrow \infty.
\end{equation}
For the case of perpendicular propagation, resonance takes place
when
\begin{equation}
S = 0.
\end{equation}

\section{Waves in an Unmagnetized Plasma}\label{s4.7}
Let us now investigate the cold-plasma dispersion relation in detail. It is
instructive to first consider the limit in which the equilibrium magnetic
field goes to zero. In the absence of the magnetic field, there is
no preferred direction, so we can, without loss of generality,
assume that ${\bf k}$ is directed along the $z$-axis ({\em i.e.}, $\theta=0$).
In the zero magnetic field limit ({\em i.e.}, ${\Omega}_e, {\Omega}_i
\rightarrow 0$), the eigenmode equation  (\ref{e4.30}) reduces to
\begin{equation}\label{e4.50}
\left(\!\begin{array}{ccc}
P-n^2 & 0 & 0\\
0 & P-n^2 & 0 \\
0 & 0 & P\end{array}\! \right)\left(\!\begin{array}{c} E_x\\E_y\\ E_z\end{array}
\!\right) = {\bf 0},
\end{equation}
where
\begin{equation}
P \simeq 1 - \frac{{\Pi}_e^{~2}}{\omega^2}.
\end{equation}
Here, we have neglected ${\Pi}_i$ with respect to ${\Pi}_e$. 

It is clear from Eq.~(\ref{e4.50}) that there are two types of wave.
The first possesses
the eigenvector $(0,0,E_z)$, and has the dispersion relation
\begin{equation}
 1- \frac{{\Pi}_e^{~2}}{\omega^2} = 0.
\end{equation}
The second possesses the eigenvector $(E_x, E_y, 0)$, and has the dispersion
relation
\begin{equation}\label{e4.53}
1 - \frac{ {\Pi}_e^{~2}}{\omega^2} - \frac{k^2\,c^2}{\omega^2} = 0.
\end{equation}
Here, $E_x$, $E_y$, and $E_z$ are arbitrary non-zero quantities. 

The first wave has ${\bf k}$ parallel to ${\bf E}$, and is, thus, a
{\em longitudinal}\/ wave. This wave is know as the {\em plasma wave}, and
possesses the fixed frequency $\omega = {\Pi}_e$. Note that if
${\bf E}$ is parallel to ${\bf k}$ then it follows from Eq.~(\ref{e4.4b})
that ${\bf B} = {\bf 0}$. In other words, the wave is purely {\em electrostatic}\/
in nature. In fact, a plasma wave is an electrostatic oscillation of the type
discussed in Sect.~\ref{s1.5}.
Since $\omega$ is independent of ${\bf k}$,  the {\em group
velocity},
\begin{equation}\label{e4.54}
{\bf v}_g = \frac{\partial\omega}{\partial {\bf k}},
\end{equation}
associated with a plasma wave, 
is zero. As we shall demonstrate later on, the group velocity is the propagation
velocity of localized wave packets. It is clear that the plasma wave
is {\em not}\/ a propagating wave, but instead has the property than an oscillation
set up in one region of the plasma remains localized in that region. It
should be noted, however,  that in a ``warm'' plasma ({\em i.e.}, a plasma with a finite
thermal velocity) the plasma wave acquires a non-zero,
albeit very small, group velocity (see Sect.~\ref{s6.2}).

The second wave is a {\em transverse}\/ wave, with ${\bf k}$ perpendicular to 
${\bf E}$. There are two independent linear polarizations of this wave,
which propagate at identical velocities, 
just like a vacuum electromagnetic wave. The dispersion relation
(\ref{e4.53}) can be rearranged to give
\begin{equation}
\omega^2 = {\Pi}_e^{~2} + k^2 c^2,
\end{equation}
showing that this wave is just the conventional electromagnetic wave,
whose vacuum dispersion relation is $\omega^2=k^2 c^2$, modified by
the presence of the plasma. An important property, which follows
immediately from the above expression, is that for the propagation of
this wave we need $\omega\geq {\Pi}_e$. Since ${\Pi}_e$ is
proportional to the square root of the plasma density, it follows that
electromagnetic radiation of a given frequency will only propagate through
a plasma when the plasma density falls below a critical value. 

\section{Low-Frequency Wave Propagation}\label{s4.8}
Let us now consider wave propagation through a magnetized plasma
at frequencies far below the ion cyclotron or plasma frequencies,
which are, in turn, well below the corresponding electron frequencies.
In the low-frequency regime ({\em i.e.}, $\omega\ll {\Omega}_i, {\Pi}_i$),
we have [see Eqs.~(\ref{e4.20a})--(\ref{e4.20c})]
\begin{eqnarray}
S &\simeq & 1 +\frac{{\Pi}_i^{~2}}{{\Omega}_i^2},\\[0.5ex]
D &\simeq & 0,\\[0.5ex]
P &\simeq & -\frac{{\Pi}_e^{~2}}{\omega^2}.
\end{eqnarray}
Here, use has been made of ${\Pi}_e^{~2}/{\Omega}_e{\Omega}_i
=-{\Pi}_i^{~2}/ {\Omega}_i^{~2}$. 
Thus, the eigenmode equation (\ref{e4.30}) reduces to 
\begin{equation}\label{e4.57}
 \left(\!\begin{array}{ccc}
{\scriptstyle 1+{\Pi}_i^{~2}/{\Omega}_i^2 - n^2\,\cos^2\theta} & 
0& {\scriptstyle n^2\,\cos\theta\,\sin\theta}\\
0 & {\scriptstyle  1+{\Pi}_i^{~2}/{\Omega}_i^2 - n^2 }    & 0 \\
{\scriptstyle n^2\,\cos\theta\,\sin\theta }& 0 & {\scriptstyle -{\Pi}_e^{~2}/\omega^2
 - n^2\,\sin^2\theta}
\end{array}\!\right)\left(\!
\begin{array}{c} E_x\\ E_y \\ E_z \end{array}\!\right) =
{\bf 0}.
\end{equation}

The solubility condition for Eq.~(\ref{e4.57}) yields the dispersion relation
\begin{equation}
\left| \!\begin{array}{ccc}
 1+{\Pi}_i^{~2}/{\Omega}_i^2 - n^2\,\cos^2\theta & 
0& n^2\,\cos\theta\,\sin\theta\\
0 &   1+{\Pi}_i^{~2}/{\Omega}_i^2 - n^2     & 0 \\
n^2\,\cos\theta\,\sin\theta & 0 & -{\Pi}_e^{~2}/\omega^2
 - n^2\,\sin^2\theta
\end{array}\!\right| = 0.
\end{equation}
Note that in the low-frequency ordering, ${\Pi}_e^{~2}/\omega^2\gg
{\Pi}_i^{~2}/{\Omega}_i^{~2}$. Thus, we can see that the bottom
right-hand   element of the above determinant  is far larger than any of the other
elements, so to a good approximation the roots of the dispersion relation 
are obtained
by equating the term multiplying this large factor to zero. In this manner,
we obtain two roots:
\begin{equation}
n^2\,\cos^2\theta = 1 + \frac{{\Pi}_i^{~2}}{{\Omega}_i^{~2}},
\end{equation}
and
\begin{equation}
n^2 = 1 +  \frac{{\Pi}_i^{~2}}{{\Omega}_i^{~2}}.
\end{equation}

It is fairly easy to show, from the definitions of the plasma and cyclotron
frequencies [see Eqs.~(\ref{e4.13a})--(\ref{e4.13d}], that
\begin{equation}
\frac{{\Pi}_i^{~2}}{{\Omega}_i^{~2}} = \frac{c^2}{B_0^{~2}/\mu_0\rho}
= \frac{c^2}{V_A^{~2}}.
\end{equation}
Here, $\rho\simeq n\,m_i$ is the plasma mass density, and
\begin{equation}
V_A = \sqrt{\frac{B_0^{~2}}{\mu_0\,\rho}}
\end{equation}
is called the {\em Alfv\'{e}n velocity}. Thus, the dispersion relations
of the two low-frequency waves can be written
\begin{equation}\label{e4.63}
\omega = \frac{k\,V_A\,\cos\theta}{\sqrt{1+V_A^{~2}/c^2}}\simeq k\,V_A\,\cos\theta
\equiv k_\parallel\,V_A,
\end{equation}
and
\begin{equation}\label{e4.64}
\omega = \frac{k\,V_A}{\sqrt{1+V_A^{~2}/c^2}}\simeq k\,V_A.
\end{equation}
Here, we have made use of the fact that $V_A\ll c$ in conventional plasmas. 

The dispersion relation (\ref{e4.63}) corresponds to the {\em slow}\/ or
{\em shear}\/ Alfv\'{e}n wave, whereas the dispersion relation (\ref{e4.64})
corresponds to the {\em fast}\/ or {\em compressional}\/ Alfv\'{e}n wave. 
The fast/slow terminology simply refers to the ordering of  the
phase velocities of  the two waves. The shear/compressional
terminology refers to the velocity fields associated with the waves. In
fact, it is clear from Eq.~(\ref{e4.57}) that $E_z=0$ for both waves, whereas
$E_y=0$ for the shear wave, and $E_x=0$ for the compressional wave. 
Both waves are, in fact,  MHD modes which satisfy the linearized MHD Ohm's law
[see Eq.~(\ref{e3.147})]
\begin{equation}
{\bf E} + {\bf V} \times {\bf B}_0 = 0.
\end{equation}
Thus, for the shear wave
\begin{equation}
V_y = - \frac{E_x}{B_0},
\end{equation}
and $V_x=V_z=0$, whereas for the compressional wave
\begin{equation}
V_x = \frac{E_y}{B_0},
\end{equation}
and $V_y=V_z=0$. Now $\nabla\!\cdot\!{\bf V} =
{\rm i}\,{\bf k}\!\cdot\!{\bf V} = {\rm i}\,k\,V_x\,\sin\theta$.
Thus, the shear-Alfv\'{e}n wave is a torsional  wave, with zero
divergence of the flow, whereas the  compressional wave involves a
non-zero divergence of the flow. It is important to realize that the
thing which is resisting compression  in the compressional wave is the
magnetic field, not the plasma, since there is negligible plasma pressure in
the cold-plasma approximation.

\begin{figure}
\epsfysize=2.5in
\centerline{\epsffile{Chapter04/shear_Alfven.eps}}
\caption{\em Magnetic field perturbation associated with a shear-Alfv\'en
wave.}\label{f10}
\end{figure}

Figure~\ref{f10} shows the characteristic distortion of the magnetic field 
associated with a shear-Alfv\'{e}n wave propagating parallel to the
equilibrium field. Clearly, this wave bends magnetic field-lines without
compressing them. Figure~\ref{f11} shows the characteristic distortion of the
 magnetic field 
associated with a compressional-Alfv\'{e}n wave propagating perpendicular to the
equilibrium field. Clearly, this wave compresses magnetic field-lines without
bending them.

\begin{figure}
\epsfysize=2.5in
\centerline{\epsffile{Chapter04/Comp_Alfven.eps}}
\caption{\em Magnetic field perturbation associated with a compressional
Alfv\'{e}n-wave.}\label{f11}
\end{figure}

It should be noted that the thermal velocity is not necessarily  negligible
compared to the Alfv\'{e}n velocity in conventional plasmas. Thus,
we can expect the dispersion relations (\ref{e4.63}) and (\ref{e4.64}) to
undergo considerable modification in a ``warm'' plasma (see Sect.~\ref{s5.4}). 

\section{Parallel Wave Propagation}
Let us now consider wave propagation, at arbitrary frequencies, {\em parallel}\/
to the equilibrium magnetic field. When $\theta=0$, the eigenmode equation
(\ref{e4.30}) simplifies to
\begin{equation}\label{e4.68}
\left(\!\begin{array}{ccc}
S - n^2 & -{\rm i}\,D & 0\\
{\rm i} \,D           & S - n^2     & 0 \\
0& 0 & P 
\end{array}\!\right)\left(\!
\begin{array}{c} E_x\\ E_y \\ E_z \end{array}\!\right) = {\bf 0}.
\end{equation}
One obvious way of solving this equation  is to have 
\begin{equation} 
P \simeq 1 - \frac{{\Pi}_e^{~2}}{\omega^2} = 0,
\end{equation}
with the eigenvector $(0,0,E_z)$. This is just the {\em electrostatic plasma wave}
which we found previously in an unmagnetized plasma. This mode is
longitudinal in nature, and, therefore, causes particles to
oscillate {\em parallel} to ${\bf B}_0$. It follows that the particles
experience zero Lorentz force due to the presence of the equilibrium magnetic
field, with the result that this field has no effect on the mode dynamics. 

The other two solutions to Eq.~(\ref{e4.68}) are obtained by setting the $2\times 2$
determinant involving the $x$- and $y$- components of the electric
field to zero. The first wave has the dispersion relation
\begin{equation}\label{e4.70}
n^2 = R \simeq 1 - \frac{{\Pi}_e^{~2}}{(\omega+{\Omega}_e)
(\omega+{\Omega}_i)},
\end{equation}
and the eigenvector $(E_x, {\rm i}\,E_x, 0)$. This is evidently
a {\em right-handed}\/ circularly polarized wave. 
The second wave has the dispersion relation
\begin{equation}\label{e4.71}
n^2 = L\simeq 1 - \frac{{\Pi}_e^{~2}}{(\omega-{\Omega}_e)
(\omega-{\Omega}_i)},
\end{equation}
and the eigenvector $(E_x, -{\rm i}\,E_x, 0)$. This is evidently
a {\em left-handed}\/ circularly polarized wave. At low frequencies
({\em i.e.}, $\omega\ll {\Omega}_i$), both waves tend to the
Alfv\'{e}n wave found previously. Note that the fast and
slow Alfv\'{e}n waves are indistinguishable for parallel
propagation. Let us now examine the high-frequency behaviour
of the right- and left-handed waves. 

For the right-handed wave, it is evident, since ${\Omega}_e$ is
negative, that $n^2\rightarrow\infty$ as $\omega\rightarrow|{\Omega}_e|$.
This resonance, which corresponds to $R\rightarrow\infty$, 
 is termed the {\em electron cyclotron resonance}. 
At the electron cyclotron  resonance the transverse electric field 
associated with a right-handed
wave rotates at the same velocity, and in the same direction, as electrons
gyrating around the equilibrium magnetic field. Thus, the electrons
experience a {\em continuous}\/ acceleration from the electric field, which tends
to increase their perpendicular energy. It is, therefore, not surprising that
right-handed waves, propagating
parallel to the equilibrium magnetic field,
and oscillating at the frequency ${\Omega}_e$, are {\em absorbed}\/ by 
electrons.

When $\omega$ is just above $|{\Omega}_e|$, we find that $n^2$ is negative,
and so there is no wave propagation in this frequency range. However, for frequencies
much greater than the electron cyclotron or plasma frequencies, the solution
to Eq.~(\ref{e4.70}) is approximately $n^2=1$. In other words, $\omega^2=k^2 c^2$: the
dispersion relation of a right-handed vacuum electromagnetic wave. 
Evidently, at some frequency above $|{\Omega}_e|$ the solution
for $n^2$ must pass through zero, and become positive again. 
Putting $n^2=0$ in Eq.~(\ref{e4.70}), we find that the equation reduces to
\begin{equation}
\omega^2+{\Omega}_e\,\omega- {\Pi}_e^{~2}\simeq 0,
\end{equation}
assuming that $V_A\ll c$. The above equation has only
one positive root, at $\omega=\omega_1$, where
\begin{equation}\label{e4.73}
\omega_1 \simeq |{\Omega}_e|/2 + \sqrt{{\Omega}_e^{~2}/4
+{\Pi}_e^{~2}}>|{\Omega}_e|.
\end{equation}
Above this frequency, the wave propagates once again.

\begin{figure}
\epsfysize=4in
\centerline{\epsffile{Chapter04/parr.eps}}
\caption{\em Dispersion relation for a right-handed wave propagating
parallel to the magnetic field in a magnetized plasma.}\label{f12}
\end{figure}

The dispersion curve for a right-handed wave propagating parallel to
 the equilibrium
magnetic field is sketched in Fig.~\ref{f12}. The continuation of the Alfv\'{e}n
wave above the ion cyclotron frequency is called the {\em electron cyclotron
wave}, or sometimes  the {\em whistler wave}. The latter terminology is prevalent
in ionospheric and space plasma physics contexts. The wave which propagates
above the cutoff frequency, $\omega_1$, is a standard right-handed
circularly polarized electromagnetic wave, somewhat modified by the
presence of the plasma. Note that the low-frequency branch of the
dispersion curve differs fundamentally from the high-frequency branch, because
the former branch corresponds to a wave which can only propagate through the
plasma in the presence of an equilibrium magnetic field, whereas the high-frequency
branch corresponds to a wave which can propagate in the absence of an equilibrium
 field.

\begin{figure}
\epsfysize=4in
\setbox0=\hbox{\epsffile{Chapter04/IonWhistler.ps}}
\centerline{\rotr0}
\caption{\em Power spectrum of a typical whistler wave.}\label{f13}
\end{figure}

The curious name ``whistler wave'' for the branch of the dispersion relation
lying between the ion  and electron cyclotron frequencies is originally
derived from ionospheric physics. Whistler waves are a very characteristic
type  of 
audio-frequency radio interference, most commonly encountered at high
latitudes, which take the form of brief,
intermittent pulses, starting at high frequencies, and rapidly descending in pitch.
Figure~\ref{f13} shows the power spectra of some typical whistler waves. 

Whistlers were discovered in the early days of radio communication, but
were not explained until much later. Whistler waves start off as ``instantaneous''
 radio
pulses, generated by lightning flashes at high latitudes. The pulses are
channeled along the Earth's dipolar magnetic field, and eventually return
to ground level in the opposite hemisphere. Fig.~\ref{f14} illustrates the typical
path of a whistler wave. Now, in the frequency
range ${\Omega}_i\ll \omega\ll |{\Omega}_e|$, the dispersion
relation (\ref{e4.70}) reduces to
\begin{equation}
n^2 = \frac{k^2\,c^2}{\omega^2}\simeq \frac{{\Pi}_e^{~2}}
{\omega\,|{\Omega}_e|}.
\end{equation}
As is well-known, pulses propagate  at the group-velocity,
\begin{equation}
v_g = \frac{d\omega}{dk} = 2c\,\frac{\sqrt{\omega\,|{\Omega}_e|}}{{\Pi}_e}.
\end{equation}
Clearly, the low-frequency components of a pulse propagate {\em more slowly}\/
than the high-frequency components. It follows that by the time a
pulse returns to ground level it has been stretched out temporally, because the
high-frequency components of the pulse arrive slightly before the low-frequency components. This also accounts for the characteristic whistling-down effect observed at
ground level.

The shape of whistler pulses, and the way in which 
the pulse frequency varies in time,
can yield a considerable amount of information about the regions of the Earth's
magnetosphere through which they have passed. For this reason, many
countries maintain observatories in polar regions, especially Antarctica, 
which monitor and collect
 whistler data: {\em e.g.}, the Halley research station, maintained
by the British Antarctic Survey, which is  located on the edge of
the  Antarctic mainland. 

\begin{figure}
\epsfysize=2.5in
\centerline{\epsffile{Chapter04/whistler.eps}}
\caption{\em Typical path of a whistler wave through the Earth's magnetosphere.}\label{f14}
\end{figure}

For a left-handed circularly polarized wave, similar considerations 
to the above give a dispersion
curve of the form sketched in Fig.~\ref{f15}. In this case, $n^2$ goes to
infinity at the ion cyclotron frequency, ${\Omega}_i$, corresponding to
the so-called {\em ion cyclotron resonance}\/ (at $L\rightarrow\infty$). At this resonance, the
rotating  electric
field associated with a left-handed wave resonates with the gyromotion
of the ions, allowing wave energy to be converted into perpendicular kinetic
energy of the ions. There is a band of frequencies, lying above the ion cyclotron
frequency, in which the left-handed wave does not propagate. At very high
frequencies a propagating mode exists, which is basically a standard
left-handed circularly polarized electromagnetic wave, somewhat modified
by the presence of the plasma. The cutoff frequency for this wave is
\begin{equation}\label{e4.76}
\omega_2 \simeq	 -|{\Omega}_e|/2 + 
\sqrt{{\Omega}_e^{~2}/4 + {\Pi}_e^{~2}}.
\end{equation}
As before, the lower branch in Fig.~\ref{f15} describes a wave that can only propagate in the
presence of an equilibrium magnetic field, whereas the upper branch
describes a wave that 
can propagate in the absence  an equilibrium  field. The continuation of the
Alfv\'{e}n wave to just below the ion cyclotron frequency is 
generally called the
{\em ion cyclotron wave}.

\begin{figure}
\epsfysize=4in
\centerline{\epsffile{Chapter04/parl.eps}}
\caption{\em Dispersion relation for a left-handed wave propagating parallel
to the magnetic field in a magnetized plasma.}\label{f15}
\end{figure}

\section{Perpendicular Wave Propagation}\label{s4.10}
Let us now consider wave propagation, at arbitrary frequencies, {\em perpendicular}\/ 
to the equilibrium magnetic field. When $\theta=\pi/2$, the eigenmode equation
(\ref{e4.30}) simplifies to 
\begin{equation}\label{e4.77}
\left(\!\begin{array}{ccc}
S & -{\rm i}\,D & 0\\
{\rm i} \,D           & S - n^2     & 0 \\
0& 0 & P - n^2
\end{array}\!\right)\left(\!
\begin{array}{c} E_x\\ E_y \\ E_z \end{array}\!\right) = {\bf 0}.
\end{equation}
One obvious way of solving this equation is to have $P-n^2 =0$, or
\begin{equation}\label{e4.78}
\omega^2 = {\Pi}_e^{~2} +k^2c^2,
\end{equation}
with the eigenvector $(0,0,E_z)$. Since the wave-vector now points in the
$x$-direction, this is clearly a transverse wave polarized with its electric
field parallel to the equilibrium magnetic field. Particle motions are
along the magnetic field, so the mode dynamics are completely unaffected  
by this field. Thus, the wave is identical to the 
{\em electromagnetic plasma wave}\/ found
previously in an unmagnetized plasma. This wave is known as the {\em ordinary},
or $O$-, mode. 

The other solution to Eq.~(\ref{e4.77}) is obtained by setting the $2\times 2$ determinant
involving the $x$- and $y$- components of the electric field to zero. The
dispersion relation reduces to 
\begin{equation}\label{e4.79}
n^2 = \frac{R\,L}{S},
\end{equation}
with the associated eigenvector $E_x\,(1,-{\rm i}\,S/D, 0)$. 

Let us, first of all, search for the cutoff frequencies, at which $n^2$ goes to
zero. 
According to Eq.~(\ref{e4.79}), these frequencies are the roots of $R=0$ and $L=0$.
In fact, we have already solved these equations (recall that
cutoff frequencies do not depend on $\theta$). 
There are two cutoff frequencies, $\omega_1$ and $\omega_2$,
which are specified by Eqs.~(\ref{e4.73}) and (\ref{e4.76}), respectively. 

Let us, next, search for the resonant frequencies, at which $n^2$ goes to
infinity. According to Eq.~(\ref{e4.79}),  the
resonant frequencies are solutions of
\begin{equation}\label{e4.80}
S = 1- \frac{{\Pi}_e^{~2}}{\omega^2-{\Omega}_e^{~2}}
- \frac{{\Pi}_i^{~2}}{\omega^2-{\Omega}_i^{~2}}=0.
\end{equation}
The roots of this equations can be obtained as follows. First, we note that
if the first two terms are equated to zero, we obtain $\omega=\omega_{\rm UH}$,
where
\begin{equation}
\omega_{\rm UH} = \sqrt{{\Pi}_e^{~2} + {\Omega}_e^{~2}}.
\end{equation}
If this frequency  is substituted into the third term, the result is
far less than unity. We conclude that $\omega_{\rm UH}$ is a good approximation
to one of the roots of Eq.~(\ref{e4.80}). To
obtain the second root, we make use of the fact that the product of the square
of the roots
is 
\begin{equation}
{\Omega}_e^{~2}\,{\Omega}_i^{~2} + {\Pi}_e^{~2}\,{\Omega}_i^{~2}
+ {\Pi}_i^{~2}\,{\Omega}_e^{~2}\simeq 
{\Omega}_e^{~2}\,{\Omega}_i^{~2}+ {\Pi}_i^{~2}\,{\Omega}_e^{~2}.
\end{equation}
We, thus, obtain $\omega= \omega_{\rm LH}$, where
\begin{equation}
\omega_{\rm LH} = \sqrt{\frac{ {\Omega}_e^{~2}\,{\Omega}_i^{~2}+ {\Pi}_i^{~2}\,{\Omega}_e^{~2} } { {\Pi}_e^{~2} + {\Omega}_e^{~2} }}.
\end{equation}

The first resonant frequency, $\omega_{\rm UH}$, is greater than the
electron cyclotron or plasma frequencies, and is called the {\em upper hybrid
frequency}. The second resonant frequency, $\omega_{\rm LH}$, lies between the
electron and ion cyclotron frequencies, and is called the 
{\em lower hybrid frequency}. 

Unfortunately, there is no simple explanation of the origins of the
two hybrid resonances in terms of the motions of individual particles. 

At low frequencies, the mode in question 
 reverts to the compressional-Alfv\'{e}n wave
discussed previously. Note that the shear-Alfv\'{e}n wave does not
propagate perpendicular to the magnetic field. 


Using the above information, and the easily demonstrated fact that
\begin{equation}
\omega_{\rm LH} < \omega_2 < \omega_{\rm UH} < \omega_1,
\end{equation}
we can deduce that the dispersion curve for the mode in question 
takes the form sketched in Fig.~\ref{f16}.
The lowest frequency branch corresponds to the compressional-Alfv\'{e}n wave.
The other two branches constitute the {\em extraordinary}, or $X$-, wave.
The upper branch is basically a linearly polarized (in the $y$-direction)
 electromagnetic wave, somewhat modified by the presence of the plasma. This 
branch corresponds to a wave which 
propagates in the absence of an equilibrium magnetic field. The lowest 
branch corresponds to a wave which does not propagate in the absence of an
equilibrium field. Finally, the middle branch corresponds to a wave which
converts into an electrostatic plasma wave in the absence of an equilibrium
magnetic field.

\begin{figure}
\epsfysize=4in
\centerline{\epsffile{Chapter04/perp.eps}}
\caption{\em Dispersion relation for a wave propagating perpendicular to the
magnetic field in a magnetized plasma.}\label{f16}
\end{figure}

Wave propagation at oblique angles is generally more complicated than
propagation parallel or perpendicular to the equilibrium magnetic field,
but does not involve any new physical effects. 

\section{Wave Propagation Through Inhomogeneous Plasmas}\label{s4.11}
Up to now, we have only analyzed  wave propagation through
homogeneous plasmas. Let us now broaden our approach to take into
account the far more realistic case of wave propagation through
{\em inhomogeneous}\/ plasmas. 

Let us start off by examining  a very simple case. Consider a plane 
electromagnetic wave,
of frequency $\omega$, propagating along the $z$-axis in an unmagnetized plasma
whose refractive index, $n$, is a function of $z$. We assume that
the wave normal is initially aligned along the $z$-axis, and, furthermore,  that
the wave starts off polarized in the $y$-direction. It is
easily demonstrated that the wave normal subsequently remains aligned along
the $z$-axis, and also that the polarization
state of the wave does not change.
Thus, the wave is fully described by 
\begin{equation}
E_y(z,t) \equiv E_y(z)\,\exp(-{\rm i}\,\omega\,t),
\end{equation}
and
\begin{equation}
B_x(z,t) \equiv B_x(z)\,\exp(-{\rm i}\,\omega\,t).
\end{equation}
It can easily be shown that $E_y(z)$ and $B_x(z)$ satisfy
the differential equations
\begin{equation}\label{e4.87}
\frac{d^2 E_y}{dz^2} + k_0^{~2}\,n^2\,E_y = 0,
\end{equation}
and
\begin{equation}\label{e4.88}
\frac{d\,cB_x}{dz} = -{\rm i}\,k_0\,n^2\,E_y,
\end{equation}
respectively. Here, $k_0=\omega/c$ is the wave-number 
in free space. Of course, the actual wave-number is $k=k_0\,n$. 

The solution to Eq.~(\ref{e4.87}) for the case of a homogeneous plasma, for which
$n$ is  constant, is straightforward:
\begin{equation}\label{e4.89}
E_y = A\,{\rm e}^{\,{\rm i}\,\phi(z)},
\end{equation}
where $A$ is a constant, and
\begin{equation}\label{e4.90}
\phi = \pm k_0\,n\,z.
\end{equation}
 The solution (\ref{e4.89})
represents a  wave of constant amplitude, $A$, and phase, $\phi(z)$. According to 
Eq.~(\ref{e4.90}),
there are, in fact, two independent waves which can propagate through the plasma.
The upper sign corresponds to a wave which propagates in the $+z$-direction,
whereas  the lower sign corresponds to a wave which propagates in the
$-z$-direction. Both waves propagate with the constant phase velocity $c/n$. 

In general, if $n=n(z)$ then the solution of Eq.~(\ref{e4.87}) does not remotely resemble
the wave-like solution (\ref{e4.89}). However, in the limit in which $n(z)$ is
a ``slowly varying'' function of $z$ (exactly how slowly varying is something which
will be established later on), we   expect to recover wave-like solutions.
Let us suppose that $n(z)$ is indeed a ``slowly varying'' function, and let us try
substituting the wave solution (\ref{e4.89}) into Eq.~(\ref{e4.87}). We obtain
\begin{equation}\label{e4.91}
\left(\frac{d\phi}{dz}\right)^2 = k_0^{~2} n^2 +{\rm i}\,\frac{d^2\phi}
{dz^2}.
\end{equation}
This is a non-linear differential equation which, in general, is very difficult
to solve. However, we note that if $n$ is a constant then $d^2\phi/dz^2=0$.
It is, therefore, reasonable to suppose that if $n(z)$ is a ``slowly varying'' function
then the last term on the right-hand side of the above equation can be
regarded as being small. Thus, to a first approximation Eq.~(4.91) yields
\begin{equation}
\frac{d\phi}{dz} \simeq \pm k_0\,n,
\end{equation}
and
\begin{equation}\label{e4.93}
\frac{d^2 \phi}{dz^2} \simeq \pm k_0\,\frac{dn}{dz}.
\end{equation}
It is clear from a comparison of Eqs.~(\ref{e4.91}) and (\ref{e4.93}) that $n(z)$ can
be regarded as a ``slowly varying'' function of $z$ as long as its variation
length-scale  is far longer than the wave-length of the wave. 
In other words, provided that $(dn/dz)/(k_0\,n^2)\ll 1$. 

The second approximation to the solution  is obtained  by substituting Eq.~(\ref{e4.93}) into
the right-hand side of Eq.~(\ref{e4.91}):
\begin{equation}
\frac{d\phi}{dz} \simeq \pm \left(k_0^{~2} n^2 \pm {\rm i} \,k_0 \,\frac{dn}{dz}\right)^{1/2}.
\end{equation}
This gives
\begin{equation}
\frac{d\phi}{dz} \simeq \pm k_0\,n\left(1\pm \frac{{\rm i}}{k_0 \,n^2}\frac{dn}{dz}
\right)^{1/2}\simeq \pm k_0\,n + \frac{\rm i}{2n}\frac{dn}{dz},
\end{equation}
where use has been made of the binomial expansion. The above expression can be integrated
to give
\begin{equation}\label{e4.96}
\phi \sim \pm k_0\! \int^z \!n\,dz +{\rm i}\,\log(n^{1/2}).
\end{equation}
Substitution of Eq.~(\ref{e4.96}) into Eq.~(\ref{e4.89}) yields the final result
\begin{equation}\label{e4.97}
E_y \simeq A\,n^{-1/2}\,\exp\left(\pm {\rm i}\, k_0\! \int^z \!n\,dz\right).
\end{equation}
It follows from Eq.~(\ref{e4.88}) that
\begin{equation}
cB_x\simeq \mp A\, n^{1/2}\,\exp\left(\pm {\rm i}\,k_0 \!\int^z\! n\,dz\right) -
\frac{ {\rm i}\, A}{2k_0\,n^{3/2}} \frac{dn}
{dz}\,\exp\left(\pm {\rm i}\,k_0 \!\int^z \!n\,dz\right).
\end{equation}
Note that the second term is small compared to the first,  and can usually be neglected.

Let us test to what extent  the expression (\ref{e4.97}) is a good solution 
of Eq. (\ref{e4.87}) by substituting this expression into the left-hand side
of the equation. The result is
\begin{equation}
\frac{A}{n^{1/2}}\left\{ \frac{3}{4}\left(\frac{1}{n}
\frac{dn}{dz}\right)^2 -\frac{1}{2n}\frac{d^2 n}{dz^2}\right\}
{\rm exp}\left( \pm {\rm i}\,k_0\!\int^z\!n\,dz\right).
\end{equation}
This must be small compared with either term on the left-hand side
of Eq.~(\ref{e4.87}). Hence, the
condition for Eq.~(\ref{e4.97}) to be a good solution of Eq.~(\ref{e4.87})
becomes
\begin{equation}\label{e4.100}
\frac{1}{k_0^{~2}}\left|  \frac{3}{4}\left(\frac{1}{n^2}
\frac{dn}{dz}\right)^2 -\frac{1}{2n^3}\frac{d^2 n}{dz^2}\right| \ll 1.
\end{equation}

The solutions
\begin{eqnarray}\label{e4.101a}
E_y &\simeq& A\,n^{-1/2}\,\exp\left(\pm {\rm i}\, k_0 \!\int^z \!n\,dz\right),\\[0.5ex]\label{e4.101b}
cB_x&\simeq& \mp A\, n^{1/2}\,\exp\left(\pm {\rm i}\,k_0\! \int^z\! n\,dz\right),
\end{eqnarray}
to the non-uniform wave equations (\ref{e4.87}) and (\ref{e4.88}) are most commonly
referred to as  the {\em WKB solutions},
in honour of G.~Wentzel, H.A.~Kramers, and L.~Brillouin, who are credited with
independently discovering these solutions
(in a quantum mechanical context) in 1926. Actually, H.~Jeffries wrote
a paper on the WKB solutions 
(in a wave propagation context) in 1923. Hence, some people call them the WKBJ
solutions (or even the JWKB solutions). To be strictly
accurate, the WKB solutions were 
first discussed
 by Liouville and Green in 1837, and again by Rayleigh in 1912. 
In the following, we  refer to Eqs.~(\ref{e4.101a})--(\ref{e4.101b})
as the WKB solutions, since this is what they are most commonly known as. However,
it should be understand that, in doing so, 
 we are not  making any  definitive statement as to the credit due
to various scientists in discovering them. 

Recall, that when a propagating wave is normally  incident on an {\em interface}, 
where the
refractive index  suddenly changes (for instance, when a light
wave propagating through
 air is normally incident on a glass slab), there is generally
significant  reflection of the wave. However, according to the WKB solutions,
(\ref{e4.101a})--(\ref{e4.101b}), when a propagating wave is normally incident on a medium in which
the refractive index changes {\em slowly}\/ along the direction of propagation of the
wave then the wave is not reflected at all. This is true 
even if the refractive index
varies {\em very substantially}\/ along the path of propagation of the wave,
as long as it varies {\em slowly}. The WKB
solutions imply that as the wave propagates through the medium its wave-length
gradually changes. In fact, the wave-length at position $z$ is approximately
$\lambda(z)= 2\pi/ k_0\,n(z)$. Equations~(\ref{e4.101a})--(\ref{e4.101b}) also imply that the amplitude
of the wave gradually changes as it propagates. In fact, the amplitude of the electric
field component is inversely proportional to $n^{1/2}$, whereas the amplitude of the
magnetic field component is directly proportional to $n^{1/2}$. 
Note, however, that the energy
flux in the $z$-direction, given by the the Poynting vector $-(E_y B_x^{~\ast}
+E_y^{~\ast} B_x)/(4\mu_0)$, remains constant (assuming that $n$ is predominately
real). 

Of course, the WKB solutions (\ref{e4.101a})--(\ref{e4.101b}) are only {\em approximations}. In reality,
a wave propagating into a medium in which the refractive index is a slowly
varying function of position is subject to a small amount of reflection. 
However, it is easily demonstrated that the ratio of the reflected amplitude
to the incident amplitude is of order $(dn/dz)/(k_0\,n^2)$. Thus, as long as
the refractive index varies on a much longer length-scale than the wave-length
of the radiation, the reflected wave is negligibly small. This conclusion remains
valid as long as the inequality (\ref{e4.100}) is satisfied.
 This inequality obviously 
 breaks down in the vicinity of a point where $n^2=0$. We would, therefore,
expect strong reflection of the incident wave from such a point.
Furthermore,  the WKB solutions also break down at a
point where $n^2\rightarrow\infty$, since the amplitude of $B_x$ becomes
infinite.

\section{Cutoffs}\label{s4.12}
We have seen that  electromagnetic wave propagation (in one dimension) through an
inhomogeneous plasma, in the physically relevant limit in which the variation
length-scale of the plasma is much greater than the wave-length of the wave,
is well described by the WKB solutions, (\ref{e4.101a})--(\ref{e4.101b}). However, these
solutions break down in the immediate vicinity of a {\em cutoff}, where
$n^2=0$, or a {\em resonance}, where $n^2\rightarrow\infty$. Let us
now examine what happens to electromagnetic waves propagating through
a plasma when they encounter a cutoff or a resonance.

Suppose that a cutoff is located at $z=0$, so that
\begin{equation}\label{e4.102}
n^2 = a\,z+ O(z^2)
\end{equation}
in the immediate vicinity of this point, where $a>0$. It is evident, from the
WKB solutions, (\ref{e4.101a})--(\ref{e4.101b}),  that
the cutoff point lies at the boundary between a region ($z>0$) in which
electromagnetic 
waves propagate, and a region ($z<0$) in which the  waves are evanescent. 
In  a physically realistic solution, we would expect the wave amplitude to
decay (as $z$ decreases) in the evanescent region $z<0$. Let us search for
such a wave solution. 

In the immediate vicinity of the cutoff point, 
$z=0$, Eqs.~(\ref{e4.87}) and (\ref{e4.102}) yield
\begin{equation}\label{e4.103}
\frac{d^2 E_y}{d\hat{z}^2} + \hat{z}\,E_y = 0,
\end{equation}
where 
\begin{equation}
\hat{z} = (k_0^{~2}\,a)^{1/3}\,z.
\end{equation}
Equation (\ref{e4.103}) is a standard
equation, known as {\em Airy's equation}, and possesses two 
independent solutions, denoted ${\rm Ai}(-\hat{z})$ and ${\rm Bi}(-\hat{z})$.\footnote{M.~Abramowitz, and I.A.~Stegun, {\em Handbook of Mathematical Functions}
(Dover, New York NY, 1964), p.~446.} The second solution, ${\rm Bi}(-\hat{z})$,
is unphysical, since it blows up as $\hat{z}\rightarrow-\infty$. 
The physical solution, ${\rm Ai}(-\hat{z})$, has the asymptotic
behaviour
\begin{equation}
{\rm Ai}(-\hat{z})\sim \frac{1}{2\,\sqrt{\pi}}\, |\hat{z}|^{-1/4}\,\exp\!
\left(-\frac{2}{3}\,|\hat{z}|^{3/2}\right)
\end{equation}
in the limit $\hat{z}\rightarrow-\infty$, and
\begin{equation}
{\rm Ai}(-\hat{z})\sim \frac{1}{\sqrt{\pi}}\, \hat{z}^{-1/4}\,\sin\!
\left(\frac{2}{3}\,\hat{z}^{3/2}+\frac{\pi}{4}\right)
\end{equation}
in the limit $\hat{z}\rightarrow +\infty$. 

Suppose that a unit amplitude plane electromagnetic wave, polarized in the
$y$-direction,   is launched
from an antenna, located at large positive $z$,  towards the cutoff point at $z=0$.
It is assumed that $n=1$ at the launch point. 
In the non-evanescent region, $z>0$,  the wave can be
represented as a linear combination
of propagating WKB solutions:
\begin{equation}
E_y(z) = n^{-1/2}\,\exp\left(- {\rm i}\, k_0 \!\int_0^z \!n\,dz\right)
+ R\,n^{-1/2}\,\exp\left(+{\rm i}\, k_0 \!\int_0^z \!n\,dz\right).
\end{equation}
The first term on the right-hand side of the above equation represents the
incident wave, whereas the second term represents the reflected wave. 
The complex constant $R$ is the {\em coefficient of reflection}. 
In the vicinity of the cutoff point ({\em i.e.}, $z$ small and positive,
or $\hat{z}$ large and positive)
the above expression reduces to
\begin{equation}\label{e4.108}
E_y(\hat{z}) = (k_0/a)^{1/6}\,\left[
\hat{z}^{-1/4}\exp\!\left(-{\rm i}\,\frac{2}{3}\,\hat{z}^{3/2}\right)
+ R\,\hat{z}^{-1/4}\,\exp\!\left(+{\rm i}\,\frac{2}{3}\,\hat{z}^{3/2}\right)
\right].
\end{equation}
However, we  have another expression for the wave in this region. Namely,
\begin{equation}
E_y(\hat{z}) = C\,{\rm Ai}(-\hat{z}) \simeq \frac{C}{\sqrt{\pi}}\, \hat{z}^{-1/4}\,\sin\!
\left(\frac{2}{3}\,\hat{z}^{3/2}+\frac{\pi}{4}\right),
\end{equation}
where $C$ is an arbitrary constant. 
The above equation can be written
\begin{equation}\label{e4.110}
E_y(\hat{z}) =\frac{C}{2}\sqrt{\frac{{\rm i}}{\pi}}
\left[\hat{z}^{-1/4}\,\exp\!\left(-{\rm i}\,\frac{2}{3}\,\hat{z}^{3/2}\right)
-{\rm i}\,\hat{z}^{-1/4}\,\exp\!\left(+{\rm i}\,\frac{2}{3}\,\hat{z}^{3/2}\right)
\right].
\end{equation}
A comparison of Eqs.~(\ref{e4.108}) and (\ref{e4.110}) yields
\begin{equation}
R = -{\rm i}.
\end{equation}
In other words, at a cutoff point there is {\em total reflection}, since
$|R|=1$, with a $-\pi/2$ phase-shift. 

\section{Resonances}
Suppose, now, that a resonance is located at $z=0$, so that
\begin{equation}\label{e4.112}
n^2 = \frac{b}{z + {\rm i}\,\epsilon} +O(1)
\end{equation}
in the immediate vicinity of this point, where $b>0$. Here, $\epsilon$
is a small real constant. We introduce $\epsilon$ at this point principally
as a mathematical artifice to ensure that $E_y$ remains single-valued and
finite. However, as will become clear later on, $\epsilon$ has a physical significance
in terms of damping or spontaneous excitation. 

In the immediate vicinity of the resonance point, $z=0$, Eqs.~(\ref{e4.87}) and (\ref{e4.112})
yield
\begin{equation}\label{e4.113}
\frac{d^2 E_y}{d\hat{z}^2} + \frac{E_y}{\hat{z}+{\rm i}\,\hat{\epsilon}} = 0,
\end{equation}
where
\begin{equation}
\hat{z} = (k_0^{~2}\,b)\,z,
\end{equation}
and $\hat{\epsilon} = (k_0^{~2}\,b)\,\epsilon$. This equation is
singular at the point $\hat{z} = -{\rm i}\,\hat{\epsilon}$. 
Thus, it is necessary to
introduce a branch-cut into the complex-$\hat{z}$ plane in order to
ensure that $E_y(\hat{z})$ is single-valued. If $\epsilon>0$ then the
branch-cut lies in the lower half-plane, whereas if $\epsilon<0$ then
the branch-cut lies in the upper half-plane---see Fig.~\ref{f17}. Suppose that the
argument of $\hat{z}$ is $0$ on the positive real $\hat{z}$-axis.
It follows that the argument of $\hat{z}$ on the negative real $\hat{z}$-axis
is $+\pi$ when $\epsilon>0$ and $-\pi$ when $\epsilon<0$. 

\begin{figure}
\epsfysize=3.5in
\centerline{\epsffile{Chapter04/sing.eps}}
\caption{\em Branch-cuts in the $z$-plane close to a wave resonance.}\label{f17}
\end{figure}

Let 
\begin{eqnarray}
y &=& 2\,\sqrt{\hat{z}},\\[0.5ex]
E_y &=& y\,\psi(y).
\end{eqnarray}
In the limit $\epsilon\rightarrow 0$, Eq.~(\ref{e4.113}) transforms into
\begin{equation}
\frac{d^2\psi}{dy^2} + \frac{1}{y}\frac{d\psi}{dy} + \left(1-\frac{1}{y^2}\right)\!
\psi = 0.
\end{equation}
This is a standard equation, known as {\em Bessel's equation}\/ of order one,\footnote{M.~Abramowitz, and I.A.~Stegun, {\em Handbook of Mathematical Functions}
(Dover, New York NY, 1964), p.~358.}
and possesses two independent solutions, denoted $J_1(y)$ and $Y_1(y)$,
respectively. Thus, on the positive real $\hat{z}$-axis we can write
the most general solution to Eq.~(\ref{e4.113}) in the form
\begin{equation}\label{e4.117}
E_y(\hat{z}) = A\,\sqrt{\hat{z}}\,J_1(2\,\sqrt{\hat{z}}) + 
B\,\sqrt{\hat{z}}\,Y_1(2\,\sqrt{\hat{z}}),
\end{equation}
where $A$ and $B$ are two arbitrary constants. 

Let 
\begin{eqnarray}
y &=& 2\,\sqrt{a\hat{z}},\\[0.5ex]
E_y &=& y\,\psi(y),
\end{eqnarray}
where
\begin{equation}
a = {\rm e}^{-{\rm i}\,\pi\,{\rm sgn}(\epsilon)}.
\end{equation}
Note that the argument of $a\hat{z}$ is zero on the negative real
$\hat{z}$-axis. 
In the limit $\epsilon\rightarrow 0$, Eq.~(\ref{e4.113}) transforms into
\begin{equation}
\frac{d^2\psi}{dy^2} + \frac{1}{y}\frac{d\psi}{dy} - \left(1+\frac{1}{y^2}\right)\!
\psi = 0.
\end{equation}
This is a standard equation, known as {\em Bessel's modified
equation}\/ of order one,\footnote{M.~Abramowitz, and I.A.~Stegun, {\em Handbook of Mathematical Functions}
(Dover, New York NY, 1964), p.~374.}
and possesses two independent solutions, denoted $I_1(y)$ and $K_1(y)$,
respectively. Thus, on the negative real $\hat{z}$-axis we can write
the most general solution to Eq.~(\ref{e4.113}) in the form
\begin{equation}\label{e4.121}
E_y(\hat{z}) = C\,\sqrt{a\hat{z}}\,I_1(2\,\sqrt{a\hat{z}}) + 
D\,\sqrt{a\hat{z}}\,K_1(2\,\sqrt{a\hat{z}}),
\end{equation}
where $C$ and $D$ are two arbitrary constants. 

Now, the Bessel functions $J_1$, $Y_1$, $I_1$, and $K_1$ are all  perfectly 
well-defined for
complex arguments, so the two expressions (\ref{e4.117}) and (\ref{e4.121}) must, in fact,  be 
{\em identical}.
In particular, the constants $C$ and $D$ must somehow be related to the
constants $A$ and $B$. In order to establish this relationship, it
is convenient to investigate the behaviour of the expressions
(\ref{e4.117}) and (\ref{e4.121}) in the limit of small $\hat{z}$: {\em i.e.}, $|\hat{z}|\ll 1$.
In this limit, 
\begin{eqnarray}\label{e4.122a}
\sqrt{\hat{z}}\,J_1(2\,\sqrt{\hat{z}})&=& \hat{z} + O(\hat{z}^2),\\[0.5ex]
\sqrt{a\hat{z}} \,I_1(2\,\sqrt{a\hat{z}}) &=& - \hat{z} + O(\hat{z}^2),\\[0.5ex]
\sqrt{\hat{z}}\,Y_1(2\,\sqrt{\hat{z}}) &=& -\frac{1}{\pi}\left[
1 -\left\{ \ln|\hat{z}| + 2\,\gamma-1\right\}\hat{z}\,\right]\nonumber\\[0.5ex]
&&+ O(\hat{z}^2),\\[0.5ex]
\sqrt{a\hat{z}}\,K_1(2\,\sqrt{a\hat{z}}) &=& \frac{1}{2}\left[
1 -\left\{ \ln|\hat{z}| + 2\,\gamma-1\right\}\hat{z} - {\rm i}\,{\rm arg}(a)\,
\hat{z}\,\right]
\nonumber\\[0.5ex]&&
+ O(\hat{z}^2),\label{e4.122d}
\end{eqnarray}
where $\gamma$ is Euler's constant, and $\hat{z}$ is assumed to lie on the positive
real $\hat{z}$-axis. It follows, by a comparison of Eqs.~(\ref{e4.117}), (\ref{e4.121}), 
and (\ref{e4.122a})--(\ref{e4.122d}), that the choice
\begin{eqnarray}\label{e4.123a}
C&=& -A +{\rm i}\,\frac{\pi}{2}\,{\rm sgn}(\epsilon)\,D = -A - {\rm i}\,
{\rm sgn}(\epsilon)\,B,\\[0.5ex]
D &=& - \frac{2}{\pi}\,B,
\end{eqnarray}
ensures that the expressions (\ref{e4.117}) and (\ref{e4.121}) are indeed identical.

Now, in the limit $|\hat{z}|\gg 1$,
\begin{eqnarray}
\sqrt{a\hat{z}}\,I_1(2\,\sqrt{a\hat{z}}) &\sim & 
\frac{|\hat{z}|^{1/4}}{2\sqrt{\pi}}\,
{\rm e}^{+2\sqrt{|\hat{z}|}},\\[0.5ex]
\sqrt{a\hat{z}}\,K_1(2\,\sqrt{a\hat{z}}) &\sim & 
\frac{\sqrt{\pi}\,|\hat{z}|^{1/4}}{2}\,
{\rm e}^{-2\sqrt{|\hat{z}|}},
\end{eqnarray}
where $\hat{z}$ is assumed to lie on the negative real $\hat{z}$-axis. 
It is clear that the $I_1$ solution is unphysical, since it blows up
in the evanescent region $(\hat{z}<0)$. Thus, the coefficient $C$ in expression
(\ref{e4.121}) must be set to zero in order to prevent $E_y(\hat{z})$ from
blowing up as $\hat{z}\rightarrow -\infty$. According to Eq.~(\ref{e4.123a}),
this constraint implies that
\begin{equation}\label{e4.125}
A = -{\rm i}\,{\rm sgn}(\epsilon)\,B.
\end{equation}
 
In the limit $|\hat{z}|\gg 1$, 
\begin{eqnarray}\label{e4.126a}
\sqrt{\hat{z}}\,J_1(2\,\sqrt{\hat{z}})&\sim& \frac{\hat{z}^{1/4}}{\sqrt{\pi}}\cos\!\left(2\,
\sqrt{z}
-\frac{3}{4}\,\pi\right),\\[0.5ex]
\sqrt{\hat{z}}\,Y_1(2\,\sqrt{\hat{z}})&\sim & \frac{\hat{z}^{1/4}}{\sqrt{\pi}}\sin\!\left(2\,
\sqrt{z}
-\frac{3}{4}\,\pi\right),\label{e4.126b}
\end{eqnarray}
where $\hat{z}$ is assumed to lie on the positive real $\hat{z}$-axis. 
It follows from Eqs.~(\ref{e4.117}), (\ref{e4.125}), and (\ref{e4.126a})--(\ref{e4.126b}) that in the
non-evanescent region ($\hat{z}> 0$) the most general
{\em physical} solution takes the form
\begin{eqnarray}
E_y(\hat{z})& =& A'\,\left[{\rm sgn}(\epsilon) + 1\right]\,\hat{z}^{1/4}\,
\exp\!\left[
+{\rm i}\,2\sqrt{\hat{z}} - \frac{3}{4}\,\pi\right]\nonumber\\[0.5ex]
&&+A'\,
\left[{\rm sgn}(\epsilon) - 1\right]\,\hat{z}^{1/4}\,\exp\!\left[
-{\rm i}\,2\sqrt{\hat{z}} + \frac{3}{4}\,\pi\right],\label{e4.127}
\end{eqnarray}
where $A'$ is an arbitrary constant.

Suppose that a plane electromagnetic wave, polarized in the
$y$-direction,   is launched
from an antenna, located at large positive $z$,  towards the resonance
 point at $z=0$.
It is assumed that $n=1$ at the launch point. 
In the non-evanescent region, $z>0$,  the wave can be
represented as a linear combination
of propagating WKB solutions:
\begin{equation}
E_y(z) = E\, n^{-1/2}\,\exp\left(- {\rm i}\, k_0 \!\int_0^z \!n\,dz\right)
+ F\,n^{-1/2}\,\exp\left(+{\rm i}\, k_0 \!\int_0^z \!n\,dz\right).
\end{equation}
The first term on the right-hand side of the above equation represents the
incident wave, whereas the second term represents the reflected wave. 
Here, $E$ is the amplitude of the incident wave, and $F$ is the amplitude of
the reflected wave.
In the vicinity of the resonance point ({\em i.e.}, $z$ small and positive,
or $\hat{z}$ large and positive)
the above expression reduces to
\begin{equation}\label{e4.129}
E_y(\hat{z}) \simeq  (k_0 b)^{-1/2}\left[
E\,\hat{z}^{1/4}\exp\left(-{\rm i}\,2\,\sqrt{\hat{z}}\right)
+ F\,\hat{z}^{1/4}\exp\left(+{\rm i}\,2\,\sqrt{\hat{z}}\right)\right].
\end{equation}
A comparison of Eqs.~(\ref{e4.127}) and (\ref{e4.129}) shows that if $\epsilon>0$ then
$E=0$. In other words, there is a reflected wave, but no incident wave. 
This corresponds to the  {\em spontaneous  excitation}\/ of  waves in the vicinity of the
resonance. On the other hand, if $\epsilon<0$ then $F=0$. In other words,
there is an incident wave, but no reflected wave. This corresponds to the
{\em total absorption}\/ of incident waves in the vicinity of the resonance. 
It is clear that if $\epsilon>0$ then $\epsilon$ represents some sort of
spontaneous wave excitation mechanism, whereas if $\epsilon<0$ then
$\epsilon$ represents a wave absorption, or damping, mechanism. 
We would normally expect plasmas to absorb incident wave energy, rather
than spontaneously emit waves, so we conclude that, under most circumstances,
$\epsilon<0$, and resonances {\em absorb}\/ incident waves {\em without reflection}.

\section{Resonant Layers}
Consider the situation under investigation in the 
preceding section, in which a plane wave, polarized in the $y$-direction,
is launched along the $z$-axis, from an antenna located at large positive $z$, 
and absorbed at a resonance located at $z=0$. In the vicinity
of the resonant point, the electric component of the wave satisfies
\begin{equation}\label{e4.130}
\frac{d^2 E_y}{dz^2} + \frac{k_0^2\,b}{z+{\rm i}\,\epsilon} E_y = 0,
\end{equation}
where $b>0$ and $\epsilon<0$. 

The time-averaged Poynting flux in the $z$-direction is written
\begin{equation}
P_z = - \frac{(E_y\,B_x^{~\ast} + E_y^{~\ast}\,B_x)}{4 \mu_0}.
\end{equation}
Now, the Faraday-Maxwell equation yields
\begin{equation}
{\rm i}\,\omega\,B_x = -\frac{d E_y}{dz}.
\end{equation}
Thus, we have
\begin{equation}
P_z = -\frac{{\rm i}}{4\,\mu_0\,\omega} \left(\frac{d E_y}{dz}\, E_y^{~\ast}
- \frac{d E_y^{~\ast}}{dz} \,E_y\right).
\end{equation}

Let us ascribe any variation of $P_z$ with $z$ to the wave energy emitted by the
plasma.  We then have
\begin{equation}
\frac{d P_z}{dz} = W,
\end{equation}
where $W$ is the power emitted by the plasma per unit volume.
It follows that
\begin{equation}\label{e4.135}
W = -\frac{{\rm i}}{4\,\mu_0\,\omega}\left(\frac{d^2 E_y}{dz^2}\,E_y^{~\ast}
- \frac{d^2 E_y^{~\ast}}{dz^2}\,E_y\right).
\end{equation}
Equations~(\ref{e4.130}) and (\ref{e4.135}) yield
\begin{equation}
W = \frac{k_0^{~2}\,b}{2\,\mu_0\,\omega} \frac{\epsilon}{z^2 + \epsilon^2}\, |E_y|^2.
\end{equation}
Note that $W<0$, since $\epsilon<0$, so wave energy is {\em absorbed}\/ by the
plasma. It is clear from the above formula that the absorption takes
place in a narrow layer, of thickness $|\epsilon|$, centred on the
resonance point, $z=0$.

\section{Collisional Damping}
Let us now consider a real-life damping mechanism. Equation~(\ref{e4.12b})
specifies the linearized Ohm's law in the {\em collisionless}\/ 
cold-plasma approximation.
However, in the presence of {\em collisions}\/ this expression acquires
an extra term (see Sect.~\ref{s3}), such  that
\begin{equation}\label{e4.137}
{\bf E} = - {\bf V}\times{\bf B}_0 + \frac{{\bf j}\times{\bf B}_0}{ne}
-{\rm i}\,\omega\,\frac{m_e}{ne^2}\,{\bf j} + \nu\,\frac{m_e}{n e^2}\,{\bf j},
\end{equation}
where $\nu \equiv\tau_e^{-1}$ is the {\em collision frequency}. Here,
we have neglected the small difference between the parallel and perpendicular
plasma electrical conductivities, for the sake of simplicity. When
Eq.~(\ref{e4.137}) is used to calculate the dielectric permittivity for a
right-handed wave, in the limit $\omega\gg {\Omega}_i$, we obtain
\begin{equation}
R \simeq 1- \frac{{\Pi}_e^{~2}}{\omega\,(\omega+{\rm i}\,\nu
-|{\Omega}_e|)}.
\end{equation}

A right-handed circularly polarized wave, propagating parallel to the
magnetic field, is governed by the dispersion relation 
\begin{equation}
n^2 = R \simeq 1+ \frac{{\Pi}_e^{~2}}{\omega\,(|{\Omega}_e|
-\omega - {\rm i}\,\nu)}.
\end{equation}
Suppose that $n=n(z)$. Furthermore, let
\begin{equation}
|{\Omega}_e| = \omega + |{\Omega}_e|'\,z,
\end{equation}
so that the electron cyclotron resonance is located at $z=0$. 
We also assume that $|{\Omega}_e|'>0$, so that the evanescent region
corresponds to $z<0$. 
It follows that in the immediate vicinity of the resonance
\begin{equation}
n^2 \simeq \frac{b}{z+ {\rm i}\,\epsilon},
\end{equation}
where
\begin{equation}
b = \frac{{\Pi}_e^{~2}}{\omega\,|{\Omega}_e|'},
\end{equation}
and 
\begin{equation}
\epsilon = -\frac{\nu}{|{\Omega_e}|'}.
\end{equation}
It can be seen that $\epsilon<0$, which is consistent with the
{\em absorption}\/  of incident wave energy by the resonant layer. The
approximate width of the resonant layer
is
\begin{equation}
\delta \sim |\epsilon| = \frac{\nu}{|{\Omega_e}|'}.
\end{equation}

Note that the damping mechanism, in this case collisions, controls the
{\em thickness}\/ of the resonant layer, but does not control the
amount of wave energy absorbed by the layer. In fact, in the simple
theory outlined above,
 {\em all}\/ of the incident wave energy is absorbed by the layer.

\section{Pulse Propagation}\label{s4.16}
Consider the situation under investigation in Sect.~\ref{s4.12}, in which a plane
wave, polarized in the $y$-direction, is launched along the $z$-axis, from
an antenna located at large positive $z$, and reflected from a cutoff located
at $z=0$. Up to now, we have only considered {\em infinite}\/ wave-trains, characterized
by a discrete frequency, $\omega$. Let us now consider the more realistic
case in which the antenna emits a finite {\em pulse}\/ of radio waves.

The pulse structure is conveniently represented as
\begin{equation}\label{e4.145}
E_y(t) = \int_{-\infty}^{\infty} F(\omega)\,{\rm e}^{-{\rm i} \, \omega \,t}\,d\omega,
\end{equation}
where $E_y(t)$ is the electric field produced by
 the antenna, which is assumed to lie at $z=a$. 
Suppose that the pulse is a signal of roughly constant (angular) 
frequency $\omega_0$,
which lasts a time $T$, where $T$  is long compared to $1/\omega_0$. It 
follows  that $F(\omega)$ possesses narrow maxima around $\omega=\pm
\omega_0$. In other words, 
only those frequencies which lie very close to the central
frequency $\omega_0$ play a significant role in the propagation of the pulse. 

Each component frequency of the pulse yields a wave which 
propagates {\em independently}\/ along the $z$-axis, in a manner specified by the
appropriate WKB solution [see Eqs.~(\ref{e4.101a})--(\ref{e4.101b})]. Thus, if Eq.~(\ref{e4.145})
specifies the signal at the antenna ({\em i.e.}, at
$z=a$), then the  signal at coordinate $z$ (where
$z<a$) 
is given by
\begin{equation}\label{e4.146}
E_y(z,t) = \int_{-\infty}^{\infty} \frac{F(\omega)}{n^{1/2}(\omega, z)}\,\,
{\rm e}^{\,{\rm i}\, \phi(\omega, z,t)}\,d\omega,
\end{equation}
where
\begin{equation}
\phi(\omega, z,t) = \frac{\omega}{c} \int_z^{a} \!n(\omega, z)\,dz - \omega \,t.
\end{equation}
Here, we have used $k_0=\omega/c$. 

Equation (\ref{e4.146}) can be regarded as a contour integral in $\omega$-space.
The quantity  $F/n^{1/2}$ is a relatively slowly varying function of
$\omega$, whereas the phase, $\phi$, is a large and rapidly varying
function of $\omega$. 
The rapid
oscillations of $\exp(\,{\rm i}\,\phi)$ over most of the path of
integration ensure that the integrand averages almost to zero. However,
this cancellation argument does not apply to places on the
integration path where the phase
is {\em stationary}: {\em i.e.}, 
 places where $\phi(\omega)$ has an extremum. The integral can, therefore, be
estimated by finding those points where $\phi(\omega)$ has a vanishing derivative,
evaluating (approximately) the integral in the neighbourhood of each of
 these points, and summing the contributions. This procedure is called
the {\em method of stationary phase}.

Suppose that $\phi(\omega)$ has a vanishing first derivative
at $\omega=\omega_s$. In the neighbourhood of this point,
$\phi(\omega)$ can be expanded as a Taylor series,
\begin{equation}
\phi(\omega) = \phi_s + \frac{1}{2} \phi''_s(\omega-\omega_s)^2+\cdots.
\end{equation}
Here, the subscript $s$ is used to indicate $\phi$ or its
second derivative evaluated at $\omega=\omega_s$. Since $F(\omega)/n^{1/2}(\omega,z)$
is slowly varying, the contribution to the integral from this
stationary phase point is approximately
\begin{equation}
E_{y\,s} \simeq \frac{F(\omega_s) \,{\rm e}^{\,{\rm i}\,\phi_s}}{n^{1/2}(\omega_s,z)}
\int_{-\infty}^{\infty} {\rm e}^{\,({\rm i}/2)\phi_s''(\omega-\omega_s)^2}\,d\omega.
\end{equation}
The above expression can be written in the form
\begin{equation}
E_{y\,s}\simeq  \frac{F(\omega_s) \,{\rm e}^{\,{\rm i}\,\phi_s}}{n^{1/2}(\omega_s,z)}
\sqrt{\frac{4\pi}{\phi_s''}}\int_0^\infty \left[
\cos(\pi \,t^2/2)+{\rm i}\,\sin(\pi\, t^2/2)\right]\,dt,
\end{equation}
where
\begin{equation}
\frac{\pi}{2}\, t^2 = \frac{1}{2}\, \phi_s'' \,(\omega-\omega_s)^2.
\end{equation}
The integrals in the above expression are
{\em Fresnel integrals},\footnote{M.~Abramowitz,
and I.A.~Stegun, {\em Handbook of Mathematical Functions}, (Dover, New York NY, 
1965), Sect.~7.3.} and can be shown to take the values
\begin{equation}
\int_0^\infty\cos(\pi \,t^2/2)\,dt = \int_0^\infty\sin(\pi\, t^2/2)\,dt
=\frac{1}{2}.
\end{equation}
It follows that
\begin{equation}\label{e4.153}
E_{y\,s}\simeq  \sqrt{\frac{2\pi\,{\rm i}}{\phi_s''}}
\, \frac{F(\omega_s)}{n^{1/2}(\omega_s, z)} \,{\rm e}^{\,{\rm i}\,\phi_s}.
\end{equation}
If there is more than one point of stationary phase in the range
of integration then the integral is approximated as a sum of terms like the
above. 

Integrals of the form (\ref{e4.146}) can be calculated exactly using the
{\em method of steepest decent}.\footnote{L\'{e}on Brillouin, 
{\em Wave Propagation and Group Velocity},
(Academic press, New York NY, 1960).} The stationary
phase approximation (\ref{e4.153}) agrees with the leading term of the
method of steepest decent (which is far more difficult to implement
than the method of stationary phase) provided that $\phi(\omega)$ is
real ({\em i.e.}, provided that
the stationary point lies on the real axis). If $\phi$ is complex, however, the stationary phase
method can yield erroneous results. 

It follows, from the above discussion,
  that the right-hand side of Eq.~(\ref{e4.146}) averages to a very small
value, expect
for those special values of $z$ and $t$ at which one of the points of stationary
 phase in $\omega$-space coincides with one of the peaks of $F(\omega)$. The
locus of these special values of $z$ and $t$ can obviously be regarded as the
equation of motion of the pulse  as it propagates along the $z$-axis. Thus, the equation of motion is
 specified by
\begin{equation}
\left(\frac{\partial\phi}{\partial\omega}\right)_{\omega=\omega_0} = 0,
\end{equation}
which yields
\begin{equation}\label{e4.155}
t = \frac{1}{c} \int_z^a \left[\frac{\partial(\omega \,n)}{\partial\omega}
\right]_{\omega=\omega_0}\, dz.
\end{equation}

Suppose that the $z$-velocity of a pulse of central frequency $\omega_0$
at coordinate $z$ is given by $-u_z(\omega_0,z)$. The differential
equation of motion of the pulse is then $dt = -dz/u_z$. This can be integrated,
using the boundary condition $z=a$ at $t=0$, to give the full equation
of motion:
\begin{equation}\label{e4.156}
t =\int_z^a \frac{dz}{u_z}.
\end{equation}
A comparison of Eqs.~(\ref{e4.155}) and (\ref{e4.156}) yields
\begin{equation}\label{e4.157}
u_z(\omega_0,z) = c\left/ \left\{\frac{\partial[\omega \,n(\omega,z)]}{\partial\omega}
\right\}_{\omega=\omega_0}\right..
\end{equation}
The velocity $u_z$ is  usually called the
 {\em group velocity}. It is easily demonstrated that
the above expression for the group velocity is
entirely  consistent with that 
given previously [see Eq.~(\ref{e4.54})].

The dispersion relation  for an electromagnetic plasma wave propagating
through an unmagnetized plasma is
\begin{equation}\label{e4.158}
n(\omega,z) = \left(1-\frac{{\Pi}_e^{~2}(z)}{\omega^2}\right)^{1/2}.
\end{equation}
Here, we have assumed that equilibrium quantities are functions of $z$ only,
and that the wave propagates along the $z$-axis. 
The phase velocity
of waves of frequency $\omega$
propagating along the $z$-axis is given
by
\begin{equation}
v_z(\omega,z) = \frac{c}{n(\omega,z)} = c\,\left(1-\frac{{\Pi}_e^{~2}(z)}{\omega^2}\right)^{-1/2}.
\end{equation}
According to Eqs.~(\ref{e4.157}) and (\ref{e4.158}), the corresponding group velocity
is
\begin{equation}
u_z(\omega,z) = c \,\left(1-\frac{{\Pi}_e^{~2}(z)}{\omega^2}\right)^{1/2}.
\end{equation}
It follows that
\begin{equation}
v_z\,u_z = c^2.
\end{equation}
It is assumed that ${\Pi}_e(0) = \omega$, 
and ${\Pi}_e(z) < \omega$ for $z>0$, which implies
that the reflection point corresponds to $z=0$.
Note that the phase velocity is always greater than the velocity of
light in vacuum, whereas the group velocity is always less than this
velocity.
Note, also,  that as the reflection point, $z=0$, is approached from positive $z$,
the phase velocity tends to infinity, whereas the group velocity tends
to zero.

 Although we have only analyzed the motion of the
pulse as it travels from the antenna to the reflection point, it is
easily demonstrated that the speed of the reflected pulse at position
$z$ is the same as that of the incident pulse. In other words, the group velocities
of pulses traveling in opposite directions are of equal magnitude. 

\section{Ray Tracing}
Let us now generalize the preceding analysis so that we can deal with pulse
propagation though a three-dimensional magnetized plasma.

A general wave problem can be written as a set of $n$ coupled, linear, homogeneous,
first-order, partial-differential equations, which take the form
\begin{equation}\label{e4.162}
{\bf M}(\,{\rm i}\,\partial/\partial t, -{\rm i}\,\nabla,
{\bf r}, t)\,{\bpsi} = {\bf 0}.
\end{equation}
The vector-field ${\bpsi}({\bf r}, t)$ has $n$ components ({e.g.}, ${\bpsi}$
might consist of ${\bf E}$, ${\bf B}$, ${\bf j}$, and ${\bf V}$) characterizing
some small disturbance, and ${\bf M}$ is an $n\times n$ matrix characterizing the
undisturbed plasma. 

The lowest order WKB  approximation  is premised on the assumption that
${\bf M}$ depends so weakly on ${\bf r}$ and $t$ that all of the
spatial and temporal dependence of the components of ${\bf \psi}({\bf r}, t)$
is specified by a common factor $\exp(\,{\rm i}\,\phi)$.
Thus,
Eq.~(\ref{e4.162}) reduces to
\begin{equation}\label{e4.163}
{\bf M}(\omega, {\bf k}, {\bf r}, t)\, {\bpsi} = {\bf 0},
\end{equation}
where
\begin{eqnarray}\label{e4.164a}
{\bf k} &\equiv & \nabla\phi,\\[0.5ex]
\omega &\equiv & - \frac{\partial\phi}{\partial t}.\label{e4.164b}
\end{eqnarray}
In general, Eq.~(\ref{e4.163}) has many solutions, corresponding to the many different
types and polarizations of wave which can propagate through the plasma in question,
all of which satisfy the dispersion relation
\begin{equation}\label{e4.165}
{\cal M}(\omega,{\bf k}, {\bf r}, t) = 0,
\end{equation}
where ${\cal M} \equiv {\rm det}({\bf M})$. 
As is easily demonstrated (see Sect.~\ref{s4.11}), the WKB approximation is valid
provided that the characteristic
variation length-scale and variation time-scale of the plasma are much longer than
the wave-length, $2\pi/k$, and the period, $2\pi/\omega$, respectively,
of the wave in question.

Let us concentrate on one particular solution of Eq.~(\ref{e4.163}) ({\em e.g.},
on one particular type of plasma wave). For this solution, the dispersion
relation (\ref{e4.165}) yields
\begin{equation}\label{e4.166}
\omega = {\Omega}({\bf k}, {\bf r}, t):
\end{equation}
{\em i.e.}, the dispersion relation yields a
 unique frequency for a wave of a given wave-vector, ${\bf k}$, located
at a given point, $({\bf r}, t)$, in space and time. There is also a unique ${\bpsi}$ associated
with this frequency, which is obtained from Eq.~(\ref{e4.163}). To lowest order, we can
neglect the variation of ${\bpsi}$ with ${\bf r}$ and $t$. 
A general pulse solution is written
\begin{equation}\label{e4.167}
{\bpsi}({\bf r}, t) = \int F({\bf k})\,{\bpsi}\,{\rm e}^{\,{\rm i}\,
\phi}\,d^3{\bf k},
\end{equation}
where (locally)
\begin{equation}
\phi = {\bf k}\!\cdot\!{\bf r} - {\Omega}\,t,
\end{equation}
and $F$ is a function which specifies the initial structure of the pulse
in ${\bf k}$-space.

The integral (\ref{e4.167}) averages to zero, except at a point of {\em stationary
phase}, where $\nabla_{\bf k} \phi=0$ (see Sect.~\ref{s4.16}). Here, $\nabla_{\bf k}$ 
is the  ${\bf k}$-space
gradient operator. It follows that the (instantaneous) trajectory of the pulse
matches that of a point of stationary phase: {\em i.e.},
\begin{equation}
\nabla_{\bf k}\phi = {\bf r} - {\bf v}_g\,t=0,
\end{equation}
where
\begin{equation}
{\bf v}_g = \frac{\partial{\Omega}}{\partial {\bf k}}
\end{equation}
is the {\em group velocity}. Thus, the instantaneous velocity of a pulse is
always
equal to the local group velocity.

Let us now determine how the wave-vector, ${\bf k}$, and frequency, $\omega$,
of a pulse evolve as the pulse propagates through the
plasma. We start from the cross-differentiation rules
[see Eqs.~(\ref{e4.164a})--(\ref{e4.164b})]: 
\begin{eqnarray}\label{e4.171a}
\frac{\partial k_i}{\partial t} + \frac{\partial\omega}{\partial r_i} &=& 0,\\[0.5ex]
\frac{\partial k_j}{\partial r_i} - \frac{\partial k_i}{\partial r_j} &=& 0.\label{e4.171b}
\end{eqnarray}
Equations~(\ref{e4.166}) and (\ref{e4.171a})--(\ref{e4.171b}) yield (making use of the Einstein summation
convention)
\begin{equation}
\frac{\partial k_i}{\partial t} + \frac{\partial{\Omega}}{\partial k_j}
\frac{\partial k_j}{\partial r_i} +\frac{\partial{\Omega}}{\partial r_i}
= \frac{\partial k_i}{\partial t} + \frac{\partial{\Omega}}{\partial k_j}
\frac{\partial k_i}{\partial r_j} +\frac{\partial{\Omega}}{\partial r_i}=0,
\end{equation}
or
\begin{equation}
\frac{d{\bf k}}{dt}\equiv\frac{\partial {\bf k}}{\partial t} + ({\bf v}_g\!\cdot\!\nabla)\,
{\bf k} = -\nabla{\Omega}.
\end{equation}
In other words, the variation of ${\bf k}$, as seen in a frame co-moving with
the pulse, is determined by the spatial gradients in ${\Omega}$. 

Partial differentiation of Eq.~(\ref{e4.166}) with respect to $t$ gives
\begin{equation}
\frac{\partial\omega}{\partial t} = \frac{\partial{\Omega}}{\partial k_j}
\frac{\partial k_j}{\partial t} + \frac{\partial{\Omega}}{\partial t}
= -\frac{\partial {\Omega}}{\partial k_j}\frac{\partial\omega}{\partial r_j}
 + \frac{\partial{\Omega}}{\partial t},
\end{equation}
which can be written
\begin{equation}
\frac{d\omega}{dt} \equiv
\frac{\partial\omega}{\partial t} + ({\bf v}_g\!\cdot\!\nabla)\, \omega = 
\frac{\partial {\Omega}}{\partial t}.
\end{equation}
In other words, the variation of $\omega$, as seen in a frame co-moving with
the pulse, is determined by the time variation of ${\Omega}$. 

According to the above analysis, the evolution of a pulse 
propagating though a spatially and temporally non-uniform
plasma can be determined  by solving the
{\em ray equations}:
\begin{eqnarray}
\frac{d{\bf r}}{dt} &=& \frac{\partial{\Omega}}{\partial {\bf k}},\\[0.5ex]
\frac{d{\bf k}}{dt} &=& -\nabla{\Omega},\\[0.5ex]
\frac{d\omega}{d t} &=& \frac{\partial{\Omega}}{\partial t}.
\end{eqnarray}
The above equations are conveniently rewritten in terms of the dispersion
relation (\ref{e4.165}):
\begin{eqnarray}\label{e4.177a}
\frac{d{\bf r}}{dt} &=& -\frac{\partial{\cal M}/\partial {\bf k} }
{\partial{\cal M}/\partial\omega},\\[0.5ex]
\frac{d{\bf k}}{dt} &=& \frac{\partial{\cal M}/\partial{\bf r} }
{\partial{\cal M}/\partial\omega},\\[0.5ex]
\frac{d\omega}{d t} &=& -\frac{\partial{\cal M}/\partial t }
{\partial{\cal M}/\partial\omega}.\label{e4.177c}
\end{eqnarray}

Note, finally, that the variation in the amplitude of the pulse, as it
propagates through though the plasma, can only be determined by expanding
the WKB solutions to higher order (see Sect.~\ref{s4.11}).

\section{Radio Wave Propagation Through the Ionosphere}
To a first approximation, the Earth's ionosphere consists of an unmagnetized,
horizontally stratified, partially ionized gas. The dispersion
relation for the electromagnetic plasma wave takes the form [see Eq.~(\ref{e4.78})]
\begin{equation}\label{e4.178}
{\cal M} = \omega^2 - k^2 c^2 - {\Pi}_e^{~2} = 0,
\end{equation}
where 
\begin{equation}
{\Pi}_e = \sqrt{\frac{N\,e^2}{\epsilon_0\,m_e}}.
\end{equation}
Here, $N=N(z)$ is the density of free electrons in the ionosphere, and
$z$ is a coordinate which measures height above the surface of the Earth. ({\em N.B.},
The curvature of the Earth is neglected in the following analysis.) 

Now,
\begin{eqnarray}
\frac{\partial{\cal M}}{\partial\omega} &=& 2\,\omega,\\[0.5ex]
\frac{\partial{\cal M}}{\partial {\bf k}} &=& - 2\,{\bf k}\,c^2,\\[0.5ex]
\frac{\partial{\cal M}}{\partial{\bf r}} &=& -\nabla{\Pi}_e^{~2},\\[0.5ex]
\frac{\partial{\cal M}}{\partial t} &=& 0.
\end{eqnarray}
Thus, the ray equations, (\ref{e4.177a})--(\ref{e4.177c}), yield
\begin{eqnarray}\label{e4.181a}
\frac{d{\bf r}}{dt}& = &\frac{{\bf k}\,\,c^2}{\omega},\\[0.5ex]
\frac{d{\bf k}}{dt} &=& - \frac{\nabla{\Pi}_e^{~2}}{2\,\omega},\\[0.5ex]
\frac{d\omega}{dt} &=& 0.\label{e4.181c}
\end{eqnarray}
Note that the frequency of a radio pulse does not change as it
propagates through the ionosphere, provided that $N(z)$ does not vary in time.
 It is clear, from Eqs.~(\ref{e4.181a})--(\ref{e4.181c}), and the fact that ${\Pi}_e={\Pi}_e(z)$, that
a radio pulse which starts off at ground level propagating in the $x$-$z$ plane, say,
will continue to propagate in this plane. 

For pulse propagation in the $x$-$z$ plane, we have
\begin{eqnarray}
\frac{dx}{dt} &=& \frac{k_x\,c^2}{\omega},\label{e4.182a}\\[0.5ex]
\frac{dz}{dt} &=& \frac{k_z\,c^2}{\omega},\label{e4.182b}\\[0.5ex]
\frac{d k_x}{dt} &=& 0.\label{e4.182c}
\end{eqnarray}
The dispersion relation (\ref{e4.178}) yields
\begin{equation} \label{e4.183}
n^2 = \frac{(k_x^{~2} + k_z^{~2})\,c^2}{\omega^2} = 1-\frac{{\Pi}_e^{~2}}{\omega^2},
\end{equation}
where $n(z)$ is the refractive index. 


We assume that $n=1$ at $z=0$, which is equivalent to the
reasonable assumption that the atmosphere is non-ionized at ground level. 
It follows from  Eq.~(\ref{e4.182c}) that
\begin{equation}\label{e4.184}
k_x= k_x(z=0) = \frac{\omega}{c}\,S,
\end{equation}
where $S$ is the sine of the angle of incidence of the pulse, with respect to
the vertical axis, at ground level. Equations (\ref{e4.183}) and (\ref{e4.184})  yield
\begin{equation}\label{e4.185}
k_z = \pm \frac{\omega}{c}\,\sqrt{n^2-S^2}.
\end{equation}
According to Eq.~(\ref{e4.182b}), the plus sign corresponds to the upward trajectory
of the pulse, whereas the minus sign corresponds to the downward trajectory.
Finally, Eqs.~(\ref{e4.182a}), (\ref{e4.182b}), (\ref{e4.184}), and (\ref{e4.185})
 yield the equations of motion of the pulse:
\begin{eqnarray}
\frac{dx}{dt} &=& c\,S,\\[0.5ex]
\frac{dz}{dt} &=& \pm c\, \sqrt{n^2-S^2}.
\end{eqnarray}
The pulse attains its maximum altitude, $z=z_0$, when
\begin{equation}\label{e4.187}
n(z_0) = |S|.
\end{equation}
The total distance traveled by the pulse ({\em i.e.}, the distance 
from its launch
point to the point where it intersects the Earth's surface again) is
\begin{equation}
x_0 = 2\,S\int_0^{z_0(S)} \frac{dz}{\sqrt{n^2(z)-S^2}}.
\end{equation}

In the limit in which the radio pulse is launched vertically ({\em i.e.},
$S=0$) into the ionosphere, the turning point condition (\ref{e4.187}) reduces to
that characteristic of a cutoff ({\em i.e.}, $n=0$). The WKB turning point
described in Eq.~(\ref{e4.187}) is a generalization of the conventional turning point,
which occurs when $k^2$ changes sign. Here, $k_z^{~2}$ changes sign, whilst
$k_x^{~2}$ and $k_y^{~2}$ are constrained by symmetry ({\em i.e.}, $k_x$ is constant,
and $k_y$ is zero). 

According to Eqs.~(\ref{e4.181a})--(\ref{e4.181c}) and (\ref{e4.183}), the equation of motion of the pulse
can also be written
\begin{equation}
\frac{d^2{\bf r}}{dt^2} = \frac{c^2}{2}\,\nabla n^2.
\end{equation}
It follows that the trajectory of the pulse is the same as that of a
particle moving in the gravitational potential $-c^2\,n^2/2$. Thus, if
$n^2$ decreases {\em linearly} with increasing height above the
ground [which is the case if $N(z)$ increases linearly with $z$] then the
trajectory of the pulse is a {\em parabola}. 