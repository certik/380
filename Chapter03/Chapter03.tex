\chapter {Plasma Fluid Theory}\label{s3}
\section{Introduction}\label{s3.1}
In plasma fluid theory, a plasma is characterized by a few local parameters---such as the particle density, the kinetic temperature, and the flow
velocity---the time evolution of which are determined by means of {\em fluid
equations}. These equations are analogous to, but generally more complicated
than, the equations of hydrodynamics.

Plasma physics can be viewed formally as a closure of Maxwell's equations
by means of {\em constitutive relations}: {\em i.e.}, expressions
for the charge density, $\rho_c$, and the current density, ${\bf j}$, in terms of
the electric and magnetic fields, ${\bf E}$ and ${\bf B}$. Such relations are
easily expressed in terms of the microscopic distribution functions,
${\cal F}_s$, for each plasma species. In fact,
\begin{eqnarray}\label{e3.1a}
\rho_c &=& \sum_s e_s \int\! {\cal F}_s({\bf r}, {\bf v}, t)\,d^3{\bf v},\\[0.5ex]
{\bf j} &=& \sum_s e_s \int {\bf v}\,{\cal F}_s({\bf r}, {\bf v}, t)\,d^3
{\bf v}.\label{e3.1b}
\end{eqnarray}
Here, ${\cal F}_s({\bf r}, {\bf v}, t)$ is the exact, ``microscopic'' phase-space
density of plasma species $s$ (charge $e_s$, mass $m_s$) near
point $({\bf r}, {\bf v})$  at time $t$. The distribution
function ${\cal F}_s$ is normalized such that its velocity integral is equal to the
particle density in coordinate space. Thus,
\begin{equation}
\int \!{\cal F}_s({\bf r}, {\bf v}, t)\,d^3
{\bf v} = n_s({\bf r}, t),
\end{equation}
where $n_s({\bf r},t)$ is the number (per unit volume) of species-$s$ particles
near point ${\bf r}$ at time $t$. 

If we could determine each ${\cal F}_s({\bf r},{\bf v}, t)$ in terms of the
electromagnetic fields, then Eqs.~(\ref{e3.1a})--(\ref{e3.1b}) would immediately give us the
desired constitutive relations. Furthermore, it is easy to see, in principle,
how each distribution function evolves. Phase-space conservation
requires that
\begin{equation}\label{e3.3}
\frac{\partial{\cal F}_s}{\partial t} + {\bf v}\cdot\nabla{\cal F}_s
+ {\bf a}_s\cdot\nabla_v{\cal F}_s = 0,
\end{equation}
where $\nabla_v$ is the velocity space grad-operator, and
\begin{equation}
{\bf a}_s = \frac{e_s}{m_s}\,({\bf E} + {\bf v}\times{\bf B})
\end{equation}
is the species-$s$ particle acceleration under the influence of the ${\bf E}$ and
${\bf B}$ fields. 

It would appear that the distribution functions for the various plasma
species, from which
the constitutive relations are trivially obtained, are determined by a set of
rather harmless looking first-order partial differential equations. At this
stage, we might wonder why, 
if plasma dynamics is apparently so  simple when written in terms
of distribution functions, we need a fluid description of plasma dynamics
at all. It is not at all obvious that fluid theory represents an advance. 

The above argument is misleading for several reasons. However, by far the
most serious flaw is the view of Eq.~(\ref{e3.3}) as a tractable equation.
Note that this equation is easy to derive, because it is exact, taking into
account all scales from the microscopic to the macroscopic. Note, in
particular, that there is {\em no}\/ statistical averaging involved in Eq.~(\ref{e3.3}). 
It follows that the microscopic distribution function ${\cal F}_s$ is
essentially a sum of Dirac delta-functions, each following the
detailed trajectory of a single particle. Furthermore, the electromagnetic
fields in Eq.~(\ref{e3.3}) are horribly spiky and chaotic on microscopic scales. In other words, solving Eq.~(\ref{e3.3}) amounts to nothing less than solving the classical
electromagnetic many-body problem---a completely hopeless task. 

A much more useful and tractable equation can be extracted from Eq.~(\ref{e3.3})
by ensemble averaging. The average distribution function,
\begin{equation}
\bar{\cal F}_s \equiv \langle {\cal F}_s\rangle_{\rm ensemble},
\end{equation}
is sensibly smooth, and is closely related to actual experimental measurements.
Similarly, the ensemble averaged electromagnetic fields are also smooth.
Unfortunately, the extraction of an ensemble averaged
equation from Eq.~(\ref{e3.3}) is a mathematically challenging exercise, and always
requires severe approximation. The problem is that, since the
exact electromagnetic fields depend on particle trajectories, ${\bf E}$ and
${\bf B}$ are {\em not}\/ statistically independent of ${\cal F}_s$. In other
words, the nonlinear acceleration term in Eq.~(\ref{e3.3}),
\begin{equation}
\left\langle{\bf a}_s\cdot\nabla_v {\cal F}_s\right\rangle_{\rm ensemble}
\neq \bar{\bf a}_s\cdot\nabla_v \bar{{\cal F}}_s,
\end{equation}
involves {\em correlations}\/ which need to be evaluated explicitly.
In the following, we introduce the short-hand
\begin{equation}
f_s\equiv \bar{{\cal F}}_s.
\end{equation}

The traditional goal of kinetic theory is to analyze the correlations,
using approximations tailored to the parameter regime of interest, and
thereby express the average  acceleration term in terms of 
$f_s$ and
the average electromagnetic fields alone. Let us assume that this ambitious
task has already been completed, giving an expression of the form
\begin{equation}
\left\langle{\bf a}_s\cdot\nabla_v {\cal F}_s\right\rangle_{\rm ensemble}
= \bar{\bf a}_s\cdot\nabla_v \bar{{\cal F}}_s -C_s(f),
\end{equation}
where $C_s$ is a generally extremely complicated operator which
accounts for the correlations. Since the most important correlations result from
close encounters between particles, $C_s$ is called the {\em collision operator}\/
(for species $s$). 
It is not necessarily a linear operator, and usually involves the distribution
functions of both species (the subscript in the
argument of $C_s$ is omitted for this reason). Hence, the ensemble averaged
version of Eq.~(\ref{e3.3}) is written
\begin{equation}\label{e3.9}
\frac{\partial{f_s}}{\partial t} + {\bf v}\cdot\nabla f_s
+ \bar{{\bf a}}_s\cdot\nabla_vf_s = C_s(f).
\end{equation}

In general, the above equation is very difficult to solve, because of the
complexity of the collision operator. However, there are some situations
where collisions can be completely neglected. In this case, the
apparent simplicity of Eq.~(\ref{e3.3}) is not deceptive. A useful kinetic
description is obtained by just  ensemble averaging this equation to
give
\begin{equation}
\frac{\partial{f_s}}{\partial t} + {\bf v}\cdot\nabla f_s
+ \bar{{\bf a}}_s\cdot\nabla_vf_s = 0.
\end{equation}
The above equation, which is known as the {\em Vlasov equation}, is tractable
in sufficiently simple geometry. Nevertheless, the fluid approach
has much
to offer even in the Vlasov limit: it has intrinsic advantages that
weigh decisively in its favour in almost every situation.

Firstly, fluid equations possess the key simplicity of involving
fewer dimensions: three spatial dimensions instead of six phase-space
dimensions. This advantage is especially important in computer simulations.

Secondly, the fluid description is intuitively appealing. We 
immediately understand the
significance of fluid quantities such as density and temperature, whereas
the significance of distribution functions is far less obvious.
Moreover, fluid variables are relatively easy to measure in experiments,
whereas, in most cases, it is extraordinarily difficult to measure a
distribution function accurately. There seems
remarkably  little point in centering our
theoretical description of plasmas on something that we cannot
generally measure. 

Finally, the kinetic approach to plasma physics is spectacularly
inefficient. The species distribution functions $f_s$ provide vastly more
information than is needed to obtain the constitutive relations.
After all, these relations only depend on the two lowest moments
of the species distribution functions. Admittedly, fluid theory cannot
generally compute $\rho_c$ and ${\bf j}$ without  reference to other
higher moments of the distribution functions, but it can be regarded as
an attempt to impose some efficiency on the task of dynamical closure. 

\section{Moments of the Distribution Function}
The $k$th moment of the (ensemble averaged) distribution function
$f_s({\bf r}, {\bf v}, t)$ is written
\begin{equation}
{\bf M}_k({\bf r}, t) = \int {\bf v v\cdots v}\,f_s({\bf r},{\bf v}, t)\,d^3{\bf v},
\end{equation}
with $k$ factors of ${\bf v}$. Clearly, ${\bf M}_k$ is a tensor of rank $k$.

The set $\{{\bf M}_k, k=0,1,2,\cdots\}$
can be viewed as an alternative description of the distribution function, which,
indeed, uniquely specifies $f_s$ when the latter is sufficiently smooth. For example,
a (displaced) Gaussian distribution is uniquely specified by three
moments: $M_0$, the vector ${\bf M}_1$, and the scalar formed by contracting
${\bf M}_2$. 

The low-order moments all have names and simple physical interpretations.
First, we have the (particle) {\em density},
\begin{equation}
n_s({\bf r},t) = \int f_s({\bf r}, {\bf v},t)\,d^3{\bf v},
\end{equation}
and the particle {\em flux density}, 
\begin{equation}
n_s\,{\bf V}_s({\bf r}, t) = \int 
{\bf v}\,f_s({\bf r}, {\bf v},t)\,d^3{\bf v}.
\end{equation}
The quantity ${\bf V}_s$ is, of course, the {\em flow velocity}. Note that
the electromagnetic sources, (\ref{e3.1a})--(\ref{e3.1b}), are determined by these lowest
moments:
\begin{eqnarray}
\rho_c &=& \sum_s e_s n_s,\\[0.5ex]
{\bf j} &=& \sum_s e_s n_s\,{\bf V}_s.
\end{eqnarray}

The second-order moment, describing the flow of momentum in the
laboratory frame, is called the {\em stress tensor}, and denoted by
\begin{equation}
{\bf P}_s({\bf r}, t) =  \int
m_s\,{\bf v}{\bf v}\, f_s({\bf r}, {\bf v},t)\,d^3{\bf v}.
\end{equation}
Finally, there is an important third-order moment
measuring the {\em energy flux density},
\begin{equation}
{\bf Q}_s({\bf r}, t) =  \int
\frac{1}{2}\,m_s\,v^2\,{\bf v}\, f_s({\bf r}, {\bf v},t)\,d^3{\bf v}.
\end{equation}


It is often convenient to measure the second- and third-order moments in the rest-frame of the species under consideration. In this case, the
moments assume different names: the stress tensor measured in the rest-frame
is called the {\em pressure tensor}, ${\bf p}_s$, whereas the energy flux
density becomes the {\em heat flux density}, ${\bf q}_s$. We introduce the
relative velocity,
\begin{equation}
{\bf w}_s\equiv {\bf v} - {\bf V}_s,
\end{equation}
in order to write
\begin{equation}
{\bf p}_s({\bf r}, t) =  \int
m_s\,{\bf w}_s{\bf w}_s\, f_s({\bf r}, {\bf v},t)\,d^3{\bf v},
\end{equation}
and
\begin{equation}
{\bf q}_s({\bf r}, t) =  \int
\frac{1}{2}\,m_s\,w_s^{~2}\,{\bf w}_s\, f_s({\bf r}, {\bf v},t)\,d^3{\bf v}.
\end{equation}

The trace of the pressure tensor measures the ordinary (or ``scalar'') pressure,
\begin{equation}
p_s\equiv \frac{1}{3}\,{\rm Tr}\,({\bf p}_s).
\end{equation}
Note that $(3/2)\,p_s$ is the kinetic energy density of species $s$:
\begin{equation}\label{e3.21}
\frac{3}{2}\,p_s = \int \frac{1}{2}\,m_s\,w_s^{~2} \,f_s\,d^3{\bf v}.
\end{equation}
In thermodynamic equilibrium, the distribution function becomes a Maxwellian
characterized by some temperature $T$, and Eq.~(\ref{e3.21}) yields $p=n\,T$. It
is, therefore, natural to define the (kinetic) temperature as
\begin{equation}
T_s \equiv \frac{p_s}{n_s}.
\end{equation}

Of course, the moments measured in the two different frames are related.
By direct substitution, it is easily verified that
\begin{eqnarray}\label{e3.23a}
{\bf P}_s &=& {\bf p}_s + m_s n_s\,{\bf V}_s{\bf V}_s,\\[0.5ex]
{\bf Q}_s &=& {\bf q}_s + {\bf p}_s\cdot{\bf V}_s + \frac{3}{2}\,p_s\,{\bf V}_s
+\frac{1}{2}\,m_s n_s\,V_s^{~2}\, {\bf V}_s.\label{e3.23b}
\end{eqnarray}
 
\section{Moments of the Collision Operator}\label{s3.3}
Boltzmann's famous collision operator for a neutral gas considers only
binary collisions, and is, therefore, bilinear in the distribution functions
of the two colliding species:
\begin{equation}
C_s(f) =\sum_{s'} C_{ss'}(f_s, f_{s'}),
\end{equation}
where $C_{ss'}$ is linear in each of its arguments. Unfortunately, such bilinearity
is not strictly valid for the case of Coulomb collisions in a plasma. 
Because of the long-range nature of the Coulomb interaction, the closest analogue
to ordinary two-particle interaction is mediated by Debye shielding, an intrinsically
many-body effect. Fortunately, the departure from bilinearity is {\em logarithmic}\/
in a weakly coupled plasma, and can, therefore,  be neglected to a fairly good approximation
(since a logarithm is a comparatively weakly varying function). 
Thus, from now
on, $C_{ss'}$ is presumed to be bilinear. 

It is important to realize that there is no simple relationship between
the quantity 
$C_{ss'}$, which describes the effect {\em on}\/ species $s$ of
collisions {\em with}\/ species $s'$, and the quantity $C_{s's}$. The two operators
can have quite different mathematical forms (for example, where the masses
$m_s$ and $m_{s'}$ are disparate), and they appear in different equations. 

Neutral particle collisions are characterized by Boltzmann's collisional
conservation laws: the collisional process conserves particles, momentum,
and energy at each point. We expect the same {\em local}\/ conservation
laws to hold for Coulomb collisions in a plasma: the maximum range of the
Coulomb force in a plasma is the Debye length, which is assumed to
be vanishingly small.

Collisional {\em particle conservation}\/ is expressed by
\begin{equation}
\int C_{ss'} \,d^3{\bf v} =0.
\end{equation}

Collisional {\em momentum conservation} requires that
\begin{equation}
\int m_s\,{\bf v}\,C_{ss'}\,d^3{\bf v} = - \int m_{s'}\,{\bf v}\,C_{s's}\,d^3{\bf v}.
\end{equation}
That is, the net momentum exchanged between species $s$ and $s'$ must vanish.
It is useful to introduce the rate of collisional momentum exchange, called the
collisional friction force, or simply the {\em friction force}:
\begin{equation}
{\bf F}_{ss'}\equiv \int m_s\,{\bf v}\,C_{ss'}\,d^3{\bf v}.
\end{equation}
Clearly, ${\bf F}_{ss'}$ is the momentum-moment of the collision operator. 
The total friction force experienced by species $s$ is
\begin{equation}
{\bf F}_s \equiv \sum_{s'} {\bf F}_{ss'}.
\end{equation}
Momentum conservation is expressed in detailed form as
\begin{equation}
{\bf F}_{ss'} = -{\bf F}_{s's},
\end{equation}
and in non-detailed form as
\begin{equation}\label{e3.30}
\sum_s {\bf F}_s = {\bf 0}.
\end{equation}

Collisional {\em energy conservation}\/ requires the quantity
\begin{equation}
W_{Lss'} \equiv \int \frac{1}{2}\,m_s\,v^2\,C_{ss'}\,d^3{\bf v}
\end{equation}
to be conserved in collisions: {\em i.e.},
\begin{equation}
W_{Lss'} + W_{Ls's} =0.
\end{equation}
Here, the $L$-subscript indicates that the kinetic energy of both
species is measured in the same ``lab'' frame. Because of Galilean invariance,
the choice of this common reference frame does not matter.

An alternative collisional energy-moment is
\begin{equation}
W_{ss'} \equiv \int \frac{1}{2}\,m_s\,w_s^{~2}\,C_{ss'}\,d^3{\bf v}:
\end{equation}
{\em i.e.}, the kinetic energy change experienced by species $s$, due to
collisions with species $s'$, measured in the rest frame of species $s$.
The total energy change for species $s$ is, of course, 
\begin{equation}
W_s \equiv \sum_{s'} W_{ss'}.
\end{equation}
It is easily verified that
\begin{equation}
W_{Lss'} = W_{ss'} + {\bf V}_s\cdot{\bf F}_{ss'}.
\end{equation}
Thus, the collisional energy conservation law can be written
\begin{equation}
W_{ss'} +W_{s's} +({\bf V}_s-{\bf V}_{s'})\cdot {\bf F}_{ss'} = 0,
\end{equation} 
or in non-detailed form
\begin{equation}\label{e3.37}
\sum_s (W_s + {\bf V}_s\cdot{\bf F}_s) = 0.
\end{equation}

\section{Moments of the Kinetic Equation}
We obtain fluid equations by taking appropriate moments of the ensemble-average 
kinetic equation, (\ref{e3.9}). In the following, we suppress all ensemble-average over-bars
for ease of notation. It is convenient to rearrange the acceleration
term,
\begin{equation}
{\bf a}_s\cdot\nabla_v f_s = \nabla_v\cdot({\bf a}_s\,f_s).
\end{equation}
The two forms are equivalent  because flow in velocity space under the
Lorentz force is incompressible: {\em i.e.},
\begin{equation}
\nabla_v\cdot {\bf a}_s = 0.
\end{equation}
Thus, Eq.~(\ref{e3.9}) becomes
\begin{equation}\label{e3.40}
\frac{\partial f_s}{\partial t} + 
\nabla\cdot({\bf v}\,f_s) +\nabla_v\cdot({\bf a}_s\,f_s) = C_s(f).
\end{equation}
The rearrangement of the flow term is, of course, trivial, since ${\bf v}$
is independent of ${\bf r}$. 

The $k$th moment of the ensemble-average kinetic equation is obtained
by multiplying the above equation by $k$ powers of ${\bf v}$ and integrating
over velocity space. The flow term is simplified by pulling the
divergence outside the velocity integral. The acceleration term is treated by partial
integration. Note that these two terms couple the $k$th moment
to the $(k+1)$th and $(k-1)$th moments, respectively. 

Making use of the collisional conservation laws, the zeroth moment of Eq. (\ref{e3.40})
yields the {\em continuity equation}\/ for species $s$:
\begin{equation}\label{e3.41}
\frac{\partial n_s}{\partial t} + \nabla\cdot(n_s\,{\bf V}_s) = 0.
\end{equation}
Likewise, the first moment gives the {\em momentum conservation equation}\/
for species $s$:
\begin{equation}\label{e3.42}
\frac{\partial(m_s n_s\,{\bf V}_s)}{\partial t}
+ \nabla\cdot{\bf P}_s -e_s n_s({\bf E} + {\bf V}_s\times {\bf B}) ={\bf F}_s.
\end{equation}
Finally, the contracted second moment yields the {\em energy conservation
equation}\/ for species $s$:
\begin{equation}\label{e3.43}
\frac{\partial}{\partial t}\left( \frac{3}{2}\,p_s + \frac{1}{2}\,m_s n_s\,V_s^{~2}
\right) + \nabla\cdot{\bf Q}_s - e_s n_s\,{\bf E}\cdot{\bf V}_s = W_s
+ {\bf V}_s\cdot {\bf F}_s.
\end{equation}

The interpretation of Eqs.~(\ref{e3.41})--(\ref{e3.43}) as {\em conservation laws}\/ is 
straightforward. Suppose that $G$ is some physical quantity ({\em e.g.},
total number of particles, total energy, \ldots), and $g({\bf r}, t)$
is its density:
\begin{equation}
G = \int g\,d^3{\bf r}.
\end{equation}
If $G$ is conserved then $g$ must evolve according to
\begin{equation}
\frac{\partial g}{\partial t} + \nabla\cdot {\bf g} = {\mit\Delta} g,
\end{equation}
where ${\bf g}$ is the flux density of $G$, and ${\mit\Delta}g$ is the
local rate per unit volume at which $G$ is created or exchanged with other
entities in the fluid. Thus, the density of $G$ at some point
changes because there is net flow of $G$ towards or away from that
point (measured by the divergence term), or because of local
sources or sinks of $G$ (measured by the right-hand side). 

Applying this reasoning to Eq.~(\ref{e3.41}), we see that $n_s\,{\bf V}_s$ is indeed the
species-$s$ particle flux density, and that there are no local sources or sinks of 
species-$s$ particles.\footnote{In general, this is not true. Atomic or nuclear processes
operating in a plasma can give rise to local sources and sinks of particles
of various species. However, if a plasma is sufficiently hot to be completely
ionized, but still cold enough to prevent nuclear reactions from
occurring, then
such sources and sinks are usually negligible.} From Eq.~(\ref{e3.42}), we
see that the stress tensor ${\bf P}_s$ is the species-$s$ momentum flux density, and that 
the species-$s$ momentum is changed locally by the Lorentz force and by collisional
friction with other species. Finally, from Eq.~(\ref{e3.43}), we see that
${\bf Q}_s$ is indeed the species-$s$ energy flux density, and that the
species-$s$ energy is changed locally  by electrical work, energy exchange with
other species, and frictional heating. 

\section{Fluid Equations}
It is conventional to rewrite our fluid equations in terms of the pressure tensor,
${\bf p}_s$, 
and the heat flux density, ${\bf q}_s$. Substituting from Eqs.~(\ref{e3.23a})--(\ref{e3.23b}), and performing a little
 tensor
algebra,  Eqs.~(\ref{e3.41})--(\ref{e3.43})  reduce to:
\begin{eqnarray}\label{e3.46a}
\frac{dn_s}{dt} + n_s\,\nabla\!\cdot\!{\bf V}_s &=& 0,\\[0.5ex]
m_s n_s\,\frac{d {\bf V}_s}{dt} + \nabla\!\cdot \!{\bf p}_s - e_s n_s
({\bf E} + {\bf V}_s\times {\bf B})& =& {\bf F}_s,\\[0.5ex]
\frac{3}{2}\frac{d p_s}{dt} + \frac{3}{2}\,p_s\,\nabla\!\cdot\!{\bf V}_s
+ {\bf p}_s:\nabla{\bf V}_s + \nabla\!\cdot\!{\bf q}_s &=& W_s.\label{e3.46c}
\end{eqnarray}
Here, 
\begin{equation}
\frac{d}{dt} \equiv \frac{\partial}{\partial t} + {\bf V_s}\cdot \nabla
\end{equation}
is the well-known {\em convective derivative}, and 
\begin{equation}
{\bf p} : \nabla {\bf V}_s \equiv (p_s)_{\alpha\beta} \,\frac{\partial (V_s)_{\beta}}{\partial r_\alpha}.
\end{equation}
In the above, $\alpha$ and $\beta$ refer to Cartesian components, and repeated
indices are summed (according to the Einstein summation convention).
The convective derivative, of course, measures time variation  in
the local rest frame of the species-$s$ fluid.
Strictly
speaking, we should include an $s$ subscript with each convective derivative,
since this operator is clearly different for different plasma species. 

There is one additional refinement to our fluid equations which is worth
carrying out. We introduce the {\em generalized viscosity tensor}, $\bpi_s$,
by writing
\begin{equation}
{\bf p}_s = p_s\,{\bf I} + \bpi_s,
\end{equation}
where ${\bf I}$ is the unit (identity) tensor. We expect the scalar
pressure term to dominate if the plasma is relatively close to thermal 
equilibrium. We also expect, by analogy with conventional fluid theory,  the
second term to describe viscous stresses. Indeed, this is generally the case
in plasmas, 
although the generalized viscosity tensor can also include terms which are
quite unrelated to conventional viscosity. Equations~(\ref{e3.46a})--(\ref{e3.46c}) can, thus, be rewritten:
\begin{eqnarray}\label{e3.50a}
\frac{dn_s}{dt} + n_s\,\nabla\!\cdot\!{\bf V}_s &=& 0,\\[0.5ex]
m_s n_s\,\frac{d {\bf V}_s}{dt} + \nabla p_s + \nabla\!\cdot \!\bpi_s - e_s n_s
({\bf E} + {\bf V}_s\times {\bf B})& =& {\bf F}_s,\label{e3.50b}\\[0.5ex]
\frac{3}{2}\frac{d p_s}{dt} + \frac{5}{2}\,p_s\,\nabla\!\cdot\!{\bf V}_s
+ \bpi_s:\nabla{\bf V}_s + \nabla\!\cdot\!{\bf q}_s &=& W_s.\label{e3.50c}
\end{eqnarray}
According to Eq.~(\ref{e3.50a}), the species-$s$ density is constant along a fluid
trajectory unless the species-$s$ flow is non-solenoidal. For this reason,
the condition
\begin{equation}
\nabla\!\cdot\!{\bf V}_s = 0
\end{equation}
is said to describe {\em incompressible}\/ species-$s$ flow. According to 
Eq.~(\ref{e3.50b}), the species-$s$ flow accelerates 
along a fluid trajectory under the influence of the
scalar pressure gradient, the viscous stresses, the Lorentz force, and the
frictional force due to collisions with other species. Finally,
according to Eq.~(\ref{e3.50c}), the species-$s$ energy density ({\em i.e.}, $p_s$)
changes along a fluid trajectory
because of the work done in compressing the fluid, viscous heating,
heat flow, and the local energy gain due to collisions with other species.
Note that the electrical contribution to plasma heating, which was explicit
in Eq.~(\ref{e3.43}), has now become entirely implicit. 

\section{Entropy Production}
It is instructive to rewrite the species-$s$ energy evolution  equation (\ref{e3.50c}) as
an {\em entropy}\/ evolution equation. The fluid definition of entropy
density, which coincides with the thermodynamic entropy density in the 
limit in which the distribution function approaches a Maxwellian, is
\begin{equation}\label{entropy}
s_s = n_s\log\!\left(\frac{T_s^{~3/2}}{n_s}\right).
\end{equation}
The corresponding entropy flux density is written
\begin{equation}
{\bf s}_s = s_s\,{\bf V}_s + \frac{{\bf q}_s}{T_s}.
\end{equation}
Clearly, entropy is convected by the fluid flow, but is also carried by the
flow of heat, in accordance with the second law of thermodynamics. After some
algebra, Eq.~(\ref{e3.50c}) can be rearranged to give
\begin{equation}
\frac{\partial s_s}{\partial t} + \nabla\!\cdot\!{\bf s}_s = {\mit\Theta}_s,
\end{equation}
where the right-hand side is given by
\begin{equation}
{\mit\Theta}_s = \frac{W_s}{T_s} - \frac{\bpi_s : \nabla {\bf V}_s}{T_s}
- \frac{{\bf q}_s}{T_s} \cdot \frac{\nabla T_s}{T_s}.
\end{equation}
It is clear, from our previous discussion of conservation laws, that the
quantity ${\mit\Theta}_s$ can be regarded as the {\em entropy production rate}\/
per unit volume for species $s$. Note that entropy is produced by
collisional heating, viscous heating, and heat flow down temperature
gradients.

\section{Fluid Closure}\label{s3.7}
No amount of manipulation, or rearrangement, can cure our fluid equations
of their most serious defect: the fact that they are {\em incomplete}. 
In their present form, (\ref{e3.50a})--(\ref{e3.50c}), our equations
relate interesting fluid quantities, such as the density, $n_s$, the
flow velocity, ${\bf V}_s$, and the scalar pressure, $p_s$, to unknown quantities,
such as the viscosity tensor, $\bpi_s$, the heat flux
density, ${\bf q}_s$, and the
moments of the collision operator, ${\bf F}_s$ and $W_s$. In order to complete
our set of equations, we need to use some additional information to express the
latter quantities in terms of the former. This process is known as {\em closure}. 

Lack of closure is an endemic problem in fluid theory. Since each moment is
coupled to the next higher moment ({\em e.g.}, the density evolution depends on
the flow velocity, the flow velocity evolution depends on the viscosity
tensor, {\em etc}.), any finite set of exact moment equations is bound to
contain more unknowns than equations. 

There are two basic types of fluid closure schemes. In {\em truncation}\/ schemes,
higher order moments are arbitrarily 
assumed to vanish, or simply prescribed in terms of lower
moments. Truncation schemes can often provide quick insight into fluid
systems, but always involve uncontrolled approximation. {\em Asymptotic}\/ schemes
depend on the rigorous exploitation of some small parameter. 
They have the advantage of being systematic, and providing some
estimate of the error involved in the closure. On the other hand,
the asymptotic approach to closure is mathematically very demanding, since it
inevitably involves working with the kinetic equation.

The classic example of an asymptotic closure scheme is the Chapman-Enskog
theory of a neutral gas dominated by collisions. In this case, the small
parameter is the ratio of the mean-free-path between collisions to the
macroscopic variation length-scale. It is instructive to briefly examine this theory,
which is very well described in a classic monograph by Chapman and 
Cowling.\footnote{S.~Chapman, and T.G.~Cowling, {\em The Mathematical Theory
of Non-Uniform Gases} (Cambridge University Press, Cambridge  UK, 1953).}

Consider a neutral gas consisting of identical hard-sphere molecules of
mass $m$ and 
diameter $\sigma$. Admittedly, this is not a particularly 
physical model of a neutral gas,
 but we are only considering it for illustrative purposes. The fluid
equations for such a gas are similar to Eqs.~(\ref{e3.50a})--(\ref{e3.50c}):
\begin{eqnarray}\label{e3.56a}
\frac{dn}{dt} + n\,\nabla\!\cdot\!{\bf V} &=& 0,\\[0.5ex]
m n\,\frac{d {\bf V}}{dt} + \nabla p + \nabla\!\cdot \!\bpi + mn\,{\bf g}
& =& {\bf 0},\\[0.5ex]
\frac{3}{2}\frac{d p}{dt} + \frac{5}{2}\,p\,\nabla\!\cdot\!{\bf V}
+ \bpi:\nabla{\bf V} + \nabla\!\cdot\!{\bf q} &=& 0.\label{e3.56c}
\end{eqnarray}
Here, $n$ is the (particle) density, ${\bf V}$ the flow velocity, 
$p$ the scalar pressure,
and ${\bf g}$  the acceleration due to gravity.
We have dropped the subscript $s$ because, in this case, there is
only a single species. Note
 that there is no collisional friction or heating
in a single species system. 
Of course, there are no electrical or magnetic forces in
a neutral gas, so we have included gravitational forces instead. 
The purpose of the closure scheme is to express the
viscosity tensor, $\bpi$, and the heat flux density, ${\bf q}$, in terms
of $n$, ${\bf V}$, or $p$, and, thereby, complete the set of equations.

The mean-free-path $l$ for  hard-sphere molecules is given by
\begin{equation}\label{e3.57}
l = \frac{1}{\sqrt{2}\,\pi\,n\,\sigma^2}.
\end{equation}
This formula is fairly easy to understand: the volume swept out by a given molecule
in moving a mean-free-path must contain, on average, approximately one
other molecule. Note that $l$ is completely independent of the speed or
mass of the molecules. The mean-free-path is assumed to be much
smaller than the variation length-scale $L$ of macroscopic quantities,
so that
\begin{equation}
\epsilon = \frac{l}{L} \ll 1.
\end{equation}

In the Chapman-Enskog scheme, the distribution function is expanded, order by order,
in the small parameter $\epsilon$:
\begin{equation}
f({\bf r}, {\bf v}, t) = f_0({\bf r}, {\bf v}, t) + \epsilon\,f_1({\bf r}, {\bf v}, t) + \epsilon^2\, f_2({\bf r}, {\bf v}, t) + \cdots.
\end{equation}
Here, $f_0$, $f_1$, $f_2$, {\em etc.}, are all assumed to be of the same order of
magnitude. In fact, only the first {\em two}\/ terms in this expansion are ever
calculated. 
To zeroth order in $\epsilon$, the kinetic equation requires that $f_0$ be
a Maxwellian:
\begin{equation}
f_0({\bf r}, {\bf v}, t) = n({\bf r})
\left(\frac{m}{2\pi\,T({\bf r})}\right)^{3/2}\,\exp\!\left[-\frac{m\,({\bf v}-
{\bf V})^2}
{2\,T({\bf r})}\right].
\end{equation}
Recall that $p=n\,T$. Note that there is zero heat flow or viscous stress associated
with a Maxwellian distribution function. Thus, both the heat flux density, 
${\bf q}$,
and the viscosity tensor, $\bpi$, depend on the first-order 
non-Maxwellian correction
to the distribution function, $f_1$. 

It is possible to {\em linearize}\/ the kinetic equation, and then rearrange
it so as to obtain an {\em integral equation}\/ for $f_1$ in terms of $f_0$.
This rearrangement depends crucially on the {\em bilinearity}\/ of the collision
operator.
Incidentally, the equation is integral because the collision operator is an integral
operator. The integral equation is solved by expanding $f_1$ in velocity space
using Laguerre polynomials (sometime called Sonine polynomials). It is
possible to reduce the integral equation to an infinite  set of simultaneous
algebraic equations for the coefficients in this expansion. If the expansion
is truncated, after $N$ terms, say, then these algebraic equations can be solved for
the coefficients. It turns out that the Laguerre polynomial expansion
converges very rapidly. Thus, it is conventional to only keep the first {\em two}\/
terms in this expansion, which is usually sufficient to ensure an accuracy of
about $1\%$ in the final result. Finally, the appropriate moments
of $f_1$ are taken, so as to obtain expression for the heat flux density
and the viscosity
tensor. Strictly speaking, after evaluating $f_1$, we should then go on to
evaluate $f_2$, so as to ensure that $f_2$ really is negligible compared to $f_1$.
In reality, this is never done because the mathematical difficulties involved
in such a calculation are prohibitive.

The Chapman-Enskog method outlined above can be applied to {\em any}\/ assumed
force law between molecules, provided that the force is sufficiently short-range
({\em i.e.}, provided that it falls off faster with increasing 
separation than the Coulomb force). For all sensible force laws, the viscosity
tensor is given by
\begin{equation}\label{e3.61}
\pi_{\alpha\beta} =- \eta \left( \frac{\partial V_\alpha}{\partial r_\beta}
+ \frac{\partial V_\beta}{\partial r_\alpha} - \frac{2}{3}\,\nabla\!\cdot\!{\bf V}\,\delta_{\alpha\beta}
\right),
\end{equation}
whereas the heat flux density takes the form
\begin{equation}\label{e3.62}
{\bf q} = - \kappa\,\nabla T.
\end{equation}
Here, $\eta$ is the {\em coefficient of viscosity}, and $\kappa$ is the
{\em coefficient of thermal conduction}. It is convenient to write
\begin{eqnarray}
\eta &= &m n \,{\mit\chi}_v,\\[0.5ex]
\kappa& =& n\,{\mit\chi}_t,
\end{eqnarray}
where ${\mit\chi}_v$ is the {\em viscous diffusivity}\/ and ${\mit\chi}_t$
is the {\em thermal diffusivity}. Note that both ${\mit\chi}_v$ and
${\mit\chi}_t$ have the dimensions ${\rm m}^2\,{\rm s}^{-1}$ and are,
effectively, {\em diffusion coefficients}. For the special
case of hard-sphere molecules, Chapman-Enskog theory yields:
\begin{eqnarray}\label{e3.64a}
{\mit\chi}_v &=& \frac{75\,\pi^{1/2}}{64}\,\left[1+ \frac{3}{202}
+\cdots\right]\,\nu\,l^2 = A_v\,\nu\,l^2,\\[0.5ex]
{\mit\chi}_t &=& \frac{5\,\pi^{1/2}}{16}\,\left[
1+ \frac{1}{44}+\cdots\right]\,\nu\,l^2 = A_t\,\nu\,l^2.\label{e3.64b}
\end{eqnarray}
Here, 
\begin{equation}\label{e3.65}
\nu \equiv \frac{v_t}{l} \equiv \frac{\sqrt{2\,T/m}}{l}
\end{equation}
is the {\em collision frequency}. Note that the first two terms in the Laguerre
polynomial expansion are shown explicitly (in the square brackets) in Eqs.~(\ref{e3.64a})--(\ref{e3.64b}).

Equations (\ref{e3.64a})--(\ref{e3.64b}) have a simple physical interpretation: the viscous and thermal
diffusivities of a neutral gas can be accounted for in terms of the {\em
random-walk diffusion}\/ of molecules with excess momentum and energy, respectively.
Recall the standard result in stochastic theory that if particles
jump an average distance $l$, in a random direction, $\nu$ times a second, then
the diffusivity associated with such motion is $\chi\sim\nu\,l^2$.
Chapman-Enskog theory basically allows us to calculate the numerical constants
$A_v$ and $A_t$,
multiplying $\nu\,l^2$ in the expressions for $\chi_v$ and $\chi_t$,
 for a given force law between molecules.
Obviously, these coefficients are different for different force laws. The
expression for the 
mean-free-path, $l$,  is also different for different force laws. 

Let $\bar{n}$, $\bar{v}_t$, and $\bar{l}$ be typical values of the
particle density, the thermal velocity, and the mean-free-path, respectively.
Suppose that the typical flow velocity is $\lambda\,\bar{v}_t$,
and the typical variation length-scale is $L$. Let us
define the following normalized quantities: $\hat{n}=n/\bar{n}$, $\hat{v}_t=
v_t/\bar{v}_t$, $\hat{l} = l/\bar{l}$, $\hat{{\bf r}}= {\bf r}/ L$,
$\hat{\nabla} = L\,\nabla$, $\hat{t} = \lambda\,\bar{v}_t\,t/L$, 
$\hat{\bf V} = {\bf V}/\lambda\,\bar{v}_t$, 
$\hat{T} = T/m\,\bar{v}_t^{~2}$, 
$\hat{\bf g} = L\,{\bf g}/ (1+\lambda^2)\,\bar{v}_t^{~2}$,
$\hat{p}= p/m\,\bar{n}\,\bar{v}_t^{~2}$, $\hat{\bpi} =
\bpi/ \lambda\,\epsilon\,m\,\bar{n}\,\bar{v}_t^{~2}$, 
$\hat{\bf q}= {\bf q} / \epsilon\,m\,\bar{n}\,\bar{v}_t^{~3}$. Here,
$\epsilon = \bar{l}/L\ll 1$. 
Note that
\begin{eqnarray}
\hat{\bpi} &=& - A_v\,\hat{n}\,\hat{v}_t\,\hat{l}\left(
\frac{\partial\hat{V}_\alpha}{\partial\hat{r}_\beta}+
\frac{\partial\hat{V}_\beta    }{\partial\hat{r}_\alpha}-\frac{2}{3}\,\hat{\nabla}\!
\cdot\!\hat{\bf V}\,\delta_{\alpha\beta}\right),\\[0.5ex]
\hat{\bf q} &=& - A_t\,\hat{n}\,\hat{v}_t\,\hat{l}\,\,\hat{\nabla}\hat{T}.
\end{eqnarray}
All hatted quantities are designed to be
 $O(1)$.
The normalized fluid equations are written:
\begin{eqnarray}\label{e3.67a}
\frac{d\hat{n}}{d\hat{t}} + \hat{n}\,\hat{\nabla}\!\cdot\!\hat{\bf V} &=& 0,\\[0.5ex]
\lambda^2\,\hat{n}\,\frac{d \hat{\bf V}}{d\hat{t}} + \hat{\nabla} \hat{p}
 + \lambda\,\epsilon\,\hat{\nabla}\!\cdot \!\hat{\bpi} + (1+\lambda^2)
\,\hat{n}\,\hat{\bf g}
& =& {\bf 0},\\[0.5ex]
\lambda\,\frac{3}{2}\frac{d \hat{p}}{d\hat{t}} + 
\lambda\,\frac{5}{2}\,\hat{p}\,\hat{\nabla}\!\cdot\!
\hat{\bf V}
+ \lambda^2\,\epsilon\,\hat{\bpi}:\hat{\nabla}\hat{\bf V} + 
\epsilon\,\hat{\nabla}\!\cdot\!\hat{\bf q} &=& 0,\label{e3.67c}
\end{eqnarray}
where
\begin{equation}
\frac{d}{d\hat{t}}\equiv \frac{\partial}{\partial \hat{t}} + 
\hat{\bf V}\!\cdot\!\hat{\nabla}.
\end{equation}
Note that the only large or small quantities in the above equations are
the parameters $\lambda$ and $\epsilon$. 

Suppose that $\lambda\gg 1$. In other words, the flow velocity is much
greater than the thermal speed. Retaining only the largest terms in Eqs.~(\ref{e3.67a})--(\ref{e3.67c}),
our system of fluid equations reduces to (in unnormalized form):
\begin{eqnarray}
\frac{dn}{dt} + n\,\nabla\!\cdot\!{\bf V} &=& 0,\\[0.5ex]
\frac{d{\bf V}}{dt} + {\bf g} &\simeq& {\bf 0}.
\end{eqnarray}
These are called the {\em cold-gas}\/ equations, because they can also
be obtained by formally taking the limit $T\rightarrow 0$. The cold-gas
 equations describe
externally driven, highly supersonic, gas dynamics.
Note that  the gas pressure ({\em i.e.}, energy density) 
can be neglected in the cold-gas limit, since the thermal velocity is much
smaller than the flow velocity, and so there is no need for an energy evolution equation. Furthermore, the 
viscosity can also be neglected, since the
viscous diffusion velocity  is also far smaller than the
flow velocity.

Suppose that $\lambda\sim O(1)$. In other words, the flow velocity is of order the
thermal speed. Again, retaining only the largest terms in Eqs.~(\ref{e3.67a})--(\ref{e3.67c}),
our system of fluid equations reduces to (in unnormalized form):
\begin{eqnarray}
\frac{dn}{dt} + n\,\nabla\!\cdot\!{\bf V} &=& 0,\\[0.5ex]
m n\,\frac{d {\bf V}}{dt} + \nabla p + mn\,{\bf g}
& \simeq& {\bf 0},\\[0.5ex]
\frac{3}{2}\frac{d p}{dt} + \frac{5}{2}\,p\,\nabla\!\cdot\!{\bf V}
 &\simeq& 0.
\end{eqnarray}
The above  equations can be rearranged to give:
\begin{eqnarray}
\frac{dn}{dt} + n\,\nabla\!\cdot\!{\bf V} &=& 0,\\[0.5ex]
m n\,\frac{d {\bf V}}{dt} + \nabla p + mn\,{\bf g}
& \simeq& {\bf 0},\\[0.5ex]
\frac{d}{dt}\!\left(\frac{p}{n^{5/3}}\right) 
 &\simeq& 0.\label{e3.71c}
\end{eqnarray}
These are called the {\em hydrodynamic}\/ equations, since they are similar to the
equations governing the dynamics of water. The hydrodynamic equations
govern relatively fast, internally driven, 
gas dynamics: in particular, the dynamics of {\em sound waves}.
Note that the gas pressure is non-negligible in the
hydrodynamic limit,  since the
thermal velocity is of  order  the flow speed, and so
 an energy evolution equation is needed. However, the
energy equation takes a particularly simple form, because
Eq.~(\ref{e3.71c}) is immediately recognizable as the {\em adiabatic}\/ equation
of state for a monatomic gas. This is not surprising, since the
flow velocity is still much faster than the viscous and thermal diffusion
velocities (hence, the absence of viscosity and thermal conductivity in the
hydrodynamic equations), in which case the gas acts effectively like a perfect
thermal insulator. 

Suppose, finally, that $\lambda\sim \epsilon$. In other words, the flow
velocity is of order the viscous and thermal diffusion velocities. Our system
of fluid equations now reduces to a force balance criterion,
\begin{equation}
\nabla p + mn\,{\bf g} \simeq {\bf 0},
\end{equation}
to lowest order. To next order, we obtain a set of equations
describing the relatively slow viscous and thermal evolution 
of the gas:
\begin{eqnarray}
\frac{dn}{dt} + n\,\nabla\!\cdot\!{\bf V} &=& 0,\\[0.5ex]
m n\,\frac{d {\bf V}}{dt} + \nabla\!\cdot \!\bpi 
& \simeq& {\bf 0},\\[0.5ex]
\frac{3}{2}\frac{d p}{dt} + \frac{5}{2}\,p\,\nabla\!\cdot\!{\bf V}
+ \nabla\!\cdot\!{\bf q} &\simeq& 0.
\end{eqnarray}
Clearly, this set of equations is only appropriate to relatively quiescent,
quasi-equilibrium, gas dynamics. Note that virtually all of the terms
in our original fluid  equations, (\ref{e3.56a})--(\ref{e3.56c}), must be retained in this limit.

The above investigation reveals an important truth in gas dynamics, which also
applies to plasma dynamics. Namely, the form of the
fluid equations depends crucially on the typical fluid {\em velocity}\/ 
associated with the type of dynamics under investigation. As a general rule,
the equations get simpler as the typical velocity  get faster, and {\em vice versa}. 

\section{Braginskii Equations}
Let now consider the problem of closure in {\em plasma}\/ fluid equations. There are, in
fact, two possible small parameters in plasmas upon which we could
 base an asymptotic
closure scheme. The first is the ratio of the mean-free-path, $l$, to the
macroscopic length-scale, $L$. This is only appropriate to {\em collisional}\/ 
plasmas. The second is the ratio of the Larmor radius, $\rho$, to the macroscopic
length-scale, $L$. This is only appropriate to {\em magnetized} plasmas. 
There is, of course, no small parameter upon which to base an asymptotic
closure scheme in a collisionless, unmagnetized plasma. However, such systems
occur predominately in accelerator physics contexts, and are not really 
``plasmas'' at all,
since they exhibit virtually no collective effects. Let us
 investigate 
Chapman-Enskog-like closure schemes
in a {\em collisional}, quasi-neutral plasma consisting of
equal numbers of electrons and ions. We shall treat the 
unmagnetized and magnetized cases
separately. 

The first step in our closure scheme is to approximate the actual collision
operator for Coulomb interactions by an operator which is strictly
{\em bilinear}\/ in its arguments (see Sect.~\ref{s3.3}). Once this has been achieved,
the closure problem is formally of the type which can be solved using the
Chapman-Enskog method. 

The electrons and ions collision times, $\tau = l/v_t = \nu^{-1}$, are
written
\begin{equation}\label{e3.74}
\tau_e = \frac{6\sqrt{2}\,\pi^{3/2}\,\epsilon_0^{~2}\,\sqrt{m_e}\,\,T_e^{~3/2}}
{\ln\Lambda\, e^4\, n},
\end{equation}
and
\begin{equation}\label{e3.75}
\tau_i = \frac{ 12\,\pi^{3/2}\,\epsilon_0^{~2}\,\sqrt{m_i}\,\,T_i^{~3/2}}
{\ln\Lambda\, e^4\, n},
\end{equation}
respectively. 
Here, $n=n_e=n_i$ is the number density of particles, and $\ln\Lambda$ is a quantity
called the {\em Coulomb logarithm}\/ whose origin is  the slight modification to the
collision operator  mentioned above. The Coulomb logarithm is equal to
the natural logarithm of the ratio of the maximum to minimum impact parameters
for Coulomb ``collisions.'' In other words, $\ln\Lambda=\ln\,(d_{\rm max}/d_{\rm min})$.
The minimum parameter is simply the distance of closest approach,
$d_{\rm min} \simeq r_c= e^2/4\pi\epsilon_0\,T_e$ [see Eq.~(\ref{e1.17})]. The
maximum parameter is the Debye length, $d_{\rm max} \simeq 
\lambda_D=\sqrt{\epsilon_0\,T_e/
n\,e^2}$, since the Coulomb potential is shielded over distances greater than
the Debye length. The Coulomb logarithm is a {\em very slowly varying}\/  function
of the plasma density and the electron temperature, and is well approximated by
\begin{equation}
\ln\Lambda \simeq 6.6 - 0.5 \, ln \,n + 1.5\,\ln T_e,
\end{equation}
where $n$ is expressed in units of $10^{20}\,{\rm m}^{-3}$, and $T_e$ is expressed in electron volts.

The basic forms of Eqs.~(\ref{e3.74}) and (\ref{e3.75}) are not hard to understand.
From Eq.~(\ref{e3.57}), we expect
\begin{equation}
\tau \sim \frac{l}{v_t} \sim \frac{1}{n\,\sigma^2\,v_t},
\end{equation}
where $\sigma^2$ is the typical ``cross-section'' of the electrons or ions for
Coulomb ``collisions.'' Of course, this cross-section is simply the square of
the distance
of closest approach, $r_c$, defined in Eq.~(\ref{e1.17}). Thus,
\begin{equation}
\tau \sim \frac{1}{n\,r_c^{~2}\,v_t} \sim \frac{\epsilon_0^{~2}\sqrt{m}\,\,T^{3/2}}
{e^4\,n}.
\end{equation}
The most significant feature of Eqs.~(\ref{e3.74}) and (\ref{e3.75}) is the strong variation of
the collision times with {\em temperature}. As the plasma gets hotter, the
distance of closest approach gets smaller, so that both electrons and
ions offer  much smaller cross-sections for Coulomb collisions. The net
result is that such collisions become far  less frequent, and the collision
times ({\em i.e.}, the mean times between $90^\circ$ degree scattering
events) get much  longer. It follows that as plasmas are heated they become
less collisional very rapidly. 

The electron and ion fluid equations in a collisional plasma
take the form [see Eqs.~(\ref{e3.50a})--(\ref{e3.50c})]:
\begin{eqnarray}\label{e3.79a}
\frac{dn}{dt} + n\,\nabla\!\cdot\!{\bf V}_e &=& 0,\\[0.5ex]
m_e n\,\frac{d {\bf V}_e}{dt} + \nabla p_e+ \nabla\!\cdot \!\bpi_e + e n\,
({\bf E} + {\bf V}_e\times {\bf B})& =& {\bf F},\\[0.5ex]
\frac{3}{2}\frac{d p_e}{dt} + \frac{5}{2}\,p_e\,\nabla\!\cdot\!{\bf V}_e
+ \bpi_e:\nabla{\bf V}_e+ \nabla\!\cdot\!{\bf q}_e &=& W_e,\label{e3.79c}
\end{eqnarray}
and
\begin{eqnarray}\label{e3.80a}
\frac{dn}{dt} + n\,\nabla\!\cdot\!{\bf V}_i &=& 0,\\[0.5ex]
m_i n\,\frac{d {\bf V}_i}{dt} + \nabla p_i + \nabla\!\cdot \!\bpi_i - e n\,
({\bf E} + {\bf V}_i\times {\bf B})& =&- {\bf F},\\[0.5ex]
\frac{3}{2}\frac{d p_i}{dt} + \frac{5}{2}\,p_i\,\nabla\!\cdot\!{\bf V}_i
+ \bpi_i:\nabla{\bf V}_i+ \nabla\!\cdot\!{\bf q}_i &=& W_i,\label{e3.80c}
\end{eqnarray}
respectively. Here, use has been made of the momentum conservation law (\ref{e3.30}). 
Equations (\ref{e3.79a})--(\ref{e3.79c}) and (\ref{e3.80a})--(\ref{e3.80c})  are called the {\em Braginskii equations}, since they were first obtained
in a celebrated article by S.I.~Braginskii.\footnote{S.I.~Braginskii,
{\em Transport Processes in a Plasma}, in {\sf Reviews of Plasma Physics}
(Consultants Bureau, New York NY, 1965), Vol.~1, p.~205.}

In the {\em unmagnetized}\/ limit, which actually
corresponds to
\begin{equation}
{\mit\Omega}_i\,\tau_i, \,\,\,{\mit\Omega}_e\,\tau_e \ll 1,
\end{equation}
the standard two-Laguerre-polynomial Chapman-Enskog closure scheme yields
\begin{eqnarray}\label{e3.82a}
{\bf F} &=& \frac{ne\,{\bf j}}{\sigma_\parallel} - 0.71\,n\,\nabla T_e,\\[0.5ex]\label{e3.82b}
W_i &=& \frac{3\,m_e}{m_i} \frac{n\,(T_e-T_i)}{\tau_e},\\[0.5ex]
W_e &=& -W_i + \frac{ {\bf j}\cdot {\bf F} }{n \,e}= -W_i + 
\frac{j^2}{\sigma_\parallel} -
0.71\,\frac{{\bf j}\cdot \nabla T_e}{e}.\label{e3.82c}
\end{eqnarray}
Here, ${\bf j} = - n\,e\,({\bf V}_e-{\bf V}_i)$ is the net plasma current,
and the {\em electrical conductivity}\/ $\sigma_\parallel$ is given by
\begin{equation}\label{e3.83}
\sigma_\parallel = 1.96\,\frac{n \,e^2\,\tau_e}{m_e}.
\end{equation} In the above, use has been made of the conservation law
(\ref{e3.37}).

Let us examine each of the above collisional terms, one by one. The first
term on the right-hand side of Eq.~(\ref{e3.82a}) is a friction force due to the
relative motion of electrons and ions, and obviously controls the electrical
conductivity of the plasma. The form of this term is fairly easy to understand.
The electrons lose their ordered velocity
with respect to the ions,  ${\bf U} = {\bf V}_e - {\bf V}_i$,
in an electron collision time, $\tau_e$, and consequently lose momentum 
$m_e\,{\bf U}$  per electron (which is given to the ions) in this time. 
This means that a frictional force $(m_e\,n/\tau_e)\,{\bf U}
\sim n\,e\,{\bf j}/(n\,e^2\,\tau_e/m_e)$ is exerted on the electrons.
An equal and opposite force is exerted on the ions. Note that, since the Coulomb
cross-section diminishes with increasing electron energy ({\em i.e.}, 
$\tau_e\sim T_e^{~3/2}$),
the conductivity of the fast electrons in the distribution function
is higher than that of the slow electrons (since, $\sigma_\parallel \sim \tau_e$).
Hence, electrical current in plasmas is carried predominately by the
{\em fast}\/  electrons. This effect has some important and interesting
consequences. 

One immediate consequence is the second term on the right-hand side of Eq.~(\ref{e3.82a}),
which is called the {\em thermal force}. To understand the origin of
a frictional force proportional to minus the gradient of the electron temperature,
let us assume that the electron and ion fluids are at rest ({\em i.e.}, 
$V_e=V_i =0$). It follows that the number of electrons moving from left to right
(along the $x$-axis, say) and from right to left per unit time is exactly the
same at a given point (coordinate $x_0$, say) in the plasma. As a result
of electron-ion collisions, these fluxes experience frictional forces,
${\bf F}_-$ and ${\bf F}_+$, respectively, of order $m_e\,n\,v_e/\tau_e$,
where $v_e$ is the electron thermal velocity. In a completely homogeneous
plasma these forces balance exactly, and so there is zero net frictional force.
Suppose, however, that the electrons coming from the right are, on average, hotter
than those coming from the left. It follows that the frictional force
${\bf F}_+$ acting on the fast electrons coming from the right is less than
the force ${\bf F}_-$ acting on the slow electrons coming from the left, since
$\tau_e$ increases with electron temperature. As a result, there is a net 
frictional force acting to the left: {\em i.e.}, in the direction of $-\nabla T_e$. 

Let us estimate the magnitude of the frictional force. At point $x_0$, collisions
are experienced by electrons which have traversed distances of order a
mean-free-path, $l_e\sim v_e\,\tau_e$. Thus, the electrons coming from the
right originate from regions in which the temperature is approximately
$l_e\,\partial T_e/\partial x$ greater than the regions from which the electrons
coming from the left originate. Since the friction force is proportional to 
$T_e^{~-1}$, the net force ${\bf F}_+ - {\bf F}_-$ is of order
\begin{equation}\label{e3.84}
{\bf F}_T \sim- \frac{l_e}{T_e} \frac{\partial T_e}{\partial x}
\frac{m_e\,n\,v_e}{\tau_e} \sim- \frac{m_e\,v_e^{~2}}{T_e}\, n\,\frac{\partial T_e}
{\partial x} \sim - n\,\frac{\partial T_e}
{\partial x}.
\end{equation}
It must be emphasized that the thermal force is a direct consequence of {\em 
collisions}, despite the fact that the expression for
the thermal force does not contain
$\tau_e$ explicitly. 

The term $W_i$, specified by Eq.~(\ref{e3.82b}),  represents the rate at which
energy is acquired by the ions due to collisions
with the electrons. 
The most striking aspect of this term is
its {\em smallness}\/
 (note that it is proportional to an inverse mass ratio,
$m_e/m_i$). The smallness of $W_i$ is a direct consequence of the
fact that electrons are considerably lighter than ions. Consider the
limit in which the ion mass is infinite, and the ions are at rest on average:
{\em i.e.}, $V_i=0$. In this case, collisions of electrons with ions
take place {\em without any}\/ exchange of energy. The electron velocities
are randomized by the collisions, so that the energy associated
with their ordered velocity, ${\bf U} = {\bf V}_e-{\bf V}_i$, is converted
into heat energy in the electron fluid [this is represented by the second term
on the extreme right-hand side of Eq.~(\ref{e3.82c})]. However, the ion energy remains
unchanged. Let us now assume that the ratio $m_i/m_e$ is large, but finite, and
that $U=0$. If $T_e=T_i$, the ions and electrons are in thermal equilibrium, so
no heat is exchanged between them. However, if $T_e>T_i$, heat
is transferred from the electrons to the ions. As is well known, when
a light particle collides with a heavy particle, the order of magnitude of the
transferred energy is given by the mass ratio $m_1/m_2$, where $m_1$ is the
mass of the lighter particle. For example, the mean fractional energy transferred
in isotropic scattering is $2m_1/m_2$. Thus, we would expect the
energy per unit time transferred from  the electrons to the ions to be roughly
\begin{equation}
W_i \sim \frac{n}{\tau_e} \,\frac{2m_e}{m_i}\,\frac{3}{2} \,(T_e-T_i).
\end{equation}
In fact, $\tau_e$ is defined so as to make the above estimate exact.

The term $W_e$, specified by Eq.~(\ref{e3.82c}), represents the rate at
which energy is acquired by the electrons due to
collisions with the ions, and consists of three terms. Not surprisingly,
the first term is simply minus the rate at which energy is
acquired by the ions due to collisions with the
electrons. The second term represents the conversion
of the ordered motion of the electrons, relative to the ions, into random
motion ({\em i.e.}, heat) via collisions with the ions. Note that this
term is positive definite, indicating that the randomization of the electron
ordered motion gives rise to {\em irreversible}\/ heat generation.
Incidentally,  this
term is usually called the {\em ohmic heating}\/ term. Finally, the third
term represents the work done against the thermal force. Note that this
term can be either positive or negative, depending on the direction of
the current flow relative to the electron temperature gradient. This
indicates that work done against the thermal force gives rise to {\em reversible}\/ 
heat generation. There is an analogous effect in metals called the {\em
Thomson effect}. 

The electron and ion heat flux densities are given by
\begin{eqnarray}\label{e3.86a}
{\bf q}_e &=& -\kappa_\parallel^e\,\nabla T_e - 
0.71\,\frac{T_e\,{\bf j}}{e},\\[0.5ex]
{\bf q}_i &=& -\kappa_\parallel^i\,\nabla T_i,\label{e3.86b}
\end{eqnarray}
respectively. The electron and ion {\em thermal conductivities}\/ are written
\begin{eqnarray}\label{e3.87a}
\kappa_\parallel^e &=& 3.2\,\,\frac{n\,\tau_e\,T_e}{m_e},\\[0.5ex]
\kappa_\parallel^i &=& 3.9\,\,\frac{n\,\tau_i\,T_i}{m_i},\label{e3.87b}
\end{eqnarray}
respectively. 

It follows, by comparison with Eqs.~(\ref{e3.62})--(\ref{e3.65}), that 
the first term on the right-hand side of Eq.~(\ref{e3.86a}) and the expression
on the right-hand side of Eq. (\ref{e3.86b}) represent straightforward
random-walk heat diffusion, with frequency $\nu$, and step-length $l$.
Recall, that $\nu=\tau^{-1}$ is the collision frequency, and
$l=\tau\,v_t$ is the mean-free-path. Note  that the
electron  heat diffusivity is generally much greater than that of the ions,
since $\kappa_\parallel^e/\kappa_\parallel^i\sim \sqrt{m_i/m_e}$,
assuming that $T_e\sim T_i$. 

The second term on the right-hand side of Eq.~(\ref{e3.86a}) describes  a convective
heat flux due to the motion of the electrons relative to the ions. 
To understand the origin of this flux, we need to recall that
electric current in plasmas is carried predominately by the fast electrons
in the distribution function. Suppose that $U$ is non-zero. In the
coordinate system in which $V_e$ is zero, more fast electron move in the
direction of ${\bf U}$, and more slow electrons move in the opposite
direction. Although the electron fluxes are balanced in this frame of reference,
the energy fluxes are not (since a fast electron possesses more energy than a slow
electron), and heat flows in the direction of ${\bf U}$: {\em i.e.}, in
the opposite direction to the electric current. The net heat flux density is of
order $n\,T_e\,U$: {\em i.e.}, there is no near cancellation of the fluxes
due to the fast and slow electrons. Like the thermal force, this effect
depends on collisions despite the fact that the expression for the convective
heat flux does not contain $\tau_e$ explicitly. 

Finally, the electron and ion viscosity tensors take the form
\begin{eqnarray}\label{e3.88a}
(\pi_e)_{\alpha\beta}& =& - \eta_0^e\, \left( \frac{\partial V_\alpha}{\partial r_\beta}
+ \frac{\partial V_\beta}{\partial r_\alpha} - \frac{2}{3}\,\nabla\!\cdot\!{\bf V}\,\delta_{\alpha\beta}\right),\\[0.5ex]
(\pi_i)_{\alpha\beta}& =& - \eta_0^i\, \left( \frac{\partial V_\alpha}{\partial r_\beta}
+ \frac{\partial V_\beta}{\partial r_\alpha} - \frac{2}{3}\,\nabla\!\cdot\!{\bf V}\,\delta_{\alpha\beta}
\right),\label{e3.88b}
\end{eqnarray}
respectively. Obviously, $V_\alpha$ refers to a Cartesian component of the
electron fluid velocity in Eq.~(\ref{e3.88a}) and the ion fluid velocity in Eq.~(\ref{e3.88b}).
Here, the electron and ion {\em viscosities}\/ are given
by
\begin{eqnarray}\label{e3.89a}
\eta_0^e &=& 0.73\,n\,\tau_e\,T_e,\\[0.5ex]
\eta_0^i &=& 0.96\,n\,\tau_i\,T_i,\label{e3.89b}
\end{eqnarray}
respectively. 
It follows, by comparison with Eqs.~(\ref{e3.61})--(\ref{e3.65}), that the above expressions
correspond to straightforward random-walk diffusion of momentum, with 
frequency $\nu$, and step-length $l$. Again, the electron diffusivity
exceeds the ion diffusivity by the square root of a mass ratio (assuming
$T_e\sim T_i$).  However,
the ion viscosity exceeds the electron viscosity by the same factor (recall
that $\eta\sim nm\,\chi_v$): {\em i.e.}, $\eta_0^i/\eta_0^e\sim\sqrt{m_i/m_e}$.
For this reason, the viscosity of a plasma is determined essentially by the
ions. This is not surprising, since viscosity is the diffusion of momentum,
and the ions possess nearly all of the momentum in a plasma by virtue of
their large masses.


Let us now examine the {\em magnetized}\/ limit,
\begin{equation}
{\mit\Omega}_i\,\tau_i, \,\,\,{\mit\Omega}_e\, \tau_e \gg 1,
\end{equation}
in which the electron and ion gyroradii are much {\em smaller}\/ than the
corresponding mean-free-paths. In this limit, the two-Laguerre-polynomial
Chapman-Enskog closure scheme yields
\begin{eqnarray}\label{e3.91a}
{\bf F}&=& ne\left(\frac{{\bf j}_\parallel}{\sigma_\parallel}
+\frac{{\bf j}_\perp}{\sigma_\perp}\right) -0.71\,n\,\nabla_\parallel T_e
-\frac{3\,n}{2\,|{\mit\Omega}_e|\,\tau_e}\,{\bf b}\times\nabla_\perp T_e,
\\[0.5ex]
W_i &=& \frac{3\,m_e}{m_i} \frac{n\,(T_e-T_i)}{\tau_e},\\[0.5ex]
W_e &=& -W_i + \frac{ {\bf j}\cdot {\bf F} }{n \,e}.
\end{eqnarray}
Here, the {\em parallel electrical conductivity}, $\sigma_\parallel$, is given by Eq.~(\ref{e3.83}),
whereas the {\em perpendicular electrical conductivity}, $\sigma_\perp$,  takes the form
\begin{equation}\label{e3.92}
\sigma_\perp = 0.51\,\sigma_\parallel = \frac{n\,e^2\,\tau_e}{m_e}.
\end{equation}
Note that $\nabla_\parallel\cdots \equiv {\bf b}\,({\bf b}\!\cdot\!\nabla
\cdots)$ denotes a
gradient parallel to the magnetic field, whereas $\nabla_\perp \equiv
 \nabla-\nabla_\parallel$ denotes a gradient perpendicular to the magnetic
field. Likewise, ${\bf j}_\parallel \equiv {\bf b}\,({\bf b}\!\cdot{\bf j})$
represents the component of the plasma current flowing parallel to the
magnetic field, whereas ${\bf j}_\perp \equiv {\bf j} - {\bf j}_\parallel$
represents the perpendicular component of the plasma current.

We expect the presence of a strong magnetic field to give rise to a 
marked {\em anisotropy}\/ in  plasma properties between directions parallel
and perpendicular to ${\bf B}$, because of the completely different motions
of the constituent ions and electrons parallel and perpendicular to the field. 
Thus, not surprisingly, we find that the electrical conductivity perpendicular
to the field is approximately half that parallel to the field [see Eqs.~(\ref{e3.91a})
and (\ref{e3.92})]. The thermal force is unchanged (relative to the unmagnetized case)
in the parallel direction, but is radically modified in the
perpendicular direction. In order to understand the origin
of the last term in Eq.~(\ref{e3.91a}), let us consider a situation in
which there is a strong magnetic field along the $z$-axis, and an electron
temperature gradient along the $x$-axis---see Fig.~\ref{f7}. The electrons gyrate
in the $x$-$y$ plane in circles of radius $\rho_e\sim v_e/|{\mit\Omega}_e|$.
At a given point, coordinate $x_0$, say, on the $x$-axis, the electrons that
come from the right and the left have traversed distances of order $\rho_e$. 
Thus, the electrons from the right originate from regions where the
electron temperature is of order $\rho_e\,\partial T_e/\partial x$ greater than
the regions from which the electrons from the left originate. Since the
friction force is proportional to $T_e^{-1}$, an unbalanced friction force
arises, directed along the $-y$-axis---see Fig.~\ref{f7}. This direction
corresponds to the direction of $-{\bf b}\times\nabla T_e$. 
Note that there is
no friction force along the $x$-axis, since the $x$-directed fluxes are due
to electrons which originate from regions where $x=x_0$. 
By analogy with Eq.~(\ref{e3.84}), the magnitude of the perpendicular
thermal force is
\begin{equation}
{\bf F}_{T\perp} \sim \frac{{\rho}_e}{T_e}\frac{\partial T_e}{\partial x}
\frac{m_e\,n\,v_e}{\tau_e} \sim \frac{n}{|{\mit\Omega}_e|\,\tau_e}
\frac{\partial T_e}{\partial x}.
\end{equation}
Note that the effect of a strong magnetic field on the perpendicular
component of the thermal force is directly analogous to a well-known
phenomenon in metals, called the {\em Nernst effect}. 

\begin{figure}
\epsfysize=3in
\centerline{\epsffile{Chapter03/brag1.eps}}
\caption{\em Origin of the perpendicular thermal force in a magnetized plasma.}\label{f7}
\end{figure}

In the magnetized limit, the electron and ion heat flux densities become
\begin{eqnarray}
{\bf q}_e &=& -\kappa_\parallel^e\,\nabla_\parallel T_e -\kappa_\perp^e\,
\nabla_\perp T_e
-\kappa_\times^e\,{\bf b}\times\nabla_\perp T_e\nonumber\\[0.5ex]
&&- 0.71\,\frac{T_e\,{\bf j}_\parallel}{e}-
\frac{3\,T_e}{2\,|{\mit\Omega}_e|\,\tau_e\,e}\,{\bf b}\times{\bf j}_\perp,\label{e3.94a}\\[0.5ex]
{\bf q}_i &=& -\kappa_\parallel^i\,\nabla_\parallel T_i -\kappa_\perp^i\,
\nabla_\perp T_i
+\kappa_\times^i\,{\bf b}\times\nabla_\perp T_i,\label{e3.94b}
\end{eqnarray}
respectively. Here, the {\em parallel thermal conductivities}\/ are 
given by Eqs.~(\ref{e3.87a})--(\ref{e3.87b}), and
the {\em perpendicular thermal conductivities}\/ take the form
\begin{eqnarray}
\kappa_\perp^e &=& 4.7\,\frac{n\,T_e}{m_e\,{\mit\Omega}_e^{~2}\,\tau_e},\\[0.5ex]
\kappa_\perp^i &=& 2\, \frac{n\,T_i}{m_i\,{\mit\Omega}_i^{~2}\,\tau_i}.
\end{eqnarray}
Finally, the {\em cross thermal conductivities}\/ are written
\begin{eqnarray}
\kappa_\times^e &=& \frac{5\,n\,T_e}{2\,m_e\,|{\mit\Omega}_e|},\\[0.5ex]
\kappa_\times^i &=&\frac{5\,n\,T_i}{2\,m_i\,{\mit\Omega}_i}.
\end{eqnarray}

The first two terms on the right-hand sides of Eqs.~(\ref{e3.94a}) and (\ref{e3.94b})
correspond to {\em diffusive}\/ heat transport by the electron and ion
fluids, respectively. According to the first terms, the diffusive transport in
the direction parallel to the magnetic field is exactly the same as that in the
unmagnetized case: {\em i.e.}, it corresponds to 
collision-induced random-walk diffusion
of the ions and electrons, with
frequency $\nu$, and step-length $l$. According to the
second terms, the diffusive transport in the direction perpendicular to the
magnetic field is {\em far smaller}\/ than that in the parallel direction.
In fact, it is smaller by a factor $(\rho/l)^2$, where $\rho$ is the
gyroradius, and $l$ the mean-free-path. Note, that the perpendicular
heat transport also corresponds to collision-induced random-walk diffusion
of charged particles,
but  with frequency $\nu$, and
step-length $\rho$. Thus, it is the greatly reduced step-length in the
perpendicular direction, relative to the parallel direction, which ultimately
gives rise to the strong reduction in the perpendicular heat transport. 
If $T_e\sim T_i$, then the ion perpendicular heat diffusivity actually
{\em exceeds}\/ that of the electrons by the square root of a mass ratio:
$\kappa_\perp^i/\kappa_\perp^e\sim \sqrt{m_i/m_e}$. 

The third terms on the right-hand sides of  Eqs.~(\ref{e3.94a}) and (\ref{e3.94b})
correspond to heat fluxes which are perpendicular to both the magnetic field
and the direction of the temperature gradient. In order to understand the
origin of these terms, let us consider the ion flux. Suppose that there
is a strong magnetic field along the $z$-axis, and an ion temperature gradient
along the $x$-axis---see Fig.~\ref{f8}. The ions gyrate in the $x$-$y$ plane
in circles of radius $\rho_i\sim v_i/{\mit\Omega}_i$, where $v_i$ is the
ion thermal velocity. At a given point, coordinate $x_0$, say, on the $x$-axis,
the ions that come from the right and the left have traversed distances of
order $\rho_i$. The ions from the right are clearly somewhat hotter than those
from the left. If the unidirectional particle fluxes, of order $n\,v_i$, are
balanced, then the unidirectional heat fluxes, of order $n\,T_i\,v_i$, will
have an unbalanced component of fractional order $(\rho_i/T_i)\partial
T_i/\partial x$. As a result, there is a net heat flux in the $+y$-direction
({\em i.e.}, the direction of ${\bf b}\times\nabla T_i$). The magnitude of
this flux is
\begin{equation}
 q_\times^i \sim n\,v_i\, \rho_i\,\frac{\partial T_i}{\partial x}
\sim \frac{n\,T_i}{m_i\,|{\mit\Omega}_i|}\,\frac{\partial T_i}{\partial x}.
\end{equation}
There is an analogous expression for the electron flux, except that the electron
flux is in the opposite direction 
to the ion flux (because the electrons gyrate in the opposite
direction to the ions). Note that both ion and electron fluxes transport
heat  {\em along isotherms}, and do not, therefore, give rise to
any plasma heating. 

\begin{figure}
\epsfysize=3in
\centerline{\epsffile{Chapter03/brag2.eps}}
\caption{\em Origin of the convective  perpendicular heat flux in a magnetized plasma.}\label{f8}
\end{figure}

The fourth and fifth terms on the right-hand side of Eq.~(\ref{e3.94a}) correspond to
the convective component of the electron heat flux density,  driven by
 motion of the electrons relative to the ions. It is clear from the
fourth  term that the convective flux parallel to the magnetic field is exactly the
same as in the unmagnetized case [see Eq.~(\ref{e3.86a})]. However, according to the fifth term, the
convective flux is radically modified in the perpendicular direction. 
Probably the easiest method of explaining the fifth
 term is via an examination
of Eqs.~(\ref{e3.82a}), (\ref{e3.86a}), (\ref{e3.91a}), and (\ref{e3.94a}). There is clearly a very close 
connection between the electron thermal force and the convective heat flux. 
In fact, starting from general principles of the thermodynamics of irreversible
processes, the so-called {\em Onsager principles}, it is possible to
demonstrate that an electron frictional force of the form 
$\alpha\,(\nabla\,T_e)_\beta\,{\bf i}$ necessarily gives rise to an electron heat flux
of the form $\alpha\,(T_e\,j_\beta/ne)\,{\bf i}$, where the
subscript $\beta$ corresponds to a general Cartesian component, and ${\bf i}$
is a unit vector. Thus, the fifth term on the right-hand side of Eq.~(\ref{e3.94a})
follows by {\em Onsager symmetry}\/ from the third term on the right-hand
side of Eq.~(\ref{e3.91a}). This is one of many Onsager symmetries which
occur in plasma transport theory. 

In order to describe the viscosity tensor in a magnetized plasma, it is
helpful to define the {\em rate-of-strain tensor}
\begin{equation}
W_{\alpha\beta} = \frac{\partial V_\alpha}{\partial r_\beta}
+ \frac{\partial V_\beta}{\partial r_\alpha} - \frac{2}{3} \,\nabla\!\cdot\!{\bf V}\,
\delta_{\alpha\beta}.
\end{equation}
Obviously, there is a separate rate-of-strain tensor for the electron and ion
fluids. It is easily demonstrated that this tensor is zero if the plasma
translates or rotates as a rigid body, or if it undergoes isotropic
compression. Thus, the rate-of-strain tensor measures the {\em deformation}\/  of
plasma volume elements. 

In a magnetized plasma, the viscosity tensor is best described as the
sum of {\em five}\/ component tensors,
\begin{equation}
\bpi  = \sum_{n=0}^4 \bpi_n,
\end{equation}
where 
\begin{equation}
\bpi_0 = - 3\,\eta_0\,\left({\bf b}{\bf b} - \frac{1}{3}\,{\bf I}\right)
\left({\bf b}{\bf b} - \frac{1}{3}\,{\bf I}\right): \nabla {\bf V},
\end{equation}
with
\begin{equation}
\bpi_1 =- \eta_1\left[{\bf I}_\perp\! \cdot\!{\bf W}\!\cdot\!{\bf I}_\perp
+ \frac{1}{2}\,{\bf I}_\perp\,({\bf b}\!\cdot\!{\bf W}\!\cdot\!{\bf b})\right],
\end{equation}
and
\begin{equation}
\bpi_2 = -4\,\eta_1\,\left[ {\bf I}_\perp\!\cdot\!
{\bf W}\!\cdot\!{\bf b}{\bf b}
+ {\bf b}{\bf b}\!\cdot\!{\bf W} \!\cdot\!{\bf I}_\perp\right].
\end{equation}
plus
\begin{equation}
\bpi_3 = \frac{\eta_3}{2}\,\left[ {\bf b}\times 
{\bf W}\!\cdot\!{\bf I}_\perp - {\bf I}_\perp\!\cdot\!{\bf W}\times{\bf b}
\right],
\end{equation}
and
\begin{equation}
\bpi_4 = 2\,\eta_3\,\left[{\bf b} \times{\bf W} \!\cdot\! {\bf b}{\bf b}
- {\bf b}{\bf b} \!\cdot\!{\bf W} \times{\bf b}\right].
\end{equation}
Here, ${\bf I}$ is the identity tensor, and 
${\bf I}_\perp = {\bf I} - {\bf b}{\bf b}$. The above
expressions are valid for both electrons and ions. 

The tensor $\bpi_0$ describes what is known as {\em parallel viscosity}.
This is a viscosity which controls the variation along magnetic field-lines of the
velocity component parallel to  field-lines.
 The parallel
viscosity coefficients, $\eta_0^e$ and $\eta_0^i$ are specified in Eqs.~(\ref{e3.89a})--(\ref{e3.89b}). 
Note that the parallel viscosity is unchanged from the unmagnetized case,
and is due to the collision-induced random-walk diffusion of particles,
with frequency $\nu$, and step-length $l$. 

The tensors $\bpi_1$ and $\bpi_2$ describe what is known
as {\em perpendicular viscosity}. This is a viscosity
which controls the variation perpendicular to magnetic field-lines
of the velocity components perpendicular to field-lines. The perpendicular
viscosity coefficients are given by
\begin{eqnarray}
\eta_1^e &=& 0.51\, \frac{n\,T_e}{{\mit\Omega}_e^{~2}\,\tau_e},\\[0.5ex]
\eta_1^i &=& \frac{3\, n\,T_i}{10\,{\mit\Omega}_i^{~2}\,\tau_i}.
\end{eqnarray}
Note that the perpendicular viscosity is {\em far smaller}\/ than the parallel
viscosity. In fact, it is smaller by a factor $(\rho/l)^2$. The
perpendicular viscosity corresponds to collision-induced random-walk diffusion
of particles, with frequency $\nu$, and step-length $\rho$. Thus, it
is the greatly reduced step-length in the perpendicular direction, relative
to the parallel direction, which accounts for the smallness of the
perpendicular viscosity compared to the parallel viscosity. 

Finally, the tensors $\bpi_3$ and $\bpi_4$ describe what is known
as {\em gyroviscosity}. This is not really viscosity at all, since the
associated viscous stresses are always perpendicular to the velocity, implying that
there is no dissipation ({\em i.e.}, viscous heating) associated with
this effect. The gyroviscosity coefficients are given by 
\begin{eqnarray}
\eta_3^e &=& -\frac{n\,T_e}{2\,|{\mit\Omega}_e|} ,\\[0.5ex]
\eta_3^i &=& \frac{n\,T_i}{2\,{\mit\Omega}_i}.
\end{eqnarray}
The origin of gyroviscosity is very similar to the origin of the
cross thermal conductivity terms in Eqs.~(\ref{e3.94a})--(\ref{e3.94b}). Note that both
cross thermal conductivity and gyroviscosity are {\em independent}\/
of the collision frequency. 

\section{Normalization of the Braginskii Equations}\label{s3.9}
As we have just seen, the Braginskii equations contain terms which describe a
very wide range of physical phenomena. For this reason, they are
extremely complicated. Fortunately, however, it is not generally 
necessary to retain all of
the terms in these equations when investigating a particular
problem in plasma physics: {\em e.g.}, electromagnetic wave propagation
through plasmas. In this section, we shall attempt to construct a systematic
normalization scheme for the Braginskii equations which will, hopefully, enable us
to determine which terms to keep, and which to discard, when investigating
a particular aspect of plasma physics. 

Let us consider a magnetized plasma. It is convenient to split the friction force
${\bf F}$ into a component ${\bf F}_U$ due to resistivity, and a
component ${\bf F}_T$ corresponding to the thermal force. Thus,
\begin{equation}
{\bf F} = {\bf F}_U+{\bf F}_T,
\end{equation}
where\begin{eqnarray}
{\bf F}_U&=& ne\left(\frac{{\bf j}_\parallel}{\sigma_\parallel}
+\frac{{\bf j}_\perp}{\sigma_\perp}\right), \\[0.5ex]
{\bf F}_T &=& - 0.71\,n\,\nabla_\parallel T_e
-\frac{3\,n}{2\,|{\mit\Omega}_e|\,\tau_e}\,{\bf b}\times\nabla_\perp T_e.
\end{eqnarray}
Likewise, the electron collisional energy gain term $W_e$ is split
into a component $-W_i$ due to the energy lost to the ions (in the
ion rest frame), a component $W_U$ due to work done by the friction
force ${\bf F}_U$, and a component $W_T$ due to work done by the
thermal force ${\bf F}_T$. Thus,
\begin{equation}
W_e = -W_i + W_U + W_T,
\end{equation}
where
\begin{eqnarray}
W_U &=& \frac{{\bf j}\cdot{\bf F}_U}{ne},\\[0.5ex]
W_T&=& \frac{{\bf j}\cdot{\bf F}_T}{ne}.
\end{eqnarray}
Finally, it is helpful to split the electron heat flux density ${\bf q}_e$ into
a diffusive component ${\bf q}_{Te}$ and a convective component ${\bf q}_{Ue}$. 
Thus,
\begin{equation}
{\bf q}_e = {\bf q}_{Te} + {\bf q}_{Ue},
\end{equation}
where
\begin{eqnarray}
{\bf q}_{Te} &=&-\kappa_\parallel^e\,\nabla_\parallel T_e -\kappa_\perp^e\,
\nabla_\perp T_e
-\kappa_\times^e\,{\bf b}\times\nabla_\perp T_e,\\[0.5ex]
{\bf q}_{Ue}&=&  0.71\,\frac{T_e\,{\bf j}_\parallel}{e}-
\frac{3\,T_e}{2\,|{\mit\Omega}_e|\,\tau_e\,e}\,{\bf b}\times{\bf j}_\perp.
\end{eqnarray}


Let us, first of all, consider the electron fluid equations, which can
be written:
\begin{eqnarray}
\frac{dn}{dt} + n\,\nabla\!\cdot\!{\bf V}_e &=& 0,\\[0.5ex]
m_e n\,\frac{d {\bf V}_e}{dt} + \nabla p_e+ \nabla\!\cdot \!\bpi_e + e n\,
({\bf E} + {\bf V}_e\times {\bf B})& =& {\bf F}_U +{\bf F}_T,\\[0.5ex]
\frac{3}{2}\frac{d p_e}{dt} + \frac{5}{2}\,p_e\,\nabla\!\cdot\!{\bf V}_e
+ \bpi_e:\nabla{\bf V}_e+ \nabla\!\cdot\!{\bf q}_{Te} 
+ \nabla\!\cdot\!{\bf q}_{Ue} &=& -W_i\\[0.5ex]
&&+W_U + W_T.\nonumber
\end{eqnarray}
Let $\bar{n}$, $\bar{v}_e$, $\bar{l}_e$, $\bar{B}$,
and $\bar{\rho}_e =\bar{ v}_e/(e\bar{B}/m_e)$,  be typical values
of the particle density, the electron thermal velocity, the electron
mean-free-path,  the magnetic field-strength, and the
electron gyroradius,  respectively. 
Suppose that the typical electron flow velocity is $\lambda_e\,\bar{v}_e$, and
the typical variation length-scale is $L$. Let
 \begin{eqnarray}
\delta_e &=& \frac{\bar{\rho}_e}{L},\\[0.5ex]
\zeta_e &=& \frac{\bar{\rho}_e}{\bar{l}_e},\\[0.5ex]
\mu &=& \sqrt{\frac{m_e}{m_i}}.
\end{eqnarray}
All three of these parameters are assumed to be {\em small}\/
compared to unity.

 We
define the following normalized quantities: $\hat{n} = n/\bar{n}$,
$\hat{v}_e = v_e/\bar{v}_e$, $\hat{\bf r} = {\bf r}/ L$, 
$\hat{\nabla} = L\,\nabla$, $\hat{t} = \lambda_e\,\bar{v}_e\,t/L$, 
$\hat{\bf V}_e = {\bf V}_e/\lambda_e\,\bar{v}_e$, $\hat{\bf B}= {\bf B}
/\bar{B}$, $\hat{\bf E} = {\bf E}/ \lambda_e\,\bar{v}_e\,\bar{B}$,
$\hat{\bf U} ={\bf U} / (1+\lambda_e^{~2})\,\delta_e\,\bar{v}_e$, 
plus 
$\hat{p}_e = p_e/m_e\,\bar{n}\,\bar{v}_e^{~2}$, $\hat{\bpi}_e= \bpi_e
/\lambda_e\,\delta_e\,\zeta_e^{-1}\,m_e\,\bar{n}\,\bar{v}_e^{~2}$, 
$\hat{\bf q}_{Te} = {\bf q}_{Te} / \delta_e\,\zeta_e^{-1}\,
m_e\,\bar{n}\,\bar{v}_e^{~3}$, $\hat{\bf q}_{Ue} = {\bf q}_{Ue} / 
(1+\lambda_e^{~2})\,\delta_e\,m_e\,\bar{n}\,\bar{v}_e^{~3}$, 
$\hat{\bf F}_U = {\bf F}_U/ (1+\lambda_e^{~2}) \,\zeta_e\,
m_e\,\bar{n}\,\bar{v}_e^{~2}/L$, 
$\hat{\bf F}_T = {\bf F}_T/m_e\,\bar{n}\,\bar{v}_e^{~2}/L$,
$\hat{W}_i = W_i/\delta_e^{-1}\,\zeta_e\,\mu^2\, m_e\,\bar{n}\,\bar{v}_e^{~3}/L$,
$\hat{W}_U = W_U/(1+\lambda_e^{~2})^2\,\delta_e\,\zeta_e\, m_e\,
\bar{n}\,\bar{v}_e^{~3}/L$,
$\hat{W}_T = W_T/(1+\lambda_e^{~2})\,\delta_e\, m_e\,\bar{n}\,\bar{v}_e^{~3}/L$.

The normalization procedure is designed to make all hatted quantities $O(1)$.
The normalization of the electric field is chosen 
 such that the ${\bf E}\times{\bf B}$
velocity is of order the electron fluid velocity. Note that the parallel viscosity
makes an $O(1)$ contribution to $\hat{\bpi}_e$, whereas the gyroviscosity
makes an $O(\zeta_e)$ contribution, and the perpendicular viscosity only 
makes an $O(\zeta_e^{~2})$ contribution. Likewise, the parallel thermal
conductivity 
makes an $O(1)$ contribution to $\hat{\bf q}_{Te}$, whereas the cross
conductivity 
makes an $O(\zeta_e)$ contribution, and the perpendicular conductivity only 
makes an $O(\zeta_e^{~2})$ contribution. Similarly, the parallel components
of ${\bf F}_T$ and ${\bf q}_{Ue}$ are $O(1)$, whereas the perpendicular
components are $O(\zeta_e)$. 

The normalized electron fluid equations take the form:
\begin{eqnarray}
\frac{d\hat{n}}{d\hat{t}} + \hat{n}\,\hat{\nabla}\!\cdot\!\hat{\bf V}_e &=&0,\label{e3.113a}\\[0.5ex]
\lambda_e^{~2}\,\delta_e\,\hat{n}\,\frac{d\hat{\bf V}_e}{d\hat{t}} + \delta_e\,
\hat{\nabla}\hat{p}_e + \lambda_e\,\delta_e^{~2}\,\zeta_e^{-1}\,\hat{\nabla}
\!\cdot\!\hat{\bpi}_e  \\[0.5ex]
+ \lambda_e\,\hat{n}\,(\hat{\bf E} + \hat{\bf V}_e
\times\hat{\bf B}) &=& (1+\lambda_e^{~2})\,\delta_e\,\zeta_e\,\hat{\bf F}_U + \delta_e
\,\hat{\bf F}_T, \nonumber\\[0.5ex]
\lambda_e\,\frac{3}{2}\frac{d\hat{p}_e}{d\hat{t}} + \lambda_e\,\frac{5}{2}\,
\hat{p}_e\,\hat{\nabla}\!\cdot\!\hat{\bf V}_e + \lambda_e^{~2}\,\delta_e\,
\zeta_e^{-1}\,\hat{\bpi}_e : \hat{\nabla}\!\cdot\!\hat{\bf V}_e&&\label{e3.113c}\\[0.5ex] 
+\delta_e\,\zeta_e^{-1} \,\hat{\nabla}\!\cdot\!\hat{\bf q}_{Te}
 +(1+\lambda_e^{~2})\,\delta_e\,\hat{\nabla}\!\cdot\!\hat{\bf q}_{Ue} 
&=& -\delta_e^{-1}\,\zeta_e\,\mu^2\,\hat{W}_i \nonumber\\[0.5ex]
&&+ (1+\lambda_e^{~2})^2\,\delta_e\,\zeta_e\,
\hat{W}_U \nonumber\\[0.5ex]&& + (1+\lambda_e^{~2})\,\delta_e\,\hat{W}_T.\nonumber
\end{eqnarray}
Note that the only large or small quantities in these equations are the
parameters $\lambda_e$, $\delta_e$, $\zeta_e$, and $\mu$. 
Here, $d/d\hat{t}\equiv\partial /\partial\hat{t} +
 \hat{\bf V}_e\!\cdot\!\hat{\nabla}$. It is assumed that $T_e\sim T_i$.



Let us now consider the ion fluid equations, which can be written:
\begin{eqnarray}
\frac{dn}{dt} + n\,\nabla\!\cdot\!{\bf V}_i &=& 0,\\[0.5ex]
m_i n\,\frac{d {\bf V}_i}{dt} + \nabla p_i + \nabla\!\cdot \!\bpi_i - e n\,
({\bf E} + {\bf V}_i\times {\bf B})& =&- {\bf F}_U -{\bf F}_T,\\[0.5ex]
\frac{3}{2}\frac{d p_i}{dt} + \frac{5}{2}\,p_i\,\nabla\!\cdot\!{\bf V}_i
+ \bpi_i:\nabla{\bf V}_i+ \nabla\!\cdot\!{\bf q}_i &=& W_i.
\end{eqnarray}
It is convenient to adopt a normalization scheme for the ion equations
which is similar to, but independent of, that employed to normalize the
electron equations. Let  $\bar{n}$, $\bar{v}_i$, $\bar{l}_i$, $\bar{B}$,
and $\bar{\rho}_i =\bar{ v}_i/(e\bar{B}/m_i)$,  be typical values
of the particle density, the ion  thermal velocity, the ion
mean-free-path,  the magnetic field-strength, and the
ion gyroradius,  respectively. 
Suppose that the typical ion flow velocity is $\lambda_i\,\bar{v}_i$, and
the typical variation length-scale is $L$. Let
 \begin{eqnarray}
\delta_i &=& \frac{\bar{\rho}_i}{L},\\[0.5ex]
\zeta_i &=& \frac{\bar{\rho}_i}{\bar{l}_i},\\[0.5ex]
\mu &=& \sqrt{\frac{m_e}{m_i}}.
\end{eqnarray}
All three of these parameters are assumed to be {\em small}\/
compared to unity.

We
define the following normalized quantities: $\hat{n} = n/\bar{n}$,
$\hat{v}_i = v_i/\bar{v}_i$, $\hat{\bf r} = {\bf r}/ L$, 
$\hat{\nabla} = L\,\nabla$, $\hat{t} = \lambda_i\,\bar{v}_i\,t/L$, 
$\hat{\bf V}_i = {\bf V}_i/\lambda_i\,\bar{v}_i$, $\hat{\bf B}= {\bf B}
/\bar{B}$, $\hat{\bf E} = {\bf E}/ \lambda_i\,\bar{v}_i\,\bar{B}$,
$\hat{\bf U} ={\bf U} / (1+\lambda_i^{~2})\,\delta_i\,\bar{v}_i$, 
$\hat{p}_i = p_i/m_i\,\bar{n}\,\bar{v}_i^{~2}$, $\hat{\bpi}_i= \bpi_i
/\lambda_i\,\delta_i\,\zeta_i^{-1}\,m_i\,\bar{n}\,\bar{v}_i^{~2}$, 
$\hat{\bf q}_{i} = {\bf q}_{i} / \delta_i\,\zeta_i^{-1}\,
m_i\,\bar{n}\,\bar{v}_i^{~3}$,  
$\hat{\bf F}_U = {\bf F}_U/ (1+\lambda_i^{~2}) \,\zeta_i\,\mu\,
m_i\,\bar{n}\,\bar{v}_i^{~2}/L$, 
$\hat{\bf F}_T = {\bf F}_T/m_i\,\bar{n}\,\bar{v}_i^{~2}/L$,
$\hat{W}_i = W_i/\delta_i^{-1}\,\zeta_i\,\mu\, m_i\,\bar{n}\,\bar{v}_i^{~3}/L$.

As before, the normalization procedure is designed to make all hatted quantities $O(1)$.
The normalization of the electric field is chosen 
 such that the ${\bf E}\times{\bf B}$
velocity is of order the ion fluid velocity. Note that the parallel viscosity
makes an $O(1)$ contribution to $\hat{\bpi}_i$, whereas the gyroviscosity
makes an $O(\zeta_i)$ contribution, and the perpendicular viscosity only 
makes an $O(\zeta_i^{~2})$ contribution. Likewise, the parallel thermal
conductivity 
makes an $O(1)$ contribution to $\hat{\bf q}_{i}$, whereas the cross
conductivity 
makes an $O(\zeta_i)$ contribution, and the perpendicular conductivity only 
makes an $O(\zeta_i^{~2})$ contribution. Similarly, the parallel component
of ${\bf F}_T$ is  $O(1)$, whereas the perpendicular
component is  $O(\zeta_i\,\mu)$. 

The normalized ion fluid equations take the form:
\begin{eqnarray}\label{e3.116a}
\frac{d\hat{n}}{d\hat{t}} + \hat{n}\,\hat{\nabla}\!\cdot\!\hat{\bf V}_i &=&0,\\[0.5ex]
\lambda_i^{~2}\,\delta_i\,\hat{n}\,\frac{d\hat{\bf V}_i}{d\hat{t}} + \delta_i\,
\hat{\nabla}\hat{p}_i+ \lambda_i\,\delta_i^{~2}\,\zeta_i^{-1}\,\hat{\nabla}
\!\cdot\!\hat{\bpi}_i  \\[0.5ex]
-\lambda_i\,\hat{n}\,(\hat{\bf E} + \hat{\bf V}_i
\times\hat{\bf B}) &=& -(1+\lambda_i^{~2})\,\delta_i\,\zeta_i\,\mu\,\hat{\bf F}_U
 - \delta_i
\,\hat{\bf F}_T, \nonumber\\[0.5ex]
\lambda_i\,\frac{3}{2}\frac{d\hat{p}_i}{d\hat{t}} + \lambda_i\,\frac{5}{2}\,
\hat{p}_i\,\hat{\nabla}\!\cdot\!\hat{\bf V}_i + \lambda_i^{~2}\,\delta_i\,
\zeta_i^{-1}\,\hat{\bpi}_i : \hat{\nabla}\!\cdot\!\hat{\bf V}_i&&\label{e3.116c}\\[0.5ex] 
+\delta_i\,\zeta_i^{-1} \,\hat{\nabla}\!\cdot\!\hat{\bf q}_{i}
&=& \delta_i^{-1}\,\zeta_i\,\mu\,\hat{W}_i. \nonumber
\end{eqnarray}
Note that the only large or small quantities in these equations are the
parameters $\lambda_i$, $\delta_i$, $\zeta_i$, and $\mu$. 
Here, $d/d\hat{t}\equiv\partial /\partial\hat{t} +
 \hat{\bf V}_i\!\cdot\!\hat{\nabla}$.

Let us adopt the ordering
\begin{equation}
\delta_e, \delta_i \ll \zeta_e, \zeta_i,\,\,\, \mu \ll 1,
\end{equation}
which is appropriate to a collisional, {\em highly magnetized}\/ plasma. 
In the first stage of our ordering procedure, we shall treat $\delta_e$ and $\delta_i$
as small parameters, and $\zeta_e$, $\zeta_i$, and $\mu$ as $O(1)$.
 In the second stage, we shall take note of the smallness of 
$\zeta_e$, $\zeta_i$, and $\mu$. Note that the parameters $\lambda_e$ and $\lambda_i$
are ``free ranging:'' {\em i.e.}, they can be either large, small, or $O(1)$. 
In the initial stage of the
ordering procedure, the ion and electron
normalization schemes we have adopted become  essentially
identical [since $\mu\sim O(1)$], and 
it is convenient to write 
\begin{eqnarray}
\lambda_e\sim \lambda_i&\sim & \lambda,\\[0.5ex]
\delta_e \sim \delta_i &\sim &\delta,\\[0.5ex]
V_e \sim V_i &\sim & V,\\[0.5ex]
v_e \sim v_i &\sim & v_t,\\[0.5ex]
{\mit\Omega}_e \sim {\mit\Omega}_i &\sim & {\mit\Omega}.
\end{eqnarray}


There are {\em three fundamental orderings}\/  in plasma fluid theory. These are
analogous to the three orderings in neutral gas fluid theory discussed in Sect.~\ref{s3.7}.

The first ordering is
\begin{equation}\label{e3.119}
\lambda \sim \delta^{-1}.
\end{equation}
This corresponds to
\begin{equation}\label{e3.120}
V\gg v_t.
\end{equation}
In other words, the fluid velocities are much greater than the 
thermal velocities. We also have
\begin{equation}\label{e3.121}
\frac{V}{L} \sim {\mit\Omega}.
\end{equation}
Here, $V/L$ is conventionally termed the {\em transit frequency}, and is
the frequency with which fluid elements traverse the system. It is clear
that the transit frequencies are of order the gyrofrequencies
in this ordering. Keeping only the largest terms in Eqs.~(\ref{e3.113a})--(\ref{e3.113c}) and
(\ref{e3.116a})--(\ref{e3.116c}), the Braginskii equations reduce to (in unnormalized form):
\begin{eqnarray}\label{e3.122a}
\frac{dn}{dt} + n\,\nabla\!\cdot\!{\bf V}_e &=& 0,\\[0.5ex]
m_e n\,\frac{d {\bf V}_e}{dt}  + e n\,
({\bf E} + {\bf V}_e\times {\bf B})& =& [\zeta]\,{\bf F}_U,\label{e3.122b}
\end{eqnarray}
and
\begin{eqnarray}\label{e3.123a}
\frac{dn}{dt} + n\,\nabla\!\cdot\!{\bf V}_i &=& 0,\\[0.5ex]
m_i n\,\frac{d {\bf V}_i}{dt}  - e n\,
({\bf E} + {\bf V}_i\times {\bf B})& =&- [\zeta]\,{\bf F}_U.\label{e3.123b}
\end{eqnarray}
The factors in square brackets are just to remind us that the terms they precede
are  smaller than the other terms in the equations (by the 
corresponding factors inside
the  
brackets). 

Equations (\ref{e3.122a})--(\ref{e3.122b}) and (\ref{e3.123a})--(\ref{e3.123b}) are called the {\em cold-plasma equations}, because
they can be obtained from the Braginskii equations by formally taking the
limit $T_e, T_i\rightarrow 0$. Likewise, the ordering (\ref{e3.119}) is called
the {\em cold-plasma approximation}. Note that the cold-plasma approximation
applies not only to cold plasmas, but also to very {\em fast disturbances}\/  which
propagate through conventional plasmas. In particular,
the cold-plasma equations provide a good description of the propagation
of {\em electromagnetic waves}\/ through plasmas. After all, electromagnetic
waves generally have very high velocities ({\em i.e.}, $V\sim c$), 
which they impart to
plasma fluid elements, so there is usually
no difficulty satisfying the inequality (\ref{e3.120}). 

Note that the electron and ion pressures can be neglected in the cold-plasma
limit, since the thermal velocities are much smaller than the
fluid velocities. It follows that there is no need for an electron
or  ion energy evolution equation. Furthermore, the motion of the
plasma is so fast, in this limit, that relatively slow ``transport'' effects,
such as viscosity and thermal conductivity, play no role in the cold-plasma
fluid equations. In fact, the only collisional effect which appears in these
equations is {\em resistivity}. 


The second ordering is
\begin{equation}\label{e3.124}
\lambda \sim 1,
\end{equation}
which corresponds to
\begin{equation}
V \sim v_t.
\end{equation}
In other words, the fluid velocities are of order the thermal
velocities. Keeping only the largest terms in Eqs.~(\ref{e3.113a})--(\ref{e3.113c}) and (\ref{e3.116a})--(\ref{e3.116c}),
the Braginskii equations reduce to (in unnormalized form):
\begin{eqnarray}\label{e3.126a}
\frac{dn}{dt} + n\,\nabla\!\cdot\!{\bf V}_e &=& 0,\\[0.5ex]
m_e n\,\frac{d {\bf V}_e}{dt} + \nabla p_e+   [\delta^{-1}]\,e n\,
({\bf E} + {\bf V}_e\times {\bf B})& =& [\zeta]\,{\bf F}_U + {\bf F}_T,\\[0.5ex]
\frac{3}{2}\frac{d p_e}{dt} + \frac{5}{2}\,p_e\,\nabla\!\cdot\!{\bf V}_e
&=& -[\delta^{-1}\,\zeta\,\mu^2]\,W_i,\label{e3.126c}
\end{eqnarray}
and
\begin{eqnarray}\label{e3.127a}
\frac{dn}{dt} + n\,\nabla\!\cdot\!{\bf V}_i &=& 0,\\[0.5ex]
m_i n\,\frac{d {\bf V}_i}{dt} + \nabla p_i - [\delta^{-1}]\, e n\,
({\bf E} + {\bf V}_i\times {\bf B})& =&- [\zeta]\,{\bf F}_U
-{\bf F}_T,\label{e3.127b}\\[0.5ex]
\frac{3}{2}\frac{d p_i}{dt} + \frac{5}{2}\,p_i\,\nabla\!\cdot\!{\bf V}_i
 &=&[\delta^{-1}\,\zeta\,\mu^2]\, W_i.\label{e3.127c}
\end{eqnarray}
Again, the factors in square brackets remind us that the terms they precede
are larger, or smaller, than the other terms in the equations. 

Equations (\ref{e3.126a})--(\ref{e3.126c}) and (\ref{e3.127a})--(\ref{e3.127b}) are called the {\em magnetohydrodynamical equations},
or {\em MHD equations}, for short. Likewise, the ordering (\ref{e3.124}) is called
the {\em MHD approximation}. The MHD equations are conventionally
used to study macroscopic  plasma
instabilities possessing  relatively fast growth-rates: {\em e.g.},
``sausage'' modes, ``kink'' modes. 

Note that the electron and ion pressures cannot be neglected in the MHD
limit, since the fluid velocities are of order the thermal
velocities. Thus, electron and ion energy evolution equations are needed
in this limit. However, MHD motion is sufficiently fast that 
``transport'' effects, such as viscosity and thermal conductivity,
are too slow to  play
a role in the MHD equations. In fact, the only collisional effects
which appear in these equations are resistivity, the thermal force, and
electron-ion collisional energy exchange. 

The final ordering is
\begin{equation}\label{e3.128}
\lambda\sim \delta,
\end{equation}
which corresponds to
\begin{equation}
V \sim \delta\,v_t \sim v_d,
\end{equation}
where $v_d$ is a typical drift ({\em e.g.}, a curvature or grad-B
drift---see Sect.~\ref{s2}) velocity. In other words, the fluid velocities
are of order the drift velocities. Keeping only the
largest terms in Eqs.~(3.113) and (3.116), the Braginskii equations reduce to
(in unnormalized form):
\begin{eqnarray}\label{e3.130a}
\frac{dn}{dt} + n\,\nabla\!\cdot\!{\bf V}_e &=& 0,\\[0.5ex]
m_e n\,\frac{d {\bf V}_e}{dt} + [\delta^{-2}]\,\nabla p_e+
[\zeta^{-1}]\, \nabla\!\cdot \!\bpi_e &&\\[0.5ex]
+ [\delta^{-2}]\,e n\,
({\bf E} + {\bf V}_e\times {\bf B})& =&[\delta^{-2}\,\zeta]\, {\bf F}_U + [\delta^{-2}]\,
{\bf F}_T,\nonumber\\[0.5ex]
\frac{3}{2}\frac{d p_e}{dt} + \frac{5}{2}\,p_e\,\nabla\!\cdot\!{\bf V}_e
+ [\zeta^{-1}]\,\nabla\!\cdot\!{\bf q}_{Te}
+ \nabla\!\cdot\!{\bf q}_{Ue} &=&-[\delta^{-2}\,\zeta\,\mu^2]\, W_i\\[0.5ex]
&&+[\zeta]\,W_U + W_T,\nonumber\label{e3.130c}
\end{eqnarray}
and
\begin{eqnarray}\label{e3.131a}
\frac{dn}{dt} + n\,\nabla\!\cdot\!{\bf V}_i &=& 0,\\[0.5ex]
m_i n\,\frac{d {\bf V}_i}{dt} + [\delta^{-2}]\,\nabla p_i + 
[\zeta^{-1}]\,\nabla\!\cdot \!\bpi_i &&\\[0.5ex]
- [\delta^{-2}]\,e n\,
({\bf E} + {\bf V}_i\times {\bf B})& =& -[\delta^{-2}\,\zeta]\, {\bf F}_U - [\delta^{-2}]\,{\bf F}_T,
\nonumber\\[0.5ex]
\frac{3}{2}\frac{d p_i}{dt} + \frac{5}{2}\,p_i\,\nabla\!\cdot\!{\bf V}_i
+ [\zeta^{-1}]\,\nabla\!\cdot\!{\bf q}_i &=& [\delta^{-2}\,\zeta\,\mu^2]\,W_i.\label{e3.131c}
\end{eqnarray}
As before, the factors in square brackets remind us that the terms they
precede are larger, or smaller, than the other terms in the equations. 

Equations (\ref{e3.130a})--(\ref{e3.130c}) and (\ref{e3.131a})--(\ref{e3.131c}) are called the {\em drift equations}. Likewise,
the ordering (\ref{e3.128}) is called the {\em drift approximation}. The drift equations
are conventionally used to study equilibrium evolution, and the slow
growing ``microinstabilities'' which are responsible for
turbulent transport in tokamaks. It is clear that virtually all
of the original terms in the Braginskii equations must be retained in
this limit. 

In the following sections, we investigate the cold-plasma equations, the
MHD equations, and the drift equations, in more detail.

\section{Cold-Plasma Equations}
Previously, we  used the smallness of the magnetization parameter $\delta$
to derive the cold-plasma equations:
\begin{eqnarray}\label{e3.132a}
\frac{\partial n}{\partial t} + \nabla\!\cdot\!(n\,{\bf V}_e) &=& 0,\\[0.5ex]
m_e n\,\frac{\partial   {\bf V}_e}{\partial t} + m_e n\,
({\bf V}_e\cdot \nabla ){\bf V}_e  + e n\,
({\bf E} + {\bf V}_e\times {\bf B})& =& [\zeta]\,{\bf F}_U,\label{e3.132b}
\end{eqnarray}
and
\begin{eqnarray}\label{e3.133a}
\frac{\partial n}{\partial t} + \nabla\!\cdot\!(n\,{\bf V}_i) &=& 0,\\[0.5ex]
m_i n\,\frac{\partial {\bf V}_i}{\partial t} +m_i n\,({\bf V}_i\cdot
\nabla){\bf V}_i - e n\,
({\bf E} + {\bf V}_i\times {\bf B})& =&- [\zeta]\,{\bf F}_U.\label{e3.133b}
\end{eqnarray}
Let us now use the smallness of the mass ratio $m_e/m_i$ to further
simplify these equations. In particular, we would like to write the
electron and ion fluid velocities in terms of the centre-of-mass
velocity,
\begin{equation}\label{e3.134}
{\bf V} = \frac{m_i\,{\bf V}_i + m_e\,{\bf V}_e}{m_i+ m_e},
\end{equation}
and the plasma current
\begin{equation}\label{e3.135}
{\bf j} = -ne\,{\bf U},
\end{equation}
where ${\bf U} = {\bf V}_e-{\bf V_i}$. According to the ordering
scheme adopted in the previous section, $U\sim V_e\sim V_i$ in the cold-plasma
limit. We shall continue to regard  the mean-free-path parameter $\zeta$ 
as $O(1)$.

It follows from Eqs.~(\ref{e3.134}) and (\ref{e3.135}) that
\begin{equation}\label{e3.136}
{\bf V}_i \simeq {\bf V} + O(m_e/m_i),
\end{equation}
and
\begin{equation}\label{e3.137}
{\bf V}_e \simeq {\bf V} - \frac{{\bf j}}{ne} + O\!\left(\frac{m_e}{m_i}\right).
\end{equation}

Equations (\ref{e3.132a}), (\ref{e3.133a}), (\ref{e3.136}), and (\ref{e3.137}) yield the
{\em continuity equation}:
\begin{equation}
\frac{dn}{dt} + n\,\nabla\!\cdot\!{\bf V} = 0,
\end{equation}
where $d/dt \equiv \partial/\partial t + {\bf V}\!\cdot\!\nabla$. Here, use has
been made of the fact that  $\nabla\!\cdot\!{\bf j} =0$ in a quasi-neutral
plasma.

Equations (\ref{e3.132b}) and (\ref{e3.133b}) can be summed  to give the
{\em equation of motion}:
\begin{equation}
m_i n\,\frac{d{\bf V}}{dt} -{\bf j}\times {\bf B} \simeq 0.
\end{equation}

Finally, Eqs.~(\ref{e3.132b}), (\ref{e3.136}), and (\ref{e3.137}) can be combined and to give
a modified  {\em Ohm's law}:
\begin{eqnarray}
{\bf E} + {\bf V}\times {\bf B}& \simeq& \frac{{\bf F}_U}{ne} + \frac{{\bf j}\times{\bf B}}
{ne} + \frac{m_e}{n e^2} \frac{d{\bf j}}{dt}\\[0.5ex]
 &&+ \frac{m_e}{ne^2}\, ({\bf j}\!\cdot\!\nabla){\bf V} - \frac{m_e}{n^2 e^3}\,
({\bf j}\!\cdot\!\nabla){\bf j}.\nonumber
\end{eqnarray}
The first term on the right-hand side of the above equation corresponds to
{\em resistivity}, the second corresponds to the {\em Hall effect}, the
third corresponds to the effect of {\em electron inertia}, and the
remaining terms are usually negligible.  

\section{MHD Equations}
The MHD equations take the form:
\begin{eqnarray}\label{e3.141a}
\frac{\partial n}{\partial t} + \nabla\!\cdot\!(n\,{\bf V}_e) &=& 0,\\[0.5ex]
\label{e3.141b}
m_e n\,\frac{\partial  {\bf V}_e}{\partial t} +
m_e n\,({\bf V}_e\!\cdot\!\nabla){\bf V}_e+
 \nabla p_e\\[0.5ex]
+   [\delta^{-1}]\,e n\,
({\bf E} + {\bf V}_e\times {\bf B})& =& [\zeta]\,{\bf F}_U + {\bf F}_T,
\nonumber\\[0.5ex]\label{e3.141c}
\frac{3}{2}\frac{\partial  p_e}{\partial t} + \frac{3}{2}
 \,({\bf V}_e\!\cdot\!\nabla)\, p_e 
+ \frac{5}{2}\,p_e\,\nabla\!\cdot\!{\bf V}_e
&=& -[\delta^{-1}\,\zeta\,\mu^2]\,W_i,
\end{eqnarray}
and
\begin{eqnarray}\label{e3.142a}
\frac{\partial n}{\partial t} +\nabla\!\cdot\!(n\,{\bf V}_i) &=& 0,\\[0.5ex]
\label{e3.142b}
m_i n\,\frac{\partial  {\bf V}_i}{\partial t} +
m_i n\,({\bf V}_i\!\cdot\!\nabla) {\bf V}_i
+ \nabla p_i \\[0.5ex]
- [\delta^{-1}]\, e n\,
({\bf E} + {\bf V}_i\times {\bf B})& =&- [\zeta]\,{\bf F}_U
-{\bf F}_T,\nonumber\\[0.5ex]
\frac{3}{2}\frac{\partial  p_i}{\partial t} +   \frac{3}{2}
 \,({\bf V}_i\!\cdot\!\nabla) \,p_i +
\frac{5}{2}\,p_i\,\nabla\!\cdot\!{\bf V}_i
 &=&[\delta^{-1}\,\zeta\,\mu^2]\, W_i.\label{e3.142c}
\end{eqnarray}
These equations can also be simplified by making use of the smallness
of the mass ratio $m_e/m_i$. Now, according to the ordering adopted in Sect.~\ref{s3.9},
$U \sim \delta\,V_e\sim \delta\,V_i$ in the MHD limit. It follows from Eqs.~(\ref{e3.136}) and (\ref{e3.137})
that 
\begin{equation}
{\bf V}_i \simeq {\bf V} + O(m_e/m_i),
\end{equation}
and
\begin{equation}
{\bf V}_e \simeq {\bf V} - [\delta]\,\frac{{\bf j}}{ne} + O\!\left(\frac{m_e}
{m_i}\right).
\end{equation}
The main point, here, is that in the MHD limit the velocity difference between
the electron and ion fluids is relatively small. 

Equations (\ref{e3.141a}) and (\ref{e3.142a}) yield the {\em continuity equation}:
\begin{equation}\label{e3.145}
\frac{dn}{dt} + n\, \nabla\!\cdot\!{\bf V} = 0,
\end{equation}
where $d/dt\equiv \partial/\partial t + {\bf V}\!\cdot\!\nabla$. 

Equations (\ref{e3.141b}) and (\ref{e3.142b}) can be summed to give the
{\em equation of motion}:
\begin{equation}
m_i n\,\frac{d{\bf V}}{dt} + \nabla p - {\bf j}\times{\bf B} \simeq 0.
\end{equation}
Here, $p=p_e+p_i$ is the total pressure. 
Note that all terms in the above  equation are the same order in $\delta$. 

The $O(\delta^{-1})$ components of Eqs.~(\ref{e3.141b}) and (\ref{e3.142b}) yield
the {\em Ohm's law}:
\begin{equation}\label{e3.147}
{\bf E} + {\bf V}\times {\bf B} \simeq 0.
\end{equation}
This is sometimes called the {\em perfect conductivity equation}, since
it is identical to the Ohm's law in a perfectly conducting liquid. 

Equations (\ref{e3.141c}) and (\ref{e3.142c}) can be summed to give the
{\em energy evolution equation}:
\begin{equation}\label{e3.148}
\frac{3}{2} \frac{dp}{dt} + \frac{5}{2}\, p \,\nabla\!\cdot\!{\bf V} \simeq 0.
\end{equation}
Equations (\ref{e3.145}) and (\ref{e3.148}) can be combined to give the more familiar
{\em adiabatic equation of state:}
\begin{equation}
\frac{d}{dt}\!\left(\frac{p}{n^{5/3}}\right) \simeq 0.
\end{equation}

Finally, the $O(\delta^{-1})$ components of Eqs.~(\ref{e3.141c}) and (\ref{e3.142c}) 
yield
\begin{equation}
W_i \simeq 0,
\end{equation}
or $T_e\simeq T_i$ [see Eq.~(\ref{e3.82b})]. Thus, we expect equipartition of the
thermal energy between electrons and ions in  the MHD limit. 

\section{Drift Equations}\label{s3.12}
The drift equations take the form:
\begin{eqnarray}
\frac{\partial n}{\partial t} + \nabla\!\cdot\!(n\,{\bf V}_e) &=& 0,\label{e3.151a}\\[0.5ex]
m_e n\,\frac{\partial  {\bf V}_e}{\partial t} +
m_e n\,({\bf V}_e\!\cdot\!\nabla){\bf V}_e+
 [\delta^{-2}]\,\nabla p_e+
[\zeta^{-1}]\, \nabla\!\cdot \!\bpi_e &&\label{e3.151b}\\[0.5ex]
 +[\delta^{-2}]\,e n\,
({\bf E} + {\bf V}_e\times {\bf B})& =&[\delta^{-2}\,\zeta]\, {\bf F}_U + [\delta^{-2}]\,
{\bf F}_T,\nonumber\\[0.5ex]
\frac{3}{2}\frac{\partial  p_e}{\partial t} + \frac{3}{2}
 \,({\bf V}_e\!\cdot\!\nabla)\, p_e 
+ \frac{5}{2}\,p_e\,\nabla\!\cdot\!{\bf V}_e&&\\[0.5ex]
+ [\zeta^{-1}]\,\nabla\!\cdot\!{\bf q}_{Te} +\nabla\!\cdot\!{\bf q}_{Ue}  &=&-[\delta^{-2}\,\zeta\,\mu^2]\,
 W_i\nonumber\\[0.5ex]
&&+ [\zeta]\,W_U + W_T,\nonumber\label{e3.151c}
\end{eqnarray}
and
\begin{eqnarray}\label{e3.152a}
\frac{\partial n}{\partial t} +\nabla\!\cdot\!(n\,{\bf V}_i) &=& 0,\\[0.5ex]
m_i n\,\frac{\partial  {\bf V}_i}{\partial t} +
m_i n\,({\bf V}_i\!\cdot\!\nabla) {\bf V}_i +
 [\delta^{-2}]\,\nabla p_i + 
[\zeta^{-1}]\,\nabla\!\cdot \!\bpi_i &&\\\label{e3.152b}[0.5ex]
- [\delta^{-2}]\,e n\,
({\bf E} + {\bf V}_i\times {\bf B})& =&-[\delta^{-2}\,\zeta]\, {\bf F}_U - [\delta^{-2}]\,{\bf F}_T,
\nonumber\\[0.5ex]
\frac{3}{2}\frac{\partial  p_i}{\partial t} +   \frac{3}{2}
 \,({\bf V}_i\!\cdot\!\nabla) \,p_i
+ \frac{5}{2}\,p_i\,\nabla\!\cdot\!{\bf V}_i\\[0.5ex]
+ [\zeta^{-1}]\,\nabla\!\cdot\!{\bf q}_i &=& [\delta^{-2}\,\zeta\,\mu^2]\,W_i.
\nonumber\label{e3.152c}
\end{eqnarray}

In the drift limit, the motions of the electron and ion fluids are
sufficiently different that there is little to be gained in rewriting the
drift equations in terms of the centre of mass velocity and the plasma
current. 
Instead, let us consider the $O(\delta^{-2})$ components of Eqs.~(\ref{e3.151b}) and (\ref{e3.152b}):
\begin{eqnarray}\label{e3.153a}
{\bf E} + {\bf V}_e\times{\bf B} &\simeq& - \frac{\nabla p_e}{en}
 -\frac{ 0.71\,\nabla_\parallel T_e}{e},\\[0.5ex]
{\bf E} + {\bf V}_i\times {\bf B} &\simeq& +\frac{\nabla p_i}{en} -
\frac{ 0.71\,\nabla_\parallel T_e}{e}.\label{e3.153b}
\end{eqnarray}
In the above equations, we
 have neglected all $O(\zeta)$ terms  for the
sake of simplicity.
Equations (\ref{e3.153a})--(\ref{e3.153b})  can be inverted to give
\begin{eqnarray}\label{e3.154a}
{\bf V}_{\perp\,e}& \simeq & {\bf V}_E + {\bf V}_{\ast\,e},
\\[0.5ex]
{\bf V}_{\perp\,i} &\simeq & {\bf V}_E + {\bf V}_{\ast\,i}.\label{e3.154b}
\end{eqnarray}
Here, ${\bf V}_E\equiv {\bf E}\times{\bf B}/B^2$ is the ${\bf E}\times{\bf B}$
velocity, whereas
\begin{equation}
{\bf V}_{\ast\,e} \equiv \frac{\nabla p_e \times{\bf B}}
{en\,B^2},
\end{equation}
and
\begin{equation}
{\bf V}_{\ast\,i} \equiv -\frac{\nabla p_i 
\times{\bf B}}
{en\,B^2},
\end{equation}
are termed the {\em electron diamagnetic velocity}\/ and the {\em ion diamagnetic
velocity}, respectively. 


According to Eqs.~(\ref{e3.154a})--(\ref{e3.154b}), in the drift approximation the velocity  of the electron 
fluid perpendicular to the magnetic field is the sum of the ${\bf E}\times{\bf B}$
velocity and the electron diamagnetic velocity. Similarly, for the ion fluid. 
Note that in the MHD approximation  the perpendicular velocities of the
two fluids consist of  the ${\bf E}\times{\bf B}$ velocity alone, and are,
therefore, identical to lowest order. The main difference between the
two ordering lies in the assumed magnitude of the electric field. In the
MHD limit
\begin{equation}
\frac{E}{B} \sim v_t,
\end{equation}
whereas in the drift limit
\begin{equation}
\frac{E}{B} \sim \delta\,v_t\sim v_d.
\end{equation}
Thus, the MHD ordering can be regarded as a {\em strong}\/ (in the sense
used in  Sect.~\ref{s2}) electric field
ordering, whereas the drift ordering corresponds to a {\em weak}\/ electric
field ordering. 

The diamagnetic velocities are so named because the {\em diamagnetic
current},
\begin{equation}
{\bf j}_\ast \equiv - en\,({\bf V}_{\ast\,e}-{\bf V}_{\ast\,i})
=-\frac{\nabla p\times{\bf B}}{B^2},
\end{equation}
generally acts to {\em reduce}\/ the magnitude of the magnetic field inside
 the plasma. 

The electron diamagnetic velocity can be written
\begin{equation}
{\bf V}_{\ast\,e} = \frac{T_e\,\nabla n\times{\bf b}}{en\,B} + \frac{
\nabla T_e\times{\bf b}}{e\,B}.
\end{equation}
In order to account for this velocity, let us consider a simplified
case in which the electron temperature is uniform, there is a
uniform {\em density}\/ gradient running along the $x$-direction, and the magnetic
field is parallel to the $z$-axis---see Fig.~\ref{f9}.
The electrons gyrate in the $x$-$y$ plane in circles of radius $\rho_e\sim v_e/
|{\mit\Omega}_e|$. At a given point, coordinate $x_0$, say, on the $x$-axis, 
the electrons that come from the right and the left have traversed distances
of order $\rho_e$. Thus, the electrons from the right originate from
regions where the particle density is of order $\rho_e\,\partial n/\partial x$
greater than the regions from which the electrons from the left originate. 
It follows  that the $y$-directed particle flux 
is unbalanced, with slightly more particles moving in the $-y$-direction
than in the $+y$-direction. Thus, there is a net particle flux
in the $-y$-direction: {\em i.e.}, in the direction of $\nabla n \times{\bf b}$. 
The magnitude of this flux is 
\begin{equation}
n\,V_{\ast\,e} \sim \rho_e\,\frac{\partial n}{\partial x}\,v_e \sim 
\frac{T_e}{e\,B} \frac{\partial n}{\partial x}.
\end{equation}
Note that there is no unbalanced particle flux in the $x$-direction, since the
$x$-directed fluxes are due to electrons which originate from regions
where $x=x_0$. We have now accounted for the first term on the
right-hand side of the above equation.
We can account for the second term using similar arguments. 
The ion diamagnetic velocity is similar in magnitude to the electron
diamagnetic velocity, but is {\em  oppositely}\/ directed, since ions gyrate
in the {\em opposite}\/ direction to electrons. 

\begin{figure}
\epsfysize=3in
\centerline{\epsffile{Chapter03/dia.eps}}
\caption{\em Origin of the diamagnetic velocity in a magnetized plasma.}\label{f9}
\end{figure}

The most curious aspect of  diamagnetic flows is that they represent fluid flows
for which there is {\em no corresponding motion}\/ of the particle guiding
centres. Nevertheless, the diamagnetic velocities are {\em real}\/ fluid
velocities, and the associated diamagnetic current is a {\em real}\/ current. 
For instance, the diamagnetic current contributes to force balance inside the plasma,
and also gives rise to ohmic heating. 

\section{Closure in Collisionless Magnetized Plasmas}\label{s3.13}
Up to now, we have only considered fluid closure in {\em collisional}\/  magnetized
plasmas.
Unfortunately, most magnetized plasmas encountered in nature---in particular,
fusion, space, and astrophysical plasmas---are {\em collisionless}. 
Let us consider what happens to the cold-plasma equations, the MHD equations,
and the drift equations, in the limit in which the mean-free-path goes
to infinity ({\em i.e.}, $\zeta\rightarrow 0$). 

In the limit $\zeta\rightarrow 0$, the cold-plasma equations reduce to
\begin{eqnarray}
\frac{dn}{dt} + n\,\nabla\!\cdot\!{\bf V} &=& 0,\label{e3.162a}\\[0.5ex]
m_i n\,\frac{d{\bf V}}{dt} -{\bf j}\times {\bf B}& =& 0,\\[0.5ex]
{\bf E} + {\bf V}\times {\bf B}& =& 
\frac{{\bf j}\times{\bf B}}
{ne} + \frac{m_e}{n e^2} \frac{d{\bf j}}{dt}\\[0.5ex]
 &&+ \frac{m_e}{ne^2}\, ({\bf j}\!\cdot\!\nabla){\bf V} - \frac{m_e}{n^2 e^3}\,
({\bf j}\!\cdot\!\nabla){\bf j}.\nonumber\label{e3.162c}
\end{eqnarray}
Here, we have neglected the resistivity term, since it is $O(\zeta)$. 
Note that none of the remaining terms in these equations depend 
explicitly on collisions. Nevertheless, the absence of collisions
poses a serious problem. Whereas the magnetic field effectively confines
charged particles in directions perpendicular to  magnetic field-lines,
by forcing them to execute tight Larmor orbits, we have now  lost all 
confinement along  field-lines. But, does this matter?
 
The typical frequency associated with fluid motion 
is the transit frequency, $V/L$. However, according to Eq.~(\ref{e3.121}),
the cold-plasma ordering implies that the transit frequency is of order a
typical gyrofrequency:
\begin{equation}\label{e3.163}
\frac{V}{L} \sim {\mit\Omega}.
\end{equation}
So, how far is a charged particle likely to drift along a field-line in
an inverse transit frequency? The answer is
\begin{equation}
{\mit\Delta} l_\parallel \sim \frac{v_t\,L}{V} \sim \frac{v_t}{\mit\Omega}\sim  \rho.
\end{equation} 
In other words, the fluid motion  in the cold-plasma limit is so fast that
charged particles only have time to drift a Larmor radius along field-lines
on a typical dynamical time-scale. Under these circumstances, it does not
really matter that the particles are not localized along field-lines---the lack
of parallel confinement manifests itself {\em too slowly}\/ 
 to affect the plasma dynamics. We conclude, therefore, that the cold-plasma
equations {\em remain valid in the collisionless limit}, provided, of course,
 that the
plasma dynamics are sufficiently rapid for the basic cold-plasma ordering
(\ref{e3.163}) to apply. In fact, the only difference between the collisional and 
collisionless cold-plasma equations  is the absence of the resistivity term in Ohm's law in the
latter case. 

Let us now consider the MHD limit. In this case, the typical
transit frequency is
\begin{equation}\label{e3.165}
\frac{V}{L}\sim \delta\,{\mit\Omega}.
\end{equation}
Thus, charged particles typically drift a distance
\begin{equation}
{\mit\Delta} l_\parallel \sim \frac{v_t\,L}{V} \sim \frac{v_t}{\delta\,{\mit\Omega}}\sim L
\end{equation}
along field-lines in an inverse transit frequency. In other words, the
fluid motion in the MHD limit is sufficiently slow that changed particles
have time to drift along field-lines  all the way across the system 
on a
typical dynamical time-scale. Thus, strictly speaking, the MHD equations
are {\em invalidated}\/ by the lack of particle confinement along magnetic
field-lines. 

In fact, in collisionless plasmas, MHD theory is  replaced by a theory
known as {\em kinetic-MHD}.\footnote{Kinetic-MHD is described in the following
two classic papers: M.D.~Kruskal, and C.R.~Oberman, Phys.\ Fluids {\bf 1}, 275
(1958): M.N.~Rosenbluth, and N.~Rostoker, Phys.\ Fluids {\bf 2}, 23 (1959).}
 The latter theory is a combination of a
one-dimensional kinetic theory, describing particle  motion along magnetic
field-lines, and a two-dimensional fluid theory, describing perpendicular
motion. As can well be imagined, the equations of kinetic-MHD are
considerably more complicated that the conventional MHD equations. 
Is there any situation in which we can salvage the simpler MHD equations in
a collisionless plasma? Fortunately,  there
is one case in which this is possible. 

It turns out that in both varieties of MHD the motion of the plasma
parallel to magnetic field-lines is associated with the dynamics of
{\em sound waves}, whereas the motion perpendicular to field-lines
is associated with the dynamics of a new type of wave called an {\em Alfv\'{e}n
wave}. As we shall see, later on,  Alfv\'{e}n
waves involve the ``twanging'' motion of magnetic field-lines---a bit
like the twanging of guitar strings. It is only the sound wave dynamics
which are significantly modified when we move from a collisional to a
collisionless plasma. It follows, therefore, that the MHD equations
remain a reasonable approximation in a collisionless plasma in situations where
the dynamics of sound waves,  parallel to the magnetic field, are unimportant
compared to the dynamics of Alfv\'{e}n waves, perpendicular to the field. 
This situation arises whenever the parameter
\begin{equation}
\beta = \frac{2\,\mu_0\,p}{B^2}
\end{equation}
(see Sect.~\ref{s1.10}) is much less than unity. In fact, it is easily
demonstrated that 
\begin{equation}
\beta \sim \left(\frac{V_S}{V_A}\right)^2,
\end{equation}
where $V_S$ is the sound speed ({\em i.e.}, thermal velocity), and $V_A$
is the speed of an Alfv\'{e}n wave. Thus, the inequality
\begin{equation}
\beta\ll 1
\end{equation}
ensures that the collisionless
parallel plasma dynamics are {\em  too slow}\/ to affect the
perpendicular dynamics. 

We conclude, therefore, that in a low-$\beta$, collisionless, magnetized
plasma the MHD equations,
\begin{eqnarray}
\frac{dn}{dt} + n\,\nabla\!\cdot\!{\bf V} &=& 0,\\[0.5ex]
m_i n\,\frac{d{\bf V}}{dt} &= & {\bf j}\times{\bf B} - \nabla p,\\[0.5ex]
{\bf E} + {\bf V} \times{\bf B} &=& 0,\\[0.5ex]
\frac{d}{dt}\!\left(\frac{p}{n^{5/3}}\right) &=& 0,
\end{eqnarray}
fairly well describe plasma dynamics which satisfy the basic MHD ordering
(\ref{e3.165}).

Let us, finally, consider the drift limit. In this case, the typical
transit frequency is
\begin{equation}
\frac{V}{L} \sim \delta^2\,{\mit\Omega}.
\end{equation}
Thus, charged particles
typically drift a distance
\begin{equation}
{\mit\Delta} l_\parallel \sim \frac{v_t\,L}{V} \sim \frac{L}{\delta}
\end{equation}
along field-lines in an inverse transit frequency. In other
words, the fluid motion in the drift limit is so slow
that charged particles drifting along
field-lines have time to traverse the system {\em very many times}\/ on a typical
dynamical time-scale. In fact, in this limit we have to
draw a distinction between those particles which always drift along field-lines
in the same direction, and those  particles which are trapped between magnetic
mirror points and, therefore,  continually reverse their direction of motion
along field-lines. The former are termed {\em passing particles}, whereas
the latter are termed {\em trapped particles}. 

Now, in the drift limit, the  perpendicular drift velocity of
charged particles, which is a combination of ${\bf E}\times{\bf B}$ drift,
grad-$B$ drift, and curvature drift (see Sect.~\ref{s2}), is of order
\begin{equation}
v_d \sim \delta\,v_t.
\end{equation}
Thus, charged particles typically drift a distance
\begin{equation}
{\mit\Delta} l_\perp \sim \frac{v_d\,L}{V} \sim L
\end{equation}
across field-lines in an inverse transit time. In other words, the
fluid motion in the drift limit is so slow that charged particles
have time
to drift perpendicular to field-lines all the way across the system on a
typical dynamical time-scale. 
It is, thus, clear that in the drift limit the absence of collisions implies
lack of confinement both parallel and perpendicular to the magnetic
field. This means that the collisional drift equations, (\ref{e3.151a})--(\ref{e3.151c}) and
(\ref{e3.152a})--(\ref{e3.152c}), are {\em completely invalid}\/ in the long mean-free-path limit. 

In fact, in collisionless plasmas, Braginskii-type transport theory---convention\-ally
known as {\em classical transport theory}---is replaced by a new theory---known as
{\em neoclassical transport  theory}\footnote{Neoclassical transport theory in
axisymmetric systems is
 described in the following classic papers: I.B.~Bernstein, Phys.\
Fluids {\bf 17}, 547 (1974): F.L.~Hinton, and R.D.~Hazeltine,
Rev.\ Mod.\ Phys.\ {\bf 48}, 239 (1976).}
---which is a combination of a two-dimensional
kinetic theory, describing particle motion on {\em drift surfaces}, and
a one-dimensional fluid theory, describing motion perpendicular to the drift
surfaces. Here, a drift surface is a closed surface formed by the locus of a
charged particle's drift orbit (including
drifts parallel and perpendicular to the magnetic field).
 Of course, the orbits only form closed surfaces if the plasma is {\em confined},
but there is little point in examining transport in an unconfined plasma. 
Unlike classical
transport theory, which is strictly {\em  local}\/ in nature,
neoclassical transport theory is {\em nonlocal}, in the sense that the
transport coefficients depend on the {\em average}\/ values of
 plasma properties taken over
drift surfaces. Needless to say, neoclassical transport theory
is horribly complicated!

\section{Langmuir Sheaths}
Virtually all terrestrial plasmas are contained inside solid vacuum vessels. So, an obvious question is: what happens to the plasma in the immediate vicinity of the  vessel wall?  Actually, to a first approximation, when ions and electrons hit a solid surface they recombine and are lost to the plasma. Hence, we can
treat the wall as a perfect sink of particles. Now, given that the electrons in a plasma generally
move much faster than the ions, the initial electron flux into the wall greatly
exceeds the ion flux, assuming that the wall starts off
unbiased with respect to the plasma. Of course, this flux imbalance causes the wall to charge up negatively, and so
generates a potential barrier which repels the electrons, and thereby
reduces the electron flux.
Debye shielding
confines this barrier to a thin layer of plasma, whose thickness is a few
Debye lengths, coating the inside surface of the wall. This layer
is known as a {\em plasma sheath}\/ or a {\em Langmuir sheath}. 
The height of the potential barrier  continues to grow as long
as there is a net  flux of negative charge into the wall.
This process presumably comes to an end, and a steady-state is attained, when the potential barrier becomes sufficiently large to make electron flux  equal to the ion flux.

Let us construct a one-dimensional model of an unmagnetized, steady-state,   Langmuir sheath. 
Suppose that the wall lies at $x=0$, and that the plasma occupies the
region $x>0$. Let us treat the ions and the electrons inside the sheath
as {\em collisionless fluids}. 
The ion and electron equations of
motion are thus written
\begin{eqnarray}
m_i\,n_i\,V_i\,\frac{dV_i}{dx} &=& - T_i\,\frac{dn_i}{dx} - e\,n_i\,\frac{d\phi}{dx},\label{e3.261}\\[0.5ex]
m_e\,n_e\,V_e\,\frac{dV_e}{dx} &=& - T_e\,\frac{dn_e}{dx} + e\,n_e\,\frac{d\phi}{dx},\label{e3.262}
\end{eqnarray}
respectively. Here, $\phi(x)$ is the electrostatic potential. Moreover, we have
assumed {\em uniform}\/ ion and electron temperatures, $T_i$ and $T_e$,
respectively, for the sake of simplicity. We have also neglected any off-diagonal
terms in the ion and electron stress-tensors, since these terms are
comparatively small. 
 Note that quasi-neutrality
does not apply inside the sheath, and so the ion and
electron number densities, $n_e$ and $n_i$, respectively, are not
necessarily equal to one another.

Consider the ion fluid. Let us assume that the mean ion velocity, $V_i$,
is {\em much greater}\/ than the ion thermal velocity, $(T_i/m_i)^{1/2}$. Since,
as will become apparent, $V_i\sim (T_e/m_i)^{1/2}$, this ordering necessarily  implies
that $T_i\ll T_i$: {\em i.e.}, that the ions are
{\em cold}\/ with respect to the electrons. It turns out that plasmas in the
immediate vicinity of solid walls often  have comparatively cold ions, so our ordering assumption
is fairly reasonable.  In the cold ion limit, the pressure term in Eq.~(\ref{e3.261})
is negligible, and the  equation can be integrated to
give
\begin{equation}
\frac{1}{2}\,m_i\,V_i^{\,2}(x) + e\,\phi(x) = \frac{1}{2}\,m_i\,V_s^{\,2} + e\,\phi_s.
\end{equation}
Here, $V_s$ and $\phi_s$ are the mean ion velocity and electrostatic
potential, respectively, at the edge of the sheath ({\em i.e.}, $x\rightarrow\infty$). 
Now,  ion fluid continuity requires that
\begin{equation}
n_i(x)\,V_i(x) = n_s\,V_s,
\end{equation}
where $n_s$ is the ion number density at the  sheath boundary. Incidentally, since we
expect quasi-neutrality to hold in the plasma outside the sheath, the electron number density
at the edge of the sheath must also be $n_s$ (assuming singly charged
ions). The previous two equations can be combined to give
\begin{eqnarray}
V_i &=& V_s\left[1- \frac{2\,e}{m_i\,V_s^{\,2}}\,(\phi-\phi_s)\right]^{1/2},\\[0.5ex]
n_i&=& n_s\left[1- \frac{2\,e}{m_i\,V_s^{\,2}}\,(\phi-\phi_s)\right]^{-1/2}.
\end{eqnarray}

Consider the electron fluid. Let us assume that the mean electron
velocity, $V_e$, is {\em much less}\/ than the electron thermal
velocity, $(m_e/T_e)^{1/2}$. In fact, this must be the case, otherwise,
the electron flux to the wall would greatly exceed the ion flux. 
Now, if the electron fluid is essentially stationary then the left-hand side of
Eq.~(\ref{e3.262}) is negligible, and the equation can be integrated to
give
\begin{equation}\label{e3.267}
n_e = n_s\,\exp\left[\frac{e\,(\phi-\phi_s)}{T_e}\right].
\end{equation}
Here, we have made use of the  fact  that $n_e=n_s$ at
the edge of the sheath. 

Now, Poisson's equation is written
\begin{equation}
\epsilon_0\,\frac{d^2\phi}{dx^2} = e\,(n_e-n_i).
\end{equation}
It follows that
\begin{equation}\label{e3.269}
\epsilon_0\,\frac{d^2\phi}{dx^2}= e\,n_s\left(\exp\left[\frac{e\,(\phi-\phi_s)}{T_e}\right]-\left[1- \frac{2\,e}{m_i\,V_s^{\,2}}\,(\phi-\phi_s)\right]^{-1/2}\right).
\end{equation}
Let $\Phi=-e\,(\phi-\phi_s)/T_e$, $y = \sqrt{2}\,x/\lambda_D$, and
\begin{equation}
K =\frac{m_i\,V_s^{\,2}}{2\,T_e},
\end{equation}
 where $\lambda_D= (\epsilon_0\,T_e/e^2\,n_s)^{1/2}$ is the Debye length. Equation~(\ref{e3.269}) transforms to
\begin{equation}
2\,\frac{d^2 \Phi}{dy^2} = - {\rm e}^{-\Phi}+ \left(1+\frac{\Phi}{K}\right)^{-1/2},
\end{equation}
subject to the boundary condition  $\Phi\rightarrow 0$ as $y\rightarrow \infty$. 
Multiplying through by $d\Phi/dy$, integrating with respect to $y$, and making use of the
boundary condition, we obtain
\begin{equation}\label{e3.271}
\left(\frac{d\Phi}{dy}\right)^2 = {\rm e}^{-\Phi}-1+ 2\,K\left[
\left(1+\frac{\Phi}{K}\right)^{1/2}-1\right].
\end{equation}
Unfortunately, the above equation is highly nonlinear, and can only be solved numerically. 
However, it is not necessary to attempt this to see that 
 a physical solution can only exist
if the right-hand side of the equation is {\em positive}\/ for all  $y\geq 0$. Consider the  the
limit  $y\rightarrow\infty$. It follows from the boundary condition that $\Phi\rightarrow 0$. Expanding
the right-hand side of Eq.~(\ref{e3.271}) in powers of $\Phi$, we find that the zeroth- and first-order terms cancel, and  we are left with
\begin{equation}
\left(\frac{d\Phi}{dy}\right)^2 \simeq \frac{\Phi^2}{2}\left(1- \frac{1}{2\,K}\right) + \frac{\Phi^3}{3}\left(\frac{3}{8\,K^2}-1\right) + O(\Phi^4).
\end{equation}
Now, the purpose of the sheath is to {\em shield}\/ the plasma from the wall
potential. It can be seen, from the above expression, that the physical
solution with
maximum possible shielding corresponds to $K=1/2$, since this
choice eliminates the first term on the right-hand side (thereby making
$\Phi$ as small as possible at large $y$) leaving the much smaller,
but positive (note that $\Phi$ is positive), second term.
Hence, we conclude that
\begin{equation}\label{e3.273}
V_s = \left(\frac{T_e}{m_i}\right)^{1/2}.
\end{equation}
This result is known as the {\em Bohm sheath criterion}.
It is a somewhat surprising result, since it indicates that
ions at the edge of the sheath are already moving toward the wall at a
considerable velocity. Of course, the ions are further accelerated 
as they pass through the sheath. Since the ions are presumably at
rest in the interior of the plasma, it is clear that there
must exist a region sandwiched between the sheath and the main plasma
in which the ions are accelerated from rest to the Bohm velocity,
$V_s=(T_e/m_i)^{1/2}$. This region is called the {\em pre-sheath}, 
and  is both quasi-neutral and much wider than the sheath (the actual
width depends on the nature of the ion source).

The ion current density at the wall is
\begin{equation}
j_i = -e\,n_i(0)\,V_i(0) = -e\,n_s\,V_s = - e\,n_s\left(\frac{T_e}{m_i}\right)^{1/2}.
\end{equation}
This current density is negative because the ions are moving in the negative $x$-direction. What about the electron current density? Well, the
number density of electrons at the wall is $n_e(0) = n_s\,\exp[\,e\,(\phi_w-\phi_s)/T_e)]$, where $\phi_w=\phi(0)$ is the wall potential. Let us assume
that the electrons have a Maxwellian velocity distribution peaked at
zero velocity (since the electron fluid velocity is much less than the
electron thermal velocity). It follows that half of the electrons at $x=0$ are moving in
the negative-$x$ direction, and half in the positive-$x$ direction. 
Of course, the former electrons hit the wall, and thereby constitute an electron
current to the wall. This current is $j_e=(1/4)\,e\,n_e(0)\,\bar{V}_e$, where
the $1/4$ comes from averaging over solid angle, and $\bar{V}_e=(8\,T_e/\pi\,m_e)^{1/2}$ is the mean electron speed corresponding to a
Maxwellian velocity distribution. Thus, the electron current density at the wall
is
\begin{equation}
j_e= e\,n_s\,\left(\frac{T_e}{2\,\pi\,m_e}\right)^{1/2} \exp\left[\frac{e\,(\phi_w-\phi_s)}{T_e}\right].
\end{equation}
Now, in order to replace the electrons lost to the wall, the electrons must have a mean velocity
\begin{equation}
V_{e\,s} = \frac{j_e}{e\,n_s}= \left(\frac{T_e}{2\pi\,m_e}\right)^{1/2} \exp\left[\frac{e\,(\phi_w-\phi_s)}{T_e}\right]
\end{equation}
at the edge of the sheath. However, we previously assumed that any
electron fluid velocity was much less than the electron thermal
velocity, $(T_e/m_e)^{1/2}$. As is clear from the above equation,
this is only possible provided that
\begin{equation}
\exp\left[\frac{e\,(\phi_w-\phi_s)}{T_e}\right]\ll 1.
\end{equation}
{\em i.e.}, provided that the wall potential is sufficiently negative to
strongly reduce the electron number density at the wall.
The net current density at the wall is
\begin{equation}\label{e3.279}
j = e\,n_s\,\left(\frac{T_e}{m_i}\right)^{1/2}\left\{\left(\frac{m_i}{2\pi\,m_e}\right)^{1/2}\exp\left[\frac{e\,(\phi_w-\phi_s)}{T_e}\right]-1\right\}.
\end{equation}
Of course, we require $j=0$ in a steady-state sheath, in order to prevent wall charging, and so we obtain
\begin{equation}
e\,(\phi_w-\phi_s)= -T_e\,\ln\left(\frac{m_i}{2\pi\,m_e}\right)^{1/2}.
\end{equation}
We conclude that, in a steady-state sheath, the wall is biased {\em negatively}\/ with respect to  the
sheath edge by an amount which is proportional to the electron temperature.

For a hydrogen plasma, $\ln(m_i/2\pi\,m_e)\simeq 2.8$. Thus, hydrogen
ions enter the sheath with an initial energy $(1/2)\,m_i\,V_s^{\,2} = 0.5\,T_e\,{\rm eV}$, fall through the sheath potential, and so impact the
wall with energy $3.3\,T_e\,{\rm eV}$. 

A {\em Langmuir probe}\/ is a device  used to determine the electron temperature and electron number density of a plasma. It works by inserting an electrode  which is biased with respect to the vacuum vessel into the plasma. Provided that the bias voltage is not too positive, we would
expect the probe current to vary as
\begin{equation}\label{e3.280}
I = A\,e\,n_s\,\left(\frac{T_e}{m_i}\right)^{1/2}\left[\left(\frac{m_i}{2\pi\,m_e}\right)^{1/2}\exp\left(\frac{e\,V}{T_e}\right)-1\right],
\end{equation}
where $A$ is the surface area of the probe, and $V$ its bias with respect to the vacuum vessel---see Eq.~(\ref{e3.279}). For strongly negative
biases, the probe current saturates in the ion (negative) direction. The
characteristic current which flows in this situation is called the {\em ion
saturation current}, and is of magnitude
\begin{equation}\label{e3.281}
I_s = A\,e\,n_s\,\left(\frac{T_e}{m_i}\right)^{1/2}.
\end{equation}
For less negative biases, the current-voltage relation of the probe
has the general form
\begin{equation}
\ln I = C + \frac{e\,V}{T_e},
\end{equation}
where $C$ is a constant. Thus, a plot of $\ln I$ versus $V$  gives
a {\em straight-line}\/  from whose slope the electron temperature can be deduced. Note, however, that if the bias voltage becomes too positive then
electrons cease to be effectively repelled from the probe surface, and the
current-voltage relation (\ref{e3.280}) breaks down.
Given the electron temperature, a measurement of the ion saturation current  allows the electron
number density at the sheath edge, $n_s$, to be calculated from Eq.~(\ref{e3.281}). Now, in order to accelerate ions to the Bohm velocity, the
potential drop across the pre-sheath needs to be $e\,(\phi_p-\phi_s) =- T_e/2$,
where $\phi_p$ is the electric potential in the interior of the plasma. It follows
from Eq.~(\ref{e3.267}) that the relationship between the electron number
density at the sheath boundary, $n_s$, and the number density   in the interior of the plasma, $n_p$,  is
\begin{equation}
n_s = n_p\,{\rm e}^{-0.5} \simeq 0.61\,n_p.
\end{equation}
Thus, $n_p$ can also be determined from the probe.