\chapter{Waves in Warm Plasmas}
\section{Introduction}
In this section we shall investigate  wave propagation in a warm collisionless
plasma, extending the discussion given in Sect.~\ref{s4} to take 
thermal effects into account. It turns out that  thermal modifications to
wave
propagation are not very well
described by fluid equations. We shall, therefore, adopt  a {\em kinetic}\/ 
description of the plasma. The appropriate kinetic equation is, of course,
the {\em Vlasov equation}, which is described in Sect.~\ref{s3.1}.

\section{Landau Damping}\label{s6.2}
Let us begin our study of the Vlasov equation by
examining what appears, at first sight, to be a fairly simple and
straight-forward problem. Namely, the propagation of
small amplitude plasma waves through  a uniform plasma with no
equilibrium magnetic field. For the sake of simplicity, we shall only consider
 electron motion, assuming that the ions form an immobile,
neutralizing background. The ions are also  assumed to
be singly-charged.
We shall look for  electrostatic
plasma waves of the type discussed in Sect.~\ref{s4.7}. Such waves 
are longitudinal in nature, and possess
a perturbed electric field, but no perturbed magnetic field.

Our starting point is the Vlasov equation for an unmagnetized, collisionless
plasma:
\begin{equation}\label{e6.1}
\frac{\partial f_e}{\partial t} + {\bf v}\!\cdot\!\nabla f_e - \frac{e}{m_e}\,
{\bf E}\!\cdot\!\nabla_v f_e = 0,
\end{equation}
where $f_e({\bf r},{\bf v}, t)$ is the ensemble averaged electron distribution
function. The electric field satisfies
\begin{equation}\label{e6.2}
{\bf E} = -\nabla\phi.
\end{equation}
where
\begin{equation}\label{e6.3}
\nabla^2\phi = -\frac{e}{\epsilon_0}\left(n-\int \!f_e\,d^3{\bf v}\right).
\end{equation}
Here, $n$ is the number density of ions (which is the same
as the number density of electrons). 

Since we are dealing with small amplitude waves, it is appropriate to
{\em linearize}\/  the Vlasov equation. Suppose that the electron
distribution function is written
\begin{equation}
f_e({\bf r}, {\bf v}, t) = f_0({\bf v}) + f_1({\bf r}, {\bf v}, t).
\end{equation}
Here, $f_0$ represents the equilibrium electron distribution, whereas $f_1$
represents the small perturbation due to the wave. Note that $\int f_0\,d^3{\bf v}
=n$, otherwise the equilibrium state is not quasi-neutral. The electric
field is assumed to be zero in the unperturbed state, so that ${\bf E}$
can be regarded as a small quantity. Thus, linearization of
Eqs.~(\ref{e6.1}) and (\ref{e6.3}) yields
\begin{equation}\label{e6.5}
\frac{\partial f_1}{\partial t} + {\bf v}\!\cdot\!\nabla f_1
- \frac{e}{m_e}\,{\bf E}\!\cdot\!\nabla_v f_0 =0,
\end{equation}
and
\begin{equation}\label{e6.6}
\nabla^2\phi = \frac{e}{\epsilon_0}\int\! f_1\,d^3{\bf v},
\end{equation}
respectively.

Let us now follow the standard procedure for analyzing small amplitude
waves, by assuming that all perturbed quantities vary with
${\bf r}$ and $t$ like $\exp[\,{\rm i}\,({\bf k}\!\cdot\!{\bf r} - \omega\,t)]$.
Equations (\ref{e6.5}) and (\ref{e6.6}) reduce to
\begin{equation}
-{\rm i}\,(\omega - {\bf k}\!\cdot\!{\bf v}) f_1
+ {\rm i}\,\frac{e}{m_e}\,\phi \,{\bf k}\!\cdot\!\nabla_v f_0 = 0,
\end{equation}
and
\begin{equation}
-k^2\,\phi = \frac{e}{\epsilon_0} \int\! f_1\,d^3{\bf v},
\end{equation}
respectively. Solving the first of these equations for $f_1$, and substituting 
into the integral in the second, we conclude that if $\phi$ is non-zero
then we must have
\begin{equation}\label{e6.9}
1 + \frac{e^2}{\epsilon_0\,m_e\,k^2} \int\frac{ {\bf k}\!\cdot\!\nabla_v f_0}
{\omega - {\bf k}\!\cdot\!{\bf v}}\,d^3{\bf v} = 0.
\end{equation}

We can interpret Eq.~(\ref{e6.9}) as the dispersion relation for electrostatic plasma
waves, relating the wave-vector,  ${\bf k}$, to the frequency, $\omega$. 
However, in doing so, we run up against a serious problem, since the integral has
a {\em singularity}\/  in velocity space, where $\omega={\bf k}\!\cdot\!{\bf v}$,
and is, therefore, not properly defined.

The way around this problem was first pointed out by Landau\footnote{L.D.~Landau,
Sov.\ Phys.--JETP {\bf 10}, 25 (1946).} in a very
influential paper which laid the basis of much subsequent
research on plasma oscillations and instabilities. Landau showed that,
instead of simply assuming that $f_1$ varies in time as $\exp(-{\rm i}\,\omega\,t)$,
the problem must be regarded as an initial value problem in which $f_1$
is given at $t=0$ and found at later times. 
We may still Fourier analyze with respect to ${\bf r}$, so we write
\begin{equation}
f_1({\bf r}, {\bf v}, t) = f_1({\bf v},t)\,{\rm e}^{\,{\rm i}\,{\bf k}\cdot
{\bf r}}.
\end{equation}
It is helpful to define $u$ as the velocity component along ${\bf k}$ 
({\em i.e.}, $u= {\bf k}\!\cdot\!{\bf v}/k$) \, and to define 
$F_0(u)$ and $F_1(u,t)$ to be the integrals of $f_0({\bf v})$ and
$f_1({\bf v},t)$ over the velocity components perpendicular to ${\bf k}$. 
Thus, we obtain
\begin{equation}\label{e6.11}
\frac{\partial F_1}{\partial t} + {\rm i}\,k\,u\,F_1 - \frac{e}{m_e}\,E\,
\frac{\partial F_0}{\partial u} = 0,
\end{equation}
and
\begin{equation}\label{e6.12}
{\rm i}\,k\,E =-\frac{e}{\epsilon_0} \int_{-\infty}^{\infty} F_1(u)\,du.
\end{equation}

In order to solve Eqs.~(\ref{e6.11}) and (\ref{e6.12}) as an initial value problem, we
introduce the Laplace transform of $F_1$ with respect to $t$:
\begin{equation}
\bar{F}_1(u,p) = \int_0^\infty F_1(u,t)\,{\rm e}^{-p\,t}\,dt.
\end{equation}
If the growth of $F_1$ with $t$ is no faster than exponential then the above
integral converges and defines $\bar{F}_1$ as an analytic function of $p$, provided 
that the real part of $p$ is sufficiently large. 

Noting that the Laplace transform of $\partial F_1/\partial t$ is $p\,\bar{F}_1-F_1(u,t=0)$
(as is easily shown by integration by parts), we can Laplace transform Eqs.~(\ref{e6.11})
and (\ref{e6.12}) to obtain
\begin{equation}
p\,\bar{F}_1 + {\rm i}\,k\,u\,\bar{F}_1 = \frac{e}{m_e}\,\bar{E}\,\frac{\partial
F_0}{\partial u} + F_1(u,t=0),
\end{equation}
and
\begin{equation}
{\rm i}\,k\,\bar{E} =-\frac{e}{\epsilon_0}\int_{-\infty}^{\infty}\!\bar{F}_1(u)\,
du,
\end{equation}
respectively.
The above two equations can be combined to give
\begin{equation}
{\rm i}\,k\,\bar{E} = -\frac{e}{\epsilon_0}\int_{-\infty}^{\infty}
\left[ \frac{e}{m_e}\,\bar{E}\,\frac{\partial F_0/\partial u}
{p + {\rm i}\,k\,u} + \frac{F_1(u,t=0)}{p+{\rm i}\,k\,u}\right] du,
\end{equation}
yielding
\begin{equation}\label{e6.17}
\bar{E} = -\frac{(e/\epsilon_0)}{{\rm i}\,k\,\epsilon(k,p)}
\int_{-\infty}^\infty \frac{F_1(u,t=0)}{p+{\rm i}\,k\,u}\,du,
\end{equation}
where
\begin{equation}
\epsilon(k,p) = 1 + \frac{e^2}{\epsilon_0\,m_e\,k}
\int_{-\infty}^{\infty} \frac{\partial F_0/\partial u}{{\rm i}\,p-
k\,u}\,du.
\end{equation}
The function $\epsilon(k,p)$ is known as the {\em plasma dielectric
function}. Note that if $p$ is replaced by $-{\rm i}\,\omega$ then
the dielectric function becomes equivalent to the left-hand side
of Eq.~(\ref{e6.9}). However, since $p$ possesses a positive real part, the above
integral is well defined.

The Laplace transform of the distribution function is written
\begin{equation}
\bar{F}_1 = \frac{e}{m_e}\,\bar{E}\, \frac{\partial F_0/\partial u}
{p+ {\rm i}\,k\,u} + \frac{F_1(u,t=0)}{p+ {\rm i}\,k\,u},
\end{equation}
or
\begin{equation}\label{e6.20}
\bar{F}_1 = - \frac{e^2}{\epsilon_0\,m_e\,{\rm i}\,k} \frac{\partial F_0/\partial u}
{\epsilon(k, p)\,(p + {\rm i}\,k\,u)}
\int_{-\infty}^\infty \frac{F_1(u',t=0)}
{p+ {\rm i}\,k\,u'}\,du' + \frac{F_1(u,t=0)}{p+ {\rm i}\,k\,u}.
\end{equation}

Having found the Laplace transforms of the electric field and the perturbed
distribution function, we must now invert them to obtain
 $E$ and $F_1$  as functions of time. The inverse Laplace transform
of the distribution function is given by
\begin{equation}\label{e6.21}
F_1(u,t) = \frac{1}{2\pi\,{\rm i}}\int_C \bar{F}_1(u,p)\,{\rm e}^{\,p\,t}\,dp,
\end{equation}
where $C$, the so-called {\em Bromwich contour}, is a contour running parallel to
the imaginary axis, and lying to the right of all singularities of $\bar{F}_1$
in the complex-$p$ plane (see Fig.~\ref{f30}). There is an analogous
expression for the parallel electric field, $E(t)$. 

\begin{figure}
\epsfysize=3in
\centerline{\epsffile{Chapter06/brom.eps}}
\caption{\em The Bromwich contour.}\label{f30}
\end{figure}

Rather than trying to obtain a general expression for $F_1(u,t)$, from
Eqs.~(\ref{e6.20}) and (\ref{e6.21}), we shall concentrate on the behaviour of the
perturbed distribution function at {\em large}\/ times. Looking at
Fig.~\ref{f30}, we note that if $\bar{F}_1(u,p)$ has only a finite
number of simple poles in the region ${\rm Re}(p)>-\sigma$, then
we may deform the contour as shown in Fig.~\ref{f31}, with a loop around
each of the singularities. A pole at $p_0$ gives a contribution
going as ${\rm e}^{\,p_0\,t}$, whilst the vertical part of the
contour goes as ${\rm e}^{-\sigma\,t}$. For sufficiently long times
this latter contribution is negligible, and the behaviour is
dominated by contributions from the poles furthest to the right.

\begin{figure}
\epsfysize=3in
\centerline{\epsffile{Chapter06/brom1.eps}}
\caption{\em The distorted Bromwich contour.}\label{f31}
\end{figure}

Equations~(\ref{e6.17})--(\ref{e6.20}) all involve integrals of the form
\begin{equation}\label{e6.22}
\int_{-\infty}^\infty \frac{G(u)}{u-{\rm i}\,p/k}\,du,
\end{equation}
which become singular as $p$ approaches the imaginary axis. In order to
distort the contour $C$, in the manner shown in Fig.~31, we need to continue
these integrals smoothly across the imaginary $p$-axis. By virtue of the
way in which the Laplace transform was originally defined, for ${\rm Re}(p)$
sufficiently large, the appropriate way to do this is to take the values
of these integrals when $p$ is in the right-hand half-plane, and find the
analytic continuation into the left-hand half-plane.

If $G(u)$ is sufficiently well-behaved that it can be continued off the
real axis as an analytic function of a complex variable $u$ then the
continuation of (\ref{e6.22}) as the singularity crosses the real axis
in the complex $u$-plane, from the upper to the lower half-plane, is obtained
by letting the singularity take the contour with it, as shown
in Fig.~\ref{f32}.

\begin{figure}
\epsfysize=2.5in
\centerline{\epsffile{Chapter06/brom3.eps}}
\caption{\em The Bromwich contour for Landau damping.}\label{f32}
\end{figure}

Note that the ability to deform the contour $C$ into that of Fig.~\ref{f31}, and find
a dominant contribution to $E(t)$ and
$F_1(u,t)$  from a few poles, depends on $F_0(u)$ and $F_1(u,t=0)$
having  {\em smooth}\/ enough
 velocity dependences that the integrals appearing in
Eqs.~(\ref{e6.17})--(\ref{e6.20}) can be continued sufficiently far into the left-hand
half of the complex $p$-plane.

If we consider the electric field given by the inversion of Eq.~(\ref{e6.17}),
we see that its behaviour at large times is dominated by the zero of $\epsilon(k, p)$
which lies furthest to the right in the complex $p$-plane.
According to Eqs.~(\ref{e6.20}) and (\ref{e6.21}), $F_1$
has a similar contribution, as well as a contribution going as ${\rm e}^{-{\rm i}\,k\,u\,t}$. Thus, for sufficiently long times after the initiation of
the wave, the electric field depends only on the positions of the
roots of $\epsilon(k,p)=0$ in the complex $p$-plane. The distribution function
has a corresponding contribution
from the poles, as well as a component going as ${\rm e}^{-{\rm i}\,k\,u\,t}$. 
For large times, the latter component of the distribution function is
a rapidly oscillating function of velocity, and its contribution to the
charge density, obtained by integrating over $u$, is negligible. 

As we have already noted, the function $\epsilon(k, p)$ is equivalent to the
left-hand side of Eq.~(\ref{e6.9}), provided that $p$ is replaced by $-{\rm i}\,\omega$.
Thus, the dispersion relation, (\ref{e6.9}), obtained via Fourier transformation of the
Vlasov equation,
gives the correct behaviour at large times as long as the singular integral
is treated correctly. Adapting the procedure which we found using the
variable $p$, we see that the integral is defined as it is written for
${\rm Im}(\omega)>0$, and analytically continued, by deforming the
contour of integration in the $u$-plane (as shown in Fig.~\ref{f32}), into the region
${\rm Im}(\omega)<0$. The simplest way to remember how to do the
analytic continuation is to note that the integral is
continued from the part of the $\omega$-plane corresponding to {\em growing}\/
perturbations, to that corresponding to {\em damped}\/ perturbations. Once we
know this rule, we can obtain kinetic dispersion relations in a fairly direct manner
via  Fourier
transformation of the Vlasov
equation, and there is no need to attempt the more complicated Laplace transform
solution.

In Sect.~\ref{s4}, where we investigated the cold-plasma dispersion relation, we found that
for any given $k$ there were a finite number of values of $\omega$, say $\omega_1$,
$\omega_2$, $\cdots$, and a general solution was a linear superposition of
functions varying in time as ${\rm e}^{-{\rm i}\,\omega_1\,t}$,  ${\rm e}^{-{\rm i}\,\omega_2\,t}$, {\em etc}. This set of values of $\omega$ is called the
{\em spectrum}, and the cold-plasma equations yield a {\em discrete}\/  spectrum.
On the other hand, in the kinetic problem we obtain contributions
to the distribution function going as ${\rm e}^{-{\rm i}\,k\,u\,t}$,
with $u$ taking any real value. 
All of the mathematical difficulties of the kinetic
problem arise from the existence of this {\em continuous spectrum}. At
short times, the behaviour is very complicated, and depends on the details
of the initial perturbation. It is only asymptotically that a mode
varying as ${\rm e}^{-{\rm i}\,\omega\,t}$ is obtained, with $\omega$ determined
by a dispersion relation which is solely a function of the unperturbed state.
As we have seen, the emergence of such a mode depends on the initial velocity
disturbance being sufficiently smooth.

Suppose, for the sake of simplicity, that the background plasma state is a
Maxwellian distribution. Working in terms of $\omega$, rather than $p$, the kinetic dispersion
relation for electrostatic waves takes the form
\begin{equation}\label{e6.23}
\epsilon(k,\omega) =  1 + \frac{e^2}{\epsilon_0\,m_e\,k}
\int_{-\infty}^{\infty} \frac{\partial F_0/\partial u}{\omega-
k\,u}\,du=0,
\end{equation}
where
\begin{equation}
F_0(u) = \frac{n}{(2\pi\,T_e/m_e)^{1/2}}\,\exp(-m_e\,u^2/2\,T_e).
\end{equation}
Suppose that, to a first approximation, $\omega$ is real. Letting $\omega$
tend to the real axis from the domain ${\rm Im}(\omega)>0$, we obtain
\begin{equation}
\int_{-\infty}^{\infty} \frac{\partial F_0/\partial u}{\omega-
k\,u}\,du= P\!\int_{-\infty}^{\infty}\frac{\partial F_0/\partial u}{\omega-
k\,u}\,du - \frac{{\rm i}\,\pi}{k} \left(\frac{\partial F_0}{\partial u}
\right)_{u=\omega/k},
\end{equation}
where $P$ denotes the {\em principal part}\/ of the integral. The origin
of the two terms on the right-hand side of the above equation is illustrated
in Fig.~\ref{f33}. The first term---the principal part---is obtained by removing an
interval of length $2\,\epsilon$, symmetrical about the pole, $u=\omega/k$,
from the range of integration, and then letting $\epsilon\rightarrow 0$. The
second term comes from the small semi-circle linking the two halves of the
principal part integral. Note that the semi-circle deviates {\em below}\/ the
real $u$-axis, rather than above, because the integral is calculated by
letting the pole approach the axis from the {\em  upper}\/ half-plane 
in $u$-space. 

\begin{figure}
\epsfysize=2in
\centerline{\epsffile{Chapter06/prin.eps}}
\caption{\em Integration path about a pole.}\label{f33}
\end{figure}

Suppose that $k$ is sufficiently small that $\omega\gg k\,u$ over the
range of $u$ where $\partial F_0/\partial u$ is non-negligible. It follows
that we can expand the denominator of the principal part integral in a
Taylor series:
\begin{equation}
\frac{1}{\omega-k\,u} \simeq \frac{1}{\omega}\left(1+ \frac{k\,u}{\omega}
+ \frac{k^2\,u^2}{\omega^2} + \frac{k^3 u^3}{\omega^3} + \cdots\right).
\end{equation}
Integrating the result term by term, and remembering that $\partial F_0/\partial u$
is an odd function, Eq.~(\ref{e6.23}) reduces to
\begin{equation}
1-\frac{\omega_p^{~2}}{\omega^2} - 3\,k^2\,\frac{T_e\,\omega_p^{~2}}{m_e\,\omega^4}
- \frac{e^2}{\epsilon_0\,m_e} \frac{{\rm i}\,\pi}{k^2} \left(\frac{\partial
F_0}{\partial u}\right)_{u=\omega/k} \simeq 0,
\end{equation}
where $\omega_p=\sqrt{n\,e^2/\epsilon_0\,m_e}$ is the electron plasma frequency.
Equating the real part of the above expression to zero yields
\begin{equation}\label{e6.28}
\omega^2 \simeq \omega_p^{~2}(1+ 3\,k^2\,\lambda_D^2),
\end{equation}
where $\lambda_D =\sqrt{T_e/m_e\,\omega_p^{~2}}$ is the Debye length, and it
is assumed that $k\,\lambda_D\ll 1$. We can regard the imaginary
part of $\omega$ as a small perturbation, and write $\omega=\omega_0+\delta\omega$,
where $\omega_0$ is the root of Eq.~(\ref{e6.28}). It follows
that
\begin{equation}
2\,\omega_0\,\delta\omega \simeq \omega_0^{~2}\, \frac{e^2}{\epsilon_0\,m_e} \frac{{\rm i}\,\pi}{k^2} \left(\frac{\partial
F_0}{\partial u}\right)_{u=\omega/k},
\end{equation}
and so
\begin{equation}\label{e6.30}
\delta\omega \simeq \frac{{\rm i}\,\pi}{2} \frac{e^2\,\omega_p}{\epsilon_0\,m_e\,k^2}
 \left(\frac{\partial
F_0}{\partial u}\right)_{u=\omega/k},
\end{equation}
giving
\begin{equation}\label{e6.31}
\delta\omega \simeq - \frac{{\rm i}}{2}\sqrt{\frac{\pi}{2}} \frac{\omega_p}
{(k\,\lambda_D)^3} \exp\left[-\frac{1}{2\,(k\,\lambda_D)^2}\right].
\end{equation}

If we compare the above results with those  for a cold-plasma, where
the dispersion relation for an electrostatic plasma wave was found to
be simply $\omega^2=\omega_p^{~2}$, we see, firstly,  that  $\omega$ now depends on $k$,
according to Eq.~(\ref{e6.28}), so that in a warm plasma the electrostatic plasma
wave is a {\em propagating}\/ mode, with a non-zero group velocity. Secondly,  we 
now have
an imaginary part to $\omega$, given by Eq.~(\ref{e6.31}), corresponding, since
it is negative, to the {\em damping}\/ of the wave in time. This damping is generally
known as {\em Landau damping}. If $k\,\lambda_D\ll 1$ ({\em i.e.},
if the
wave-length is much larger than the Debye length) then the imaginary part
of $\omega$ is small compared to the real part, and the wave is only
lightly damped. However, as the wave-length becomes comparable to the
Debye length, the imaginary part of $\omega$ becomes comparable to the
real part, and the damping becomes strong. 
Admittedly, the approximate solution given above
is not very accurate in the short wave-length case, but it is sufficient to indicate
the existence of very strong damping. 

There are {\em no}\/ dissipative effects included in the collisionless Vlasov equation.
Thus, it can easily be verified that if the particle velocities are
reversed at any time then the solution up to that point is simply reversed in
time. At first sight, this reversible behaviour does not seem to be
consistent with the fact that an initial perturbation dies out. However,
we should note that it is only the electric field which decays. The
distribution function contains an undamped term going as ${\rm e}^{-{\rm i}\,k\,
u\,t}$. Furthermore, the decay of the electric field depends on there being a
sufficiently smooth initial perturbation in velocity space. The presence
of the  ${\rm e}^{-{\rm i}\,k\,
u\,t}$ term means that as time advances the velocity space dependence of the
perturbation becomes more and more convoluted. It follows that if we
reverse the velocities after some time then we are not starting
with a smooth distribution. Under these circumstances, there is
no contradiction in the fact that under time reversal the electric field will
grow initially, until the smooth initial state is recreated, and subsequently
decay away. 

\section{Physics of Landau Damping}\label{s6.3}
We  have explained Landau damping in terms of mathematics.
Let us now consider the physical explanation for this effect. The motion of a
charged particle situated  in a one-dimensional electric field varying as $E_0\,\exp[\,{\rm i}\,
(k\,x-\omega\,t)]$ is determined by
\begin{equation}
\frac{d^2x}{dt^2} = \frac{e}{m}\,E_0\,{\rm e}^{\,{\rm i}\,
(k\,x-\omega\,t)}.
\end{equation}
Since we are dealing with a linearized theory in which the
perturbation due to the wave is small, it follows that if the particle
starts with velocity $u_0$ at position $x_0$ then we may substitute
$x_0+u_0\,t$ for $x$ in the electric field term. This is actually the position of
the particle on its unperturbed trajectory, starting at $x=x_0$ at $t=0$.
Thus, we obtain 
\begin{equation}
\frac{du}{dt} = \frac{e}{m}\,E_0\,{\rm e}^{\,{\rm i}\,(k\,x_0+k\,u_0\,t-\omega\,t)},
\end{equation}
which yields
\begin{equation}
u-u_0 = \frac{e}{m}\,E_0\,\left[\frac{{\rm e}^{\,{\rm i}\,(k\,x_0+k\,u_0\,t-\omega\,t)}
- {\rm e}^{\,{\rm i}\,k\,x_0}}{{\rm i}\,(k\,u_0-\omega)}\right].
\end{equation}
As $k\,u_0-\omega\rightarrow 0$, the above expression reduces to
\begin{equation}
u -u_0 = \frac{e}{m}\,E_0\,t\,{\rm e}^{\,{\rm i}\,k\,x_0},
\end{equation}
showing that particles with $u_0$ close to $\omega/k$, that is with velocity
components along the $x$-axis close to the phase velocity of the wave, have
velocity perturbations which {\em grow}\/ in time. These so-called {\em resonant particles}\/ 
gain energy from, or lose energy to, the wave, and are responsible for the
damping. This explains why the damping rate, given by Eq.~(\ref{e6.30}), depends on the
slope of the distribution function calculated at $u=\omega/k$. The remainder of the
particles are non-resonant, and have an oscillatory response to the
wave field.

To understand why energy should be transferred from the electric
field to the resonant particles requires more detailed consideration. 
Whether the speed of a resonant particle increases or decreases depends on the
phase of the wave at its initial position, and it is not the case that
all particles moving slightly faster than the wave lose energy, whilst all
particles moving slightly slower than the wave gain energy. Furthermore, the
density perturbation is out of phase with the wave electric field, so there
is no initial wave generated excess of particles gaining or losing
energy. However, if we consider those particles which start off with velocities
slightly above the phase velocity of the wave then if they gain energy they
move away from the resonant velocity whilst if they lose energy they approach
the resonant velocity. The result is that the particles which lose energy interact more effectively with the wave, and, on average, there is a transfer of
energy from the particles to the electric field. Exactly the opposite
is true for particles with initial velocities lying just below the
phase velocity of the wave. In the case of a Maxwellian distribution
there are more particles in the latter class than in the former, so there is a net transfer
of energy from the electric field to the particles: {\em i.e.}, the 
electric field is damped. In the limit as the wave amplitude
tends to zero, it is clear that the gradient of the distribution function
at the wave speed is what determines the damping rate.

It is of some interest to consider the limitations of the above result, in terms
of the magnitude of the initial electric field above which it is seriously
in error and {\em nonlinear}\/ effects become important. The basic
requirement for the validity of the linear result is that a resonant
particle should maintain its position relative to the phase of the
electric field over a sufficiently long time for the damping to
take place. To obtain a condition that this be the case, let us consider the
problem in the frame of reference in which the wave is at rest, and the
potential $-e\,\phi$ seen by an electron is as sketched in Fig.~\ref{f34}.

\begin{figure}
\epsfysize=2.5in
\centerline{\epsffile{Chapter06/res1.eps}}
\caption{\em Wave-particle interaction.}\label{f34}
\end{figure}

If the electron starts at rest ({\em i.e.}, in resonance with the wave) at $x_0$ 
then it begins to move towards the potential minimum, as shown. The time for
the electron to shift its position relative to the wave may be estimated
as the period with which it bounces back and forth in the potential well.
Near the bottom of the well the equation of motion of the electron is written
\begin{equation}
\frac{d^2 x}{dt^2} = - \frac{e}{m_e}\,k^2\,x\,\phi_0,
\end{equation}
where $k$ is the wave-number, and so the bounce time is
\begin{equation}
\tau_b \sim 2\pi\,\sqrt{\frac{m_e}{e\,k^2\,\phi_0}} = 
2\pi\,\sqrt{\frac{m_e}{e\,k\,E_0}},
\end{equation}
where $E_0$ is the amplitude of the electric field. We may expect the wave
to damp according to linear theory if the bounce time, $\tau_b$, given
above, is much greater than the damping time. Since the former varies
inversely with the square root of the electric field amplitude, whereas the
latter is amplitude independent, this criterion gives us an estimate of
the maximum allowable initial perturbation which is consistent with linear
damping. 

If the initial amplitude is large enough for the resonant electrons to
bounce back and forth in the potential well a number of times before the
wave is damped, then it can be demonstrated that the result to be expected is
a {\em non-monotonic}\/ decrease in the amplitude of the electric field, as shown
in Fig.~\ref{f35}. The period of the amplitude oscillations is similar to the bounce
time, $\tau_b$. 

\begin{figure}
\epsfysize=2.5in
\centerline{\epsffile{Chapter06/res2.eps}}
\caption{\em Nonlinear Landau damping.}\label{f35}
\end{figure}

\section{Plasma Dispersion Function}
If the unperturbed distribution function, $F_0$, appearing in Eq.~(\ref{e6.23}), is
a Maxwellian then it is readily seen that, with a suitable scaling of the
variables, the dispersion relation for electrostatic
plasma waves can be expressed in terms of the
function
\begin{equation}
Z(\zeta) = \pi^{-1/2}\int_{-\infty}^{\infty} \frac{{\rm e}^{-t^2}}{t-\zeta}\,dt,
\end{equation}
which is defined as it is written for ${\rm Im}(\zeta)>0$, and is
analytically continued  for ${\rm Im}(\zeta)\leq 0$. This function is
known as the {\em plasma dispersion function}, and very often crops up
in problems involving small-amplitude waves propagating through 
warm plasmas. Incidentally, $Z(\zeta)$ is the Hilbert transform of a Gaussian.

In view of the importance of the plasma dispersion function, and its regular
occurrence in the literature of plasma physics, let us briefly examine its main
properties. We first of all note that if we differentiate $Z(\zeta)$ with
respect to $\zeta$ we obtain
\begin{equation}
Z'(\zeta) = \pi^{-1/2}\int_{-\infty}^{\infty} \frac{{\rm e}^{-t^2}}{(t-\zeta)^2}\,
dt,
\end{equation}
which yields, on integration by parts,
\begin{equation}\label{e6.40}
Z'(\zeta) =-\pi^{-1/2}\int_{-\infty}^{\infty} \frac{2\,t}{t-\zeta}\,
{\rm e}^{-t^2}\,dt = -2\,[1+\zeta\,Z].
\end{equation}

If we let $\zeta$ tend to zero from the upper half  of the complex plane, we get
\begin{equation}\label{e6.41}
Z(0) = \pi^{-1/2}\,{\rm P}\!\int_{-\infty}^{\infty} \frac{{\rm e}^{-t^2}}{t}\,dt
+ {\rm i}\,\pi^{1/2} = {\rm i}\,\pi^{1/2}.
\end{equation}
Note that the principle part integral is zero  because its integrand is an
odd function of $t$.

Integrating the linear differential equation (\ref{e6.40}), which possesses an
integrating factor ${\rm e}^{\zeta^2}$, and using the boundary condition
(\ref{e6.41}), we obtain an alternative expression for the plasma dispersion
function:
\begin{equation}
Z(\zeta) = {\rm e}^{-\zeta^2}\left({\rm i}\,\pi^{1/2}
-2\!\int_0^\zeta {\rm e}^{x^2}\,dx\right).
\end{equation}
Making the substitution $t={\rm i}\,x$ in the integral, and
noting that
\begin{equation}
\int_{-\infty}^0 {\rm e}^{-t^2}\,dt = \frac{\pi^{1/2}}{2},
\end{equation}
we finally arrive at the expression
\begin{equation}
Z(\zeta) = 2\,{\rm i}\, {\rm e}^{-\zeta^2}\int_{-\infty}^{{\rm i}\,\zeta}
{\rm e}^{-t^2}\,dt.
\end{equation}
This formula, which relates the plasma dispersion function to an
error function of imaginary argument, is valid for all values
of $\zeta$. 

For small $\zeta$ we have the expansion
\begin{equation}
Z(\zeta) = {\rm i}\,\pi^{1/2}\,{\rm e}^{-\zeta^2}-2\,\zeta\left[
1-\frac{2\,\zeta^2}{3} + \frac{4\,\zeta^4}{15} - \frac{8\,\zeta^6}{105}
+\cdots\right].
\end{equation}
For large $\zeta$, where $\zeta=x+{\rm i}\,y$, the asymptotic expansion
for $x>0$ is written
\begin{equation}\label{e6.46}
Z(\zeta) \sim {\rm i}\,\pi^{1/2}\,\sigma\,{\rm e}^{-\zeta^2}
-\zeta^{-1}\left[1+\frac{1}{2\,\zeta^2} + \frac{3}{4\,\zeta^4}+\frac{15}{8\,\zeta^6}
+\cdots\right].
\end{equation}
Here,
\begin{equation}
\sigma = \left\{
\begin{array}{lll}
0&\mbox{\hspace{1cm}}& y>1/|x|\\[0.5ex]
1&& |y|<1/|x|\\[0.5ex]
2&&y< -1/|x|
\end{array}
 \right..
\end{equation}
In deriving our expression for the Landau damping rate we have, in effect, used the
first few terms of the above asymptotic expansion.

The properties of the plasma dispersion function are specified in exhaustive
detail in a well-known book by Fried and Conte.\footnote{B.D.~Fried, and
S.D.~Conte, {\em The Plasma Dispersion Function} (Academic Press, New York NY,
1961.)}

\section{Ion Sound Waves}\label{s6.5}
If we now take ion dynamics into account then the dispersion relation (\ref{e6.23}),
for electrostatic plasma waves, generalizes to
\begin{equation}
1 + \frac{e^2}{\epsilon_0\,m_e\,k}\int_{-\infty}^{\infty}\frac{\partial F_{0\,e}/
\partial u}{\omega - k\,u}\,du + \frac{e^2}{\epsilon_0\,m_i\,k}
\int_{-\infty}^{\infty}\frac{\partial F_{0\,i}/\partial u}{\omega-k\,u}\,du = 0:
\end{equation}
{\em i.e.}, we simply add an extra term for the ions which has an
analogous form to the electron term. Let us search for a wave with
a phase velocity, $\omega/k$, which is much less than the electron thermal
velocity, but much greater than the ion thermal velocity. We may assume
that $\omega\gg k\,u$ for the ion term, as we did previously for the
electron term. It follows that, to
lowest order, this term reduces to $-\omega_{p\,i}^{~2}/\omega^2$. Conversely, we
may assume that $\omega\ll k\,u$ for the electron term. Thus, to
lowest order we may neglect $\omega$ in the velocity space
integral. Assuming $F_{0\,e}$
to be a Maxwellian with temperature $T_e$, the electron term reduces to
\begin{equation}
\frac{{\omega}_{p\,e}^{~2}}{k^2}\frac{m_e}{T_e} = \frac{1}{(k\,\lambda_D)^2}.
\end{equation}

Thus, to a first approximation, the dispersion relation can be written
\begin{equation}\label{e6.50}
1 + \frac{1}{(k\,\lambda_D)^2} - \frac{\omega_{p\,i}^{~2}}{\omega^2} = 0,
\end{equation}
giving
\begin{equation}
\omega^2 = \frac{\omega_{p\,i}^{~2}\,k^2\,\lambda_D^{~2}}{1+k^2\,\lambda_D^{~2}}
= \frac{T_e}{m_i} \frac{k^2}{1+k^2\,\lambda_D^{~2}}.
\end{equation}
For $k\,\lambda_D\ll 1$, we have $\omega=(T_e/m_i)^{1/2}\,k$, a dispersion relation
which is like that of an ordinary sound wave, with the pressure provided by the
electrons, and the inertia by the ions. As the wave-length is reduced towards the
Debye length, the frequency levels off and approaches the ion plasma
frequency.

Let us check our original assumptions. In the long wave-length limit, we see that
the wave phase velocity $(T_e/m_i)^{1/2}$ is indeed much less than the
electron thermal velocity [by a factor $(m_e/m_i)^{1/2}$], but that it
is only much greater than the ion thermal velocity if the ion temperature, $T_i$,
is much less than the electron temperature, $T_e$. In fact, if $T_i\ll T_e$
then the wave phase velocity can lie on almost flat portions of the
ion and electron distribution functions, as shown in Fig.~\ref{f36}, implying that
the wave is subject to
very little Landau damping. Indeed, an ion sound wave can only propagate a distance of order its wave-length
without being strongly damped  provided that $T_e$ is at least five to ten times greater than $T_i$.

Of course, it is possible to obtain the ion sound wave dispersion relation,
$\omega^2/k^2 = T_e/m_i$, using fluid theory. The kinetic treatment used here
is an improvement on the fluid theory to the extent that no equation of
state is assumed, and it makes it clear to us that ion sound waves are subject to
strong Landau damping ({\em i.e.}, they cannot be considered normal modes of the
plasma) unless $T_e\gg T_i$. 

\begin{figure}
\epsfysize=2.5in
\centerline{\epsffile{Chapter06/sound.eps}}
\caption{\em Ion and electron distribution functions with $T_i\ll T_e$.}\label{f36}
\end{figure}

\section{Waves in  Magnetized Plasmas}\label{s6.6}
Consider  waves propagating through a plasma placed in a uniform magnetic field, ${\bf B}_0$. 
Let us take the perturbed magnetic field into account in our
calculations, in order to allow for
electromagnetic, as well as electrostatic, waves. The linearized Vlasov equation
takes the form
\begin{equation}\label{e6.52}
\frac{\partial f_1}{\partial t} + {\bf v}\!\cdot\!\nabla f_1 + 
\frac{e}{m}\,({\bf v}\times{\bf B}_0)\!\cdot\!\nabla_v f_1 = -\frac{e}{m}\,
({\bf E} + {\bf v}\times{\bf B})\!\cdot\!\nabla_v f_0
\end{equation}
for both ions and electrons, where   ${\bf E}$ and
${\bf B}$ are the perturbed electric and magnetic fields, respectively. Likewise,
$f_1$ is the perturbed distribution function, and $f_0$  the equilibrium
distribution function.

In order to have an equilibrium state at all, we require that
\begin{equation}
({\bf v}\times{\bf B}_0)\!\cdot\!\nabla_v f_0 = 0.
\end{equation}
Writing the velocity, ${\bf v}$, in cylindrical polar coordinates,
$(v_\perp, \theta, v_z)$, aligned with the equilibrium magnetic
field, the above expression can easily be shown to imply that
$\partial f_0/\partial \theta =0$: {\em i.e.}, $f_0$ is a function
only of $v_\perp$ and $v_z$. 

Let the trajectory of a particle be ${\bf r}(t)$, ${\bf v}(t)$. In the
unperturbed state
\begin{eqnarray}
\frac{d{\bf r}}{dt} &=& {\bf v},\\[0.5ex]
\frac{d{\bf v}}{dt} &=& \frac{e}{m}\,({\bf v}\times{\bf B}_0).
\end{eqnarray}
It follows that Eq.~(\ref{e6.52}) can be written
\begin{equation}\label{e6.55}
\frac{D f_1}{Dt} = -\frac{e}{m}\,
({\bf E} + {\bf v}\times{\bf B})\!\cdot\!\nabla_v f_0,
\end{equation}
where $D f_1/Dt$ is the total rate of change of $f_1$, following the
unperturbed trajectories. Under the assumption that $f_1$ vanishes as
$t\rightarrow -\infty$, the solution to Eq.~(\ref{e6.55}) can be written
\begin{equation}\label{e6.56}
f_1({\bf r}, {\bf v}, t) =-\frac{e}{m}\int_{-\infty}^t
\left[{\bf E}({\bf r}', t') + {\bf v}'\times {\bf B}({\bf r}', t')\right]\!
\cdot\!\nabla_v f_0({\bf v}')\,dt',
\end{equation}
where $({\bf r}'$, ${\bf v}')$ is the unperturbed trajectory which passes
through the point $({\bf r}$, ${\bf v})$ when $t'=t$. 

It should be noted that the above method of solution is valid for any
set of equilibrium electromagnetic fields, not just a uniform magnetic
field. However, in a uniform magnetic field the unperturbed trajectories
are merely helices, whilst in a general field configuration it is difficult to
find a closed form for the particle trajectories which is sufficiently
simple to allow further progress to be made.

Let us write the velocity in terms of its Cartesian components:
\begin{equation}
{\bf v} =
(v_\perp\cos\theta, v_\perp\sin\theta, v_z).
\end{equation}
 It follows that
\begin{equation}\label{e6.58}
{\bf v}' = \left( v_\perp\cos\!\left[{\Omega}\,(t-t')+\theta\,\right],
 v_\perp\sin\!\left[{\Omega}\,(t-t')+\theta\,\right], v_z\right),
\end{equation}
where ${\Omega} = e\,B_0/m$ is the cyclotron frequency. The above
expression can be integrated to give
\begin{eqnarray}\label{e6.59a}
x'-x &=& -\frac{v_\perp}{{\Omega}}\,\left(\,\sin\!\left[{\Omega}\,(t-t')
+\theta\,\right] -\sin\theta\right),\\[0.5ex]\label{e6.59b}
y'-y &=& \frac{v_\perp}{{\Omega}}\,\left(\,\cos\!\left[{\Omega}\,(t-t')
+\theta\,\right] -\cos\theta\right),\\[0.5ex]\label{e6.59c}
z'-z &=& v_z\,(t'-t).
\end{eqnarray}
Note that both $v_\perp$ and $v_z$ are constants of the motion. This
implies that $f_0({\bf v}') = f_0({\bf v})$, because $f_0$ is only a
function of $v_\perp$ and $v_z$. Since $v_\perp = (v_x'^{~2} + v_y'^{~2})^{1/2}$,
we can write
\begin{eqnarray}\label{e6.60a}
\frac{\partial f_0}{\partial v_x'}& =& \frac{\partial v_\perp}{\partial v_x'}
\frac{\partial f_0}{\partial v_\perp} = \frac{v_x'}{v_\perp}\frac{\partial f_0}
{\partial v_\perp} = \cos\left[{\Omega}\,(t'-t)+\theta\,\right]
\frac{\partial f_0}{\partial v_\perp},\\[0.5ex]
\frac{\partial f_0}{\partial v_y'}& =& \frac{\partial v_\perp}{\partial v_y'}
\frac{\partial f_0}{\partial v_\perp} = \frac{v_y'}{v_\perp}\frac{\partial f_0}
{\partial v_\perp} = \sin\left[{\Omega}\,(t'-t)+\theta\,\right]
\frac{\partial f_0}{\partial v_\perp},\\[0.5ex]
\frac{\partial f_0}{\partial v_z'} &=& \frac{\partial f_0}{\partial v_z}.\label{e6.60c}
\end{eqnarray}

Let us assume an $\exp[\,{\rm i}\,({\bf k}\!\cdot\!{\bf r} -\omega\,t)]$
dependence of all perturbed quantities, with ${\bf k}$ lying in the $x$-$z$ plane.
Equation (\ref{e6.56}) yields
\begin{eqnarray}\label{e6.61}
f_1& =& -\frac{e}{m}\int_{-\infty}^t \left[ (E_x + v_y'\,B_z - v_z'\,B_y)\,
\frac{\partial f_0}{\partial v_{x}'} +(E_y + v_z'\,B_x - v_x'\,B_z)\,
\frac{\partial f_0}{\partial v_{y}'} \right.\nonumber\\[0.5ex]
&&\left.+(E_z + v_x'\,B_y - v_y'\,B_x)\,
\frac{\partial f_0}{\partial v_{z}'} \right]\exp\left[\,{\rm i}\,\{{\bf k}\!\cdot\!
({\bf r}' -{\bf r})-\omega\,(t'-t)\}\right]\,dt'.\nonumber\\[0.5ex]&&
\end{eqnarray}
Making use of Eqs.~(\ref{e6.58})--(\ref{e6.60c}), and the identity
\begin{equation}
{\rm e}^{\,{\rm i}\,a\,\sin x} \equiv \sum_{n=-\infty}^{\infty}
J_n(a)\,{\rm e}^{\,{\rm i}\,n\,x},
\end{equation}
Eq.~(\ref{e6.61}) gives 
\begin{eqnarray}
f_1& =& -\frac{e}{m}\int_{-\infty}^t \left[ (E_x - v_z\,B_y)\,\cos\chi\,\frac{\partial f_0}{\partial v_\perp} + (E_y + v_z\,B_x)\,\sin\chi\,\frac{\partial f_0}{\partial v_\perp}
\right.\nonumber
\\[0.5ex]
&&\left.+(E_z+ v_\perp\,B_y\,\cos\chi - v_\perp\,B_x\,\sin\chi)\,\frac{\partial f_0}{\partial v_z} \right]\,\sum_{n,m=-\infty}^\infty J_n\!\left(\frac{k_\perp\,v_\perp}
{{\Omega}}\right)J_m\!\left(\frac{k_\perp\,v_\perp}
{{\Omega}}\right)\nonumber
\\[0.5ex]
&&\times \exp\left\{\,{\rm i}\left[(n\,{\Omega} + k_z\,v_z-\omega)
\,(t'-t) + (m-n)\,\theta\,\right]\,\right\}\,dt',\label{e6.63}
\end{eqnarray}
where
\begin{equation}
\chi = {\Omega}\,(t-t') + \theta.
\end{equation}

Maxwell's equations yield
\begin{eqnarray}
{\bf k}\times {\bf E} &=& \omega\,{\bf B},\\[0.5ex]
{\bf k}\times{\bf B} &=& -{\rm i}\,\mu_0\,{\bf j} - \frac{\omega}{c^2}\,{\bf E}
=-\frac{\omega}{c^2}\,{\bf K}\!\cdot\!{\bf E},
\end{eqnarray}
where ${\bf j}$ is the perturbed current, and ${\bf K}$ is the dielectric
permittivity tensor introduced in Sect.~\ref{s4.2}. It follows that
\begin{equation}\label{e6.66}
{\bf K}\!\cdot\!{\bf E} ={\bf E} + \frac{{\rm i}}{\omega\,\epsilon_0}
\,{\bf j}= {\bf E} + \frac{{\rm i}}{\omega\,\epsilon_0}
\sum_s e_s\int {\bf v}\,f_{1\,s}\,d^3{\bf v},
\end{equation}
where $f_{1\,s}$ is the species-$s$ perturbed distribution function.

After a great deal of rather tedious analysis, Eqs.~(\ref{e6.63}) and (\ref{e6.66}) reduce to
the following expression for the dielectric permittivity tensor:
\begin{equation}\label{e6.67}
K_{ij} = \delta_{ij} + \sum_s \frac{e_s^{~2}}{\omega^2\,\epsilon_0\,m_s}
\sum_{n=-\infty}^\infty \int \frac{S_{ij}}{\omega - k_z \,v_z - n\,{\Omega}_s}
\,\,d^3{\bf v},
\end{equation}
where
\begin{equation}
S_{ij} = \left(
\begin{array}{ccc}
v_\perp\,(n\,J_n/a_s)^2\,U & {\rm i}\,v_\perp\,(n/a_s)\,J_n\,J_n' \,U&
v_\perp\,(n/a_s)\,J_n^{~2}\,U\\[0.5ex]
-{\rm i}\,v_\perp\,(n/a_s)\,J_n\,J_n'\,U& v_\perp\,J_n'^{~2}\,U & -{\rm i}\,v_\perp\,J_n\,J_n'\,W\\[0.5ex]
v_z\,(n/a_s)\,J_n^{~2}\,U & {\rm i}\,v_z\,J_n\,J_n'\,U & v_z\,J_n^{~2}\,W\end{array}
 \right),
\end{equation}
and
\begin{eqnarray}
U&=& (\omega-k_z\,v_z)\,\frac{\partial f_{0\,s}}{\partial v_\perp}
+ k_z\,v_\perp\,\frac{\partial f_{0\,s}}{\partial v_z},\\[0.5ex]
W&=& \frac{n\,{\Omega}_s\,v_z}{v_\perp} \,\frac{\partial f_{0\,s}}
{\partial v_\perp} + (\omega -n\,{\Omega}_s)\,\frac{\partial f_{0\,s}}
{\partial v_z},\\[0.5ex]
a_s &=& \frac{k_\perp\,v_\perp}{{\Omega}_s}.
\end{eqnarray}
The argument of the Bessel functions is $a_s$. In the above, $'$ denotes
differentiation with respect to argument.

The dielectric tensor (\ref{e6.67}) can be used to investigate the properties of waves
in just the same manner as the cold plasma dielectric tensor (\ref{e4.36x}) was used in 
Sect.~\ref{s4}. Note that our expression for the dielectric tensor involves
singular integrals of a type similar to those encountered in Sect.~\ref{s6.2}. In
principle, this means that we ought to treat the problem as an initial
value problem. Fortunately,  we can use the insights gained in our investigation of
the simpler unmagnetized electrostatic wave problem to recognize that the
appropriate way to treat the singular integrals is to evaluate them as
written for ${\rm Im}(\omega)>0$, and by analytic continuation
for ${\rm Im}(\omega)\leq 0$. 

For Maxwellian distribution functions, we can explicitly perform the velocity
space integral in Eq.~(\ref{e6.67}), making use of the identity
\begin{equation}
\int_0^\infty x\,J_n^{~2}(s\,x)\,{\rm e}^{-x^2}\,dx= \frac{{\rm e}^{-s^2/2}}{2}
\,I_n(s^2/2),
\end{equation}
where $I_n$ is a modified Bessel function. We obtain
\begin{equation}
K_{ij} = \delta_{ij} + \sum_s\frac{\omega_{p\,s}^{~2}}{\omega}
\left(\frac{m_s}{2\,T_s}\right)^{1/2} \frac{{\rm e}^{-\lambda_s}}{k_z}
\sum_{n=-\infty}^\infty T_{ij},
\end{equation}
where
\begin{equation}\label{e6.72}
T_{ij} = \left( \begin{array}{ccc}
{\scriptstyle n^2\,I_n\,Z/\lambda_s} & {\scriptstyle {\rm i}\,n\,(I_n'-I_n)\,Z} &
{\scriptstyle  -n\,I_n\,Z'/(2\,\lambda_s)^{1/2}}
\\[0.5ex]
{\scriptstyle -{\rm i}\,n\,(I_n'-I_n)\,Z} & {\scriptstyle (n^2\,I_n/\lambda_s +2\,\lambda_s\,I_n -2\,\lambda_s\,
I_n')\,Z} & {\scriptstyle {\rm i}\,\lambda_s^{1/2}\,(I_n'-I_n)\,Z'/2^{1/2}} \\[0.5ex]
{\scriptstyle -n\,I_n\,Z'/(2\,\lambda_s)^{1/2}} &{\scriptstyle
 -{\rm i}\,\lambda_s^{1/2}\,(I_n'-I_n)\,Z'/2^{1/2}}
&{\scriptstyle  -I_n\,Z'\,\xi_n}
\end{array}
\right).
\end{equation}
Here, $\lambda_s$, which is the argument of the Bessel functions, is written
\begin{equation}\label{e6.73}
\lambda_s = \frac{T_s\,k_\perp^{~2}}{m_s\,{\Omega}_s^{~2}},
\end{equation}
whilst $Z$ and $Z'$ represent the plasma dispersion function and its derivative,
both with argument
\begin{equation}\label{e6.74}
\xi_n = \frac{\omega - n\,{\Omega}_s}{k_z} \left(\frac{m_s}{2\,T_s}\right)^{1/2}.
\end{equation}

Let us consider the cold plasma limit, $T_s\rightarrow 0$. It follows from
Eqs.~(\ref{e6.73}) and (\ref{e6.74}) that this limit corresponds to $\lambda_s\rightarrow 0$ and
$\xi_n\rightarrow \infty$. From Eq.~(\ref{e6.46}), 
\begin{eqnarray}\label{e6.75a}
Z(\xi_n) &\rightarrow & -\frac{1}{\xi_n},\\[0.5ex]
Z'(\xi_n) &\rightarrow & \frac{1}{\xi_n^{~2}}\label{e6.75b}
\end{eqnarray}
as $\xi_n\rightarrow\infty$. Moreover,
\begin{equation}\label{e6.76}
I_n(\lambda_s) \rightarrow \left(\frac{\lambda_s}{2}\right)^{|n|}
\end{equation}
as  $\lambda_s\rightarrow 0$. It can be demonstrated  that the
only non-zero contributions to $K_{ij}$, in this limit, come from $n=0$ and
$n=\pm 1$. In fact, 
\begin{eqnarray}
K_{11} &=& K_{22} = 1-\frac{1}{2}\sum_s\frac{\omega_{p\,s}^{~2}}{\omega^2}
\left( \frac{\omega}{\omega-{\Omega}_s} + 
\frac{\omega}{\omega + {\Omega}_s}\right),\\[0.5ex]
K_{12} &=& -K_{21} = -\frac{{\rm i}}{2} \sum_s\frac{\omega_{p\,s}^{~2}}{\omega^2}
\left( \frac{\omega}{\omega-{\Omega}_s} -
\frac{\omega}{\omega + {\Omega}_s}\right),\\[0.5ex]
K_{33} &=& 1-\sum_s\frac{\omega_{p\,s}^{~2}}{\omega^2},
\end{eqnarray}
and $K_{13} = K_{31} = K_{23}=K_{32}=0$. It is easily seen, from Sect.~\ref{s4.3}, that the above
expressions are identical to those we obtained  using the cold-plasma fluid
equations. Thus, in the zero temperature limit, the kinetic
dispersion relation obtained in this section reverts to the fluid dispersion
relation obtained in Sect.~\ref{s4}. 

\section{Parallel Wave Propagation}
Let us consider wave propagation, though a warm
plasma,  {\em parallel}\/ to the equilibrium magnetic field. For parallel propagation,
$k_\perp\rightarrow 0$, and, hence, from Eq.~(\ref{e6.73}), $\lambda_s\rightarrow 0$. 
Making use of the asymptotic expansion (\ref{e6.76}), the matrix $T_{ij}$ 
simplifies to 
\begin{equation}
T_{ij} = \left( \begin{array}{ccc}
[Z(\xi_1)+Z(\xi_{-1})]/2 & {\rm i}\,[Z(\xi_1)
-Z(\xi_{-1})]/2 &
 0
\\[0.5ex]
 -{\rm i}\,[Z(\xi_1)-Z(\xi_{-1})]/2 & 
[Z(\xi_1)+Z(\xi_{-1})]/2 &  0 \\[0.5ex]
0 & 0
&-Z'(\xi_0)\,\xi_0
\end{array}
\right),
\end{equation}
where, again, the only non-zero contributions are from $n=0$ and $n=\pm 1$. 
The dispersion relation can be written [see Eq.~(\ref{e4.9})]
\begin{equation}\label{e6.79}
{\bf M}\!\cdot\! {\bf E} = {\bf 0},
\end{equation}
where
\begin{eqnarray}
M_{11}&=& M_{22} = 1-\frac{k_z^{~2}\,c^2}{\omega^2} \nonumber\\[0.5ex]
&&+\frac{1}{2}\sum_s 
\frac{\omega_{p\,s}^{~2}}
{\omega\,\,k_z v_s} \left[Z\!\left(\frac{\omega - {\Omega}_s}{k_z\,v_s}\right)
+ Z\!\left(\frac{\omega + {\Omega}_s}{k_z\,v_s}\right)\right],\\[0.5ex]
M_{12} &=&-M_{21} = \frac{\rm i}{2}\sum_s\frac{\omega_{p\,s}^{~2}}
{\omega\,k_z v_s} \left[Z\left(\frac{\omega - {\Omega}_s}{k_z\,v_s}\right)
- Z\left(\frac{\omega + {\Omega}_s}{k_z\,v_s}\right)\right],\\[0.5ex]
M_{33} &=& 1 - \sum_s \frac{\omega_{p\,s}^{~2}}
{(k_z\,v_s)^2} \,\,Z'\!\left(\frac{\omega}{k_z\,v_s}\right),
\end{eqnarray}
and $M_{13} = M_{31} = M_{23}=M_{32}=0$. Here, $v_s= \sqrt{2\,T_s/m_s}$ is
the species-$s$ thermal velocity.

The first root of Eq.~(\ref{e6.79}) is
\begin{equation}\label{e6.81}
1 +\sum_s\frac{2\,\omega_{p\,s}^{~2}}
{(k_z\,v_s)^2} \left[1+
\frac{\omega}{k_z\,v_s}\,Z\!\left(\frac{\omega}{k_z\,v_s}\right)\right] =0,
\end{equation}
with the eigenvector $(0,0,E_z)$. 
Here, use has been made of Eq.~(\ref{e6.40}).
This root evidentially corresponds to
a longitudinal, electrostatic plasma wave. In fact, it is easily
demonstrated that Eq.~(\ref{e6.81}) is equivalent to the dispersion relation
(\ref{e6.50}) that we found earlier for electrostatic
plasma waves, for the special case in which the distribution
functions are Maxwellians. Recall, from Sects.~6.3--6.5, that the
electrostatic wave described by the above expression is subject to
significant damping whenever the argument of the plasma dispersion
function becomes less than or comparable with unity: {\em i.e.}, whenever
$\omega\ltapp k_z\,v_s$. 

The second and third roots of Eq.~(\ref{e6.79}) are
\begin{equation}
\frac{k_z^{~2}\,c^2}{\omega^2} = 1 +\sum_s \frac{\omega_{p\,s}^{~2}}{\omega\,
\,k_z v_s}\,Z\!\left(\frac{\omega + {\Omega}_s}{k_z\,v_s}\right),
\end{equation}
with the eigenvector $(E_x, {\rm i}\,E_x, 0)$, and
\begin{equation}
\frac{k_z^{~2}\,c^2}{\omega^2} = 1 +\sum_s \frac{\omega_{p\,s}^{~2}}{\omega\,
\,k_z v_s}\,Z\!\left(\frac{\omega - {\Omega}_s}{k_z\,v_s}\right),
\end{equation}
with the eigenvector $(E_x, -{\rm i}\,E_x, 0)$. The former root evidently
corresponds to a right-handed circularly polarized wave, whereas the latter
root corresponds to a left-handed circularly polarized wave. 
The above two dispersion relations are essentially the same as the corresponding
fluid dispersion relations, (\ref{e4.70}) and (\ref{e4.71}), except that they explicitly
contain collisionless {\em damping}\/ at the cyclotron resonances. As before, the
damping is significant whenever the arguments of the plasma dispersion functions
are less than or of order unity. This corresponds to
\begin{equation}
\omega - |{\Omega}_e| \ltapp k_z\,v_e
\end{equation}
for the right-handed wave, and
\begin{equation}
\omega-{\Omega}_i\ltapp k_z\,v_i
\end{equation}
for the left-handed wave. 

The collisionless cyclotron damping mechanism is very similar to the
Landau damping mechanism for longitudinal waves discussed in Sect.~\ref{s6.3}. 
In this case, the resonant particles are those which gyrate about the magnetic
field with approximately the same angular frequency as the wave electric field.
On average, particles which gyrate slightly faster than the wave lose energy, whereas those which gyrate slightly slower than the wave gain
energy. In a Maxwellian distribution there are less particles in the former
class  than the latter, so there is a net transfer of energy from
the wave to the resonant particles. Note that in kinetic theory
the cyclotron resonances
possess a {\em finite}\/ width in frequency space ({\em i.e.}, the incident wave does
not have to oscillate at exactly the cyclotron frequency in order for there
to be an absorption of wave energy by the plasma), unlike in the cold plasma
model, where the resonances possess {\em zero}\/ width. 

\section{Perpendicular Wave Propagation}
Let us now consider wave propagation, through a warm plasma, {\em perpendicular}\/ 
to the equilibrium magnetic field. For perpendicular propagation, $k_z\rightarrow 0$,
and, hence, from Eq.~(\ref{e6.74}), $\xi_n\rightarrow \infty$. Making use of
the asymptotic expansions (\ref{e6.75a})--(\ref{e6.75b}), the matrix $T_{ij}$ simplifies
considerably. The dispersion relation can again be written
in the form (\ref{e6.79}), where
\begin{eqnarray}\label{e6.86a}
M_{11} &=& 1 - \sum_s\frac{\omega_{p\,s}^{~2}}{\omega}
\frac{{\rm e}^{-\lambda_s}}{\lambda_s} \!\sum_{n=-\infty}^\infty \frac{n^2\,I_n(
\lambda_s)}{\omega-n\,{\Omega}_s},\\[0.5ex]
M_{12} &=& -M_{21} = {\rm i}\sum_s \frac{\omega_{p\,s}^{~2}}{\omega}
\,\,{\rm e}^{-\lambda_s}\! \sum_{n=-\infty}^\infty \frac{n\,\left[I_n'(\lambda_s)
-I_n(\lambda_s)\right]}{\omega-n\,{\Omega}_s},\\[0.5ex]
M_{22} &=& 1 -\frac{k_\perp^{~2}\,c^2}{\omega^2} \\[0.5ex]
&&
- \sum_s \frac{\omega_{p\,s}^{~2}}{\omega} \frac{{\rm e}^{-\lambda_s}}{\lambda_s}\!
\sum_{n=-\infty}^\infty \frac{\left[n^2\,I_n(\lambda_s) +2 \,\lambda_s^{~2}\,
I_n(\lambda_s) - 2\,\lambda_s^{~2}\,I_n'(\lambda_s)\right]}{\omega-n\,{\Omega}_s},\nonumber
\\[0.5ex]
M_{33} &=& 1 - \frac{k_\perp^{~2}\,c^2}{\omega^2} - \!
\sum_s \frac{\omega_{p\,s}^{~2}}{\omega}\,\,
{\rm e}^{-\lambda_s} \!\sum_{n=-\infty}^\infty
\frac{I_n(\lambda_s)}{\omega-n\,{\Omega}_s},\label{e6.86d}
\end{eqnarray}
and $M_{13} = M_{31} = M_{23}=M_{32}=0$.
Here, 
\begin{equation}\label{e6.87}
\lambda_s = \frac{(k_\perp \rho_s)^2}{2},
\end{equation}
where $\rho_s=v_s/|{\Omega}_s|$ is the species-$s$ Larmor radius.

The first root of the dispersion relation (\ref{e6.79}) is
\begin{equation}\label{e6.88}
n_\perp^{~2}=\frac{k_\perp^{~2}\,c^2}{\omega^2} = 1 - \!
\sum_s \frac{\omega_{p\,s}^{~2}}{\omega}\,\,
{\rm e}^{-\lambda_s} \!\sum_{n=-\infty}^\infty
\frac{I_n(\lambda_s)}{\omega-n\,{\Omega}_s},
\end{equation}
with the eigenvector $(0,0,E_z)$. This dispersion relation obviously
corresponds to the electromagnetic plasma wave, or {\em ordinary}\/ mode, discussed 
in Sect.~\ref{s4.10}.
Note, however, that in a warm plasma the dispersion relation for the ordinary mode
is strongly modified by the introduction of {\em resonances}\/ (where the refractive
 index, $n_\perp$,
becomes infinite) at all the {\em harmonics}\/ of the cyclotron frequencies:
\begin{equation}
\omega_{n\,s} = n\,{\Omega}_s,
\end{equation}
where $n$ is a non-zero integer. These resonances are a {\em finite Larmor radius}\/ 
effect. In fact, they originate from the variation of the wave phase 
across a Larmor orbit. Thus, in the cold plasma limit, $\lambda_s\rightarrow 0$,
in which the Larmor radii shrink to zero, all of the resonances disappear from
the dispersion relation. In the limit in which the wave-length, $\lambda$, of
the wave is much larger than a typical Larmor radius, $\rho_s$, the
relative amplitude of the $n$th harmonic cyclotron resonance, as it
appears in the dispersion
relation (\ref{e6.88}), is approximately 
$(\rho_s/\lambda)^{|n|}$ [see Eqs.~(\ref{e6.76}) and (\ref{e6.87})]. It is clear, therefore, that
in this limit only low-order resonances [{\em i.e.}, $n\sim O(1)$] couple
strongly into the dispersion relation, and high-order resonances
({\em i.e.}, $|n| \gg 1$) can effectively be neglected. As $\lambda\rightarrow
\rho_s$, the high-order resonances become increasingly important, until,
when $\lambda\ltapp \rho_s$, all of the resonances are of approximately equal
strength. Since the ion Larmor radius is generally much larger than the
electron Larmor radius, it follows that the ion cyclotron harmonic resonances
are generally more important than the electron cyclotron harmonic resonances.

Note that the cyclotron harmonic resonances appearing in the dispersion
relation (\ref{e6.88}) are of {\em zero width}\/  in frequency space: {\em i.e.}, they are
just like the resonances which appear in the cold-plasma limit. 
Actually, this is just an artifact of the fact that the waves we are studying
propagate {\em exactly perpendicular}\/ to the equilibrium magnetic field. It is
clear from an examination of Eqs.~(\ref{e6.72}) and (\ref{e6.74}) that the cyclotron
harmonic resonances originate from the zeros of the plasma dispersion
functions. Adopting the usual rule that substantial damping takes place
whenever the arguments of the dispersion functions are less than or of
order unity, it is clear that the cyclotron harmonic resonances lead to
significant damping whenever
\begin{equation}
\omega - \omega_{n\,s} \ltapp k_z\,v_s.
\end{equation}
Thus, the cyclotron harmonic resonances possess a {\em finite}\/  width in frequency
space provided that the parallel wave-number, $k_z$, is non-zero: {\em i.e.},
provided that the wave does not propagate exactly perpendicular to the magnetic
field.

The appearance of the cyclotron harmonic resonances in a warm plasma
 is  of great practical
importance in plasma physics, since it greatly increases the number of
resonant frequencies at which waves can transfer energy to the
plasma. In magnetic fusion these resonances are routinely exploited to
heat plasmas via  externally launched electromagnetic waves. Hence, in
the fusion literature you will often come across references to
``third harmonic ion cyclotron heating'' or ``second harmonic electron
cyclotron heating.''

The other roots of the dispersion relation (\ref{e6.79}) satisfy
\begin{eqnarray}
&&\left(1 - \sum_s\frac{\omega_{p\,s}^{~2}}{\omega}
\frac{{\rm e}^{-\lambda_s}}{\lambda_s} \!\sum_{n=-\infty}^\infty \frac{n^2\,I_n(
\lambda_s)}{\omega-n\,{\Omega}_s}
\right) \left(1 -\frac{k_\perp^{~2}\,c^2}{\omega^2}\right.\nonumber\\[0.5ex]
&&\left.- 
\sum_s \frac{\omega_{p\,s}^{~2}}{\omega} \frac{{\rm e}^{-\lambda_s}}{\lambda_s}\!
\sum_{n=-\infty}^\infty \frac{\left[n^2\,I_n(\lambda_s) +2 \,\lambda_s^{~2}\,
I_n(\lambda_s) - 2\,\lambda_s^{~2}\,I_n'(\lambda_s)\right]}
{\omega-n\,{\Omega}_s}
\right)\nonumber\\[0.5ex]
&&= \left(\sum_s \frac{\omega_{p\,s}^{~2}}{\omega}
\,\,{\rm e}^{-\lambda_s}\! \sum_{n=-\infty}^\infty \frac{n\,\left[I_n'(\lambda_s)
-I_n(\lambda_s)\right]}{\omega-n\,{\Omega}_s}\right)^2,\label{e6.91}
\end{eqnarray}
with the eigenvector $(E_x, E_y, 0)$. In the cold plasma limit, $\lambda_s\rightarrow
0$, this dispersion relation reduces to that of the {\em extraordinary}\/  mode 
discussed in Sect.~\ref{s4.10}. This mode, for which $\lambda_s\ll 1$, unless the
plasma possesses a thermal velocity approaching the velocity of light, is little
affected by thermal effects, except close to the cyclotron harmonic
resonances, $\omega=\omega_{n\,s}$, where small thermal corrections are important
because of the smallness of the denominators in the above dispersion relation.

However, another mode also exists. In fact, if we look for a mode with a
phase velocity much less than the velocity of light ({\em i.e.}, 
$c^2\,k_\perp^2/\omega^2\gg 1$) then it is clear from (\ref{e6.86a})--(\ref{e6.86d})  that
the dispersion relation is approximately
\begin{equation}\label{e6.92}
1 - \sum_s\frac{\omega_{p\,s}^{~2}}{\omega}
\frac{{\rm e}^{-\lambda_s}}{\lambda_s} \!\sum_{n=-\infty}^\infty \frac{n^2\,I_n(
\lambda_s)}{\omega-n\,{\Omega}_s} = 0,
\end{equation}
and the associated eigenvector is $(E_x, 0, 0)$. The new waves, which
are called {\em Bernstein waves}\/ (after I.B.~Bernstein, who first
discovered them), are clearly slowly propagating, 
longitudinal, electrostatic waves.

\begin{figure}
\epsfysize=4in
\centerline{\epsffile{Chapter06/bern1.eps}}
\caption{\em Dispersion relation for electron Bernstein waves in a warm plasma.}\label{f37}
\end{figure}

Let us consider  electron Bernstein waves, for the sake of definiteness. 
Neglecting the contribution of the ions, which is reasonable provided that
the wave frequencies are sufficiently high, the dispersion relation (\ref{e6.92})
reduces to 
\begin{equation}
1 - \frac{\omega_{p}^{~2}}{\omega}
\frac{{\rm e}^{-\lambda}}{\lambda} \!\sum_{n=-\infty}^\infty \frac{n^2\,I_n(
\lambda)}{\omega-n\,{\Omega}} = 0,
\end{equation}
where the subscript $s$ is dropped, since it is understood that all
quantities relate to electrons. In the limit  $\lambda\rightarrow 0$ (with
$\omega\neq n\,{\Omega}$), only the $n=\pm 1$ terms survive in the
above expression. In fact, since $I_{\pm 1}(\lambda)/\lambda \rightarrow
1/2$ as $\lambda\rightarrow 0$, the dispersion relation yields
\begin{equation}
\omega^2\rightarrow \omega_p^{~2} + {\Omega}^2.
\end{equation}
It follows that there is a Bernstein wave whose frequency asymptotes
to the upper hybrid frequency [see Sect.~\ref{s4.10}] in the limit $k_\perp
\rightarrow 0$. For other non-zero values of $n$, we have $I_n(\lambda)/\lambda
\rightarrow 0$ as $\lambda\rightarrow 0$. However, a solution to Eq.~(\ref{e6.92}) can
be obtained if $\omega\rightarrow n\,{\Omega}$ at the same time. Similarly,
as $\lambda\rightarrow\infty$ we have ${\rm e}^{-\lambda}\,I_n(\lambda)\rightarrow
0$. In this case, a solution can only be obtained if $\omega\rightarrow
n\,{\Omega}$, for some $n$, at the same time. The complete solution to
Eq.~(\ref{e6.92}) is sketched in Fig.~\ref{f37}, for the case where the upper
hybrid frequency lies between $2\,|{\Omega}|$ and $3\,|{\Omega}|$. 
In fact, wherever the upper hybrid frequency lies, the Bernstein modes above
and below it behave like those in the diagram.

At small values of $k_\perp$, the phase velocity becomes large, and
it is no longer legitimate to neglect the extraordinary mode. A more detailed
examination of the complete dispersion relation shows that the extraordinary mode
and the Bernstein mode cross over near the harmonics of the cyclotron frequency
to give the pattern shown in Fig.~\ref{f38}. Here, the dashed line shows the cold
plasma extraordinary mode. 

In a lower frequency range, a similar phenomena occurs at the
harmonics of the ion cyclotron frequency, producing ion Bernstein waves, with
somewhat similar properties to electron Bernstein waves. Note, however, that
whilst the ion contribution to the dispersion relation can be neglected for
high-frequency waves, the electron contribution cannot be neglected
for low frequencies, so there is not a complete symmetry between the
two types of Bernstein waves.

\begin{figure}
\epsfysize=4in
\centerline{\epsffile{Chapter06/bern2.eps}}
\caption{\em Dispersion relation for electron Bernstein waves in a warm plasma. The dashed
line indicates the cold plasma extraordinary mode.}\label{f38}
\end{figure}